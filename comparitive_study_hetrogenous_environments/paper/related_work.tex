To the best of our knowledge, the extent of the related work comparing the performance assurance activities carried out in the physical and virtual environments is limited. In this section, the related work we discuss use statistical techniques to detect performance regression on virtual environments only. 

\subsection{Evaluation of VM overheads}

Kraft \textit{et al.} \cite{kraft2011io} discuss the issues relative to performance modeling of disk I/O. They examine the performance degradation of disk request response time by recommending a trace-driven approach. Kraft \textit{et al.} \cite{kraft2011io} emphasize on the latencies existing in virtual machines request for disc IO due to increment in time associated with request queues. 

Aravind \textit{et al.} \cite{menon2005diagnosing} audit the performance overheads in Xen virtual machines. They uncover the origins of overheads that might exist in the network I/O causing a peculiar system behavior. However, there study is limited to Xen virtual machine only while mainly focusing on network related performance overheads.


Previous literature has its limitations as most of the performance regression testing and regression modelling  is performed in virtual environments. Through our work we validate the usage of virtual environments in the field of performance engineering. 
\subsection{Performance testing} 
\subsubsection{Metric Correlation}
Foo\textit{ et al.}\cite{foo2010mining} discuss the limitations of manually detecting performance regression in large scale software systems. They propose an approach to automatically detect performance regression based on underlying through data mining techniques. Foo\textit{ et al.} use association rule between to find out metric correlation. They use the deviation in the association rules to detect performance regression.

Malik\textit{ et al.}\cite{haroon} derive "Performance Signature" based on supervised and unsupervised learning to detect performance anomalies. They use Principal Component Analysis as their unsupervised learning technique if the past tests are not marked pass or fail. PCA is used to to examine the relationships between different performance metrics.

Part of our approach is also based on the relationship between the performance metrics generated in both the environments. We look at the individual correlation values between performance metrics and their correlation with the load.


\subsubsection{Performance modeling}

Shang\textit{ et al.} \cite{Shang:2015:ADP:2668930.2688052} came up with a methodology to counter the approach of including only a limited number of performance metrics for the performance regression models. They recommend to use a an automatic clustering technique in order to select a subset of performance metrics out of the entire set of metrics. Ensued by building peformance models for each cluster. These models engulf the relationships of performance counters within each cluster. Shang\textit{ et al.} \cite{Shang:2015:ADP:2668930.2688052} also demonstrated that their approach is applicable to a system with injected performance regression. We use the same technique in our study to inject performance regression in the target system nonetheless the limitation of their study to perform their experiments in a virtual environment persists.

Belonging to the same background in the domain of performance engineering is pair-wise analysis. Nguyen \textit{et al.} \cite{Nguyen:2012:ADP:2188286.2188344} introduce the concept of using control charts in order to detect performance regression, providing a solution for keeping numerous counters, only to make the tasks like book keeping and data analysis tedious for the practitioners. Control charts use a predefined threshold to detect performance anomalies. However control charts assume that the output follow a uni-modal distribution which an inappropriate assumption for the performance load. Nguyen \textit{ et al.} propose an approach to scale the metrics accordingly. However, the experiments are only carried out on the virtual machines in contrast to our approach.

Model-based approached use the target counters(e.g. DISC I/O and CPU Utilization) to build models and these models are then used to detect performance regression int he system. What make the model-based approach celebrated is the inclusion and comparison of numerous performance counters at the same time. 
Xiong \textit{et al.} \cite{xiong2013vperfguard} proposed a model-driven approach to sofdetect software performance regression. The devised framework called \textit{vPerfGuard} helps in detecting performance anomalies in a cloud-environment. Xiong \textit{et al.} \cite{xiong2013vperfguard} only used virtual environments to test their framework, therefor our study is crucial to such state of the art research as it lays down the platform for putting confidence in such performance assurance practices.
Jiang \textit{et al. }\cite{jiang2011system} used an improved least square regression models to detect system faults.
Cohen \textit{et al. }\cite{cohen2004correlating} adapted an approach that includes fabricating probabilistic model, e.g. Tree-Augmented Bayesian Networks, to examine the causes that target the changes in the system's response time. Cohen \textit{et al. }\cite{Cohen:2005:CIC:1095810.1095821} also proposed that system faults can be detected by building statistical models based on performance metrics. The works of Cohen \textit{et al: }'s \cite{cohen2004correlating} \cite{Cohen:2005:CIC:1095810.1095821} were improved by Bodik \textit{et al.} \cite{bodik2008hilighter} by using logistic regression models.

The above literatures, in order to find results of performance testing, take the aid of performance modeling 

