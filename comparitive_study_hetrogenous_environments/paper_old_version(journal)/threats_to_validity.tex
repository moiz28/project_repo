This sections discusses the threats to our validity.
\subsection{External validity}
We chose two subject systems, CloudStore and DS2 for our study and two virtual environments, VirtualBox and VMware. All of the former mentioned entities have a credible history when it comes to the domain of software performance activities. \cite{5306331} \cite{Nguyen:2012:ADP:2188286.2188344}. Also, we made sure that our virtual environment is set up exactly the same as our physical environment by keeping a constant checks aided by scripts. Having said that, this study can be boosted by additional subject systems being tested in other types of virtual environment. 

\subsection{Internal Validity}

All of our models are dependent on the performance metrics' accuracy. Which means if the load on the server is beyond the capacity of the system to handle and builds up a queue, there is a possibility of noise sneaking in the recording process of the performance metrics. 
The scaling approach we have adopted assumes that the for a particular metric, there exists alpha and beta values. However, if the metric value is constant it not possible to have an alpha and beta value associated with it in the model. Hence looking for an alternate scaling process is necessary for such a performance metric.
We build performance regression models to compare the metrics from both of our environments. This models are accurate if there exists a high number of records for the performance counters. Additionally, we also assume that none of our dependent variable is correlated to the independent variable or vice-versa. 

\subsection{Construct Validity}
We compared our distributions using Mann-Whitney \textit{U} Tests. As this test is commonly used to compare two distributions, other tests like T-test can also be used for this purpose. 
We used R's \textit{step()} function to build our generalized linear regression models. This function automatically selects the variables contributing most to the models however the models might be different if they are only based on p-values. To address this, we plan to build models with a predefined threshold for the p-values.
