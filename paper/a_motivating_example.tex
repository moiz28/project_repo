After presenting the background and challenges in the previous section, we illustrate an example of conventional performance testing activity.

Ben is a performance engineer who is working on a large-scale distributed software system. This system is capable of generating thousands of web requests and serving a number of clients all around the globe. Before the software is deployed, or even after a modification, it is Ben's responsibility to investigate the performance of the software versus the performance benchmarks.

A classic performance testing scenario would include Ben using a virtual machine to set up the desired environment. A virtual machine is a complete segregated and secure copy of the underlying physical architecture \cite{sugerman2001virtualizing}. It can be tailored according the needs of required software hence saving resources. This is followed by deploying the system in a real world environment. He tests the system with a predefined load profile (for e.g. a set of request for logging in, browsing and logging out with x amount of users in y time) which is used to exercise the system extensively. During the testing phase, various counters are recorded. He will be comparing the results from a previously passed test in the virtual environment succeeded by the analysis of the results. \cite{5306331} Or alternatively, exercising the older version of the system with the same load and then quantitatively comparing the counters to examine any signs of performance regression \cite{sugerman2001virtualizing}.

Ben investigates if there exists performance regression by selecting counters based on his experience and intuition for e.g. \textit{disk reads/second}. Ben choses a model based approach, one of the techniques used to detect performance regression as discussed by \textit{Shang et al.} \cite{Shang:2015:ADP:2668930.2688052}. Through this approach, Ben will predict the load and then compare it to the actual load. Subsequently, he considers the load as the dependent variable and the selected performance counter as the independent variable. He uses his chosen variables to build a linear regression model. In this example, Ben uses his model for the prediction of the web requests/sec, as the load, for the system under test. He will observe if there are any unexpected deviance between the predicted and the actual load. This will be followed by comparing the errors to a predefined acceptable threshold. If he concludes that the system is not performing up to the mark, Ben will conclude that the system is exhibiting performance regression. 

Ben did not discover any performance regression and proceeds to pass this system, allowing it to be deployed in the dedicated environment. However, the performance metrics generated by the physical environment are contradictory to what they were in the virtual environment. The system's performance metrics for CPU utilization are different than expectations. Ben overlooked these as he only chose disk reads/sec for his regression model. Further investigation showed that if 50\% of the CPU is dedicated to the virtual machine, this will be considered 100\% by the performance counters in the virtual environment thus mismatching the behavior of the software system in the physical environment where it has access to 100\% of the underlying physical architecture. \cite{vmwareCPU}

The above mentioned example, we note that a system passing a performance test in the virtual environment is not bound to follow the same course in the physical environment. Ben, according to his experience, build his performance model based on his chosen performance metrics because of which he failed to take into account into the deviation of the metric responsible for CPU utilization. 


