\documentclass[review]{elsarticle}

\usepackage{lineno,hyperref}
%\modulolinenumbers[5]
\usepackage{ifpdf}
\usepackage[cmex10]{amsmath}
\usepackage{wrapfig}
\usepackage{multirow}
\usepackage{mdwmath}
\usepackage{mdwtab}
\usepackage{algorithmic}
\usepackage{times}
%\usepackage{url}
\usepackage{booktabs}
%\usepackage[tight,footnotesize]{subfigure}
\usepackage{fancybox}
\usepackage{color}
%\usepackage{array}
%\usepackage{balance}
%\usepackage{epstopdf}
\usepackage{array}
\usepackage{fixltx2e}
\usepackage{stfloats}
%\usepackage{epstopdf}
\usepackage{pdfpages}
\usepackage{amssymb}
%\usepackage{wrapfig}
%\usepackage{multirow}
\usepackage{graphicx}
\usepackage{algorithm}
%\usepackage{times}
\usepackage{url}
\usepackage{varwidth}
%\usepackage{booktabs}
%\usepackage{fancybox}
%\usepackage{color}
%\usepackage{array}
\usepackage{subfigure}
\usepackage{balance}
\usepackage{epstopdf}
%\usepackage{array}
\usepackage{xspace}
\usepackage{makecell}
\usepackage{caption}
\usepackage{pdflscape}
\usepackage{afterpage}
\usepackage{capt-of}
\usepackage{graphicx}
\usepackage{graphics}


%\usepackage{siunitx}

\journal{TBD}


%% Harvard
\bibliographystyle{model2-names.bst}%\biboptions{authoryear}

\newcommand{\conclusionbox}[1]{%
	\vspace{2mm}
	\framebox[0.45\textwidth][c]{%
		\parbox[b]{0.42\textwidth}{%
			{\it #1}
		}
	}
	\vspace{2mm}
}

%\newcommand{\rqi}{\textbf{}}
%\newcommand{\rqii}{\textbf{}}
%\newcommand{\rqiii}{\textbf{}}


\begin{document}
	
	\begin{frontmatter}
		
		\title{A Study on the Reliability of The Performance Tests Done in Virtual Environment}
		%\tnotetext[mytitlenote]{Fully documented templates are available in the elsarticle package on %\href{http://www.ctan.org/tex-archive/macros/latex/contrib/elsarticle}{CTAN}.}
		
		%% Group authors per affiliation:
		%\author{Elsevier\fnref{myfootnote}}
		%\address{Radarweg 29, Amsterdam}
		%\fntext[myfootnote]{Since 1880.}
		\author{Muhammad Moiz Arif, Weiyi Shang, Emad Shihab}%\fnref{myfootnote}
		\address{Concordia University, Montreal, Quebec}
		%\fntext[myfootnote]{Since 1880.}
		
		%% or include affiliations in footnotes:
		%\author[mymainaddress,mysecondaryaddress]
		%{Elsevier Inc}
		%\ead[url]{www.elsevier.com}
		
		%\author[mysecondaryaddress]{Global Customer Service\corref{mycorrespondingauthor}}
		%\cortext[mycorrespondingauthor]{Corresponding author}
		%\ead{support@elsevier.com}
		
		%\address[mymainaddress]{1600 John F Kennedy Boulevard, Philadelphia}
		%\address[mysecondaryaddress]{360 Park Avenue South, New York}
		
		\begin{abstract}
			Performance testing plays an indispensable role for large software systems. Performance issues cause failures of these systems in the field more often than feature bugs. Such performance tests typically require large amounts of resources, such as long running time on performance testing environments. Making it worse, the ever evolving field environment requires frequently updating the performance testing environment. To address such challenges, virtual machines (VMs) are widely exploited to provide a flexible and less costly environments for performance tests. However, the use of VMs may introduce extra overheads (e.g., a higher than expected memory utilization) to the environment, leading to un-realistic performance testing results. Yet, there exists no study that qualifies such overhead in the context of performance testing activities, or investigates the impact of performance testing results. 
			
			In this paper, we perform a case study on two open source system, i.e.,Dell DVD and Cloud Store, to measure the impact of leveraging VMs for performance testing activities. In particular, we conduct two performance testing activities, including performance modeling and performance regression detection. The two activities are conducted in both environments that consists of physical machines or VMs. \textcolor{red}{We find that the performance testing results from the VMs may not be the same as the physical machines.}
			
			(insert implications here once we have the results)
			
		\end{abstract}
		
		\begin{keyword}
			Performance engineering, 
			Performance testing
			Performance modeling
			Performance analysis            
		\end{keyword}
		
	\end{frontmatter}
	
	
	%\linenumbers
	
	 \section{Introduction}
	 \label{sec:introduction}
	 % -*- root: cuthesis_masters.tex -*-  

At the core of any software system is the software team who develop it...

\section{Research Hypothesis}

\conclusionbox{}  

\section{Thesis Overview}

\section{Thesis Contributions}
	
	 \section{Background and Related Work}
	 \label{sec:related_work}
	 To the best of our knowledge, the extent of the related work comparing the performance assurance activities carried out in the physical and virtual environments is limited. In this section, the related work we discuss use statistical techniques to detect performance regression on virtual environments only. 

\subsection{VM overheads}

Kraft \textit{et al.} \cite{kraft2011io} discuss the issues relative to performance modeling of disk I/O. They examine the performance degradation of disk request response time by recommending a trace-driven approach. Kraft \textit{et al.} \cite{kraft2011io} emphasize on the latencies existing in virtual machines request for disc IO due to increment in time associated with request queues. 

Aravind \textit{et al.} \cite{menon2005diagnosing} audit the performance overheads in Xen virtual machines. They uncover the origins of overheads that might exist in the network I/O causing a peculiar system behavior. However, there study is limited to Xen virtual machine only while mainly focusing on network related performance overheads.

\subsection{Performance regression detection} 

Shang\textit{ et al.} \cite{Shang:2015:ADP:2668930.2688052} came up with a methodology to counter the approach of including only a limited number of performance metrics for the performance regression models. They recommend to use a an automatic clustering technique in order to select a subset of performance metrics out of the entire set of metrics. Ensued by building peformance models for each cluster. These models engulf the relationships of performance counters within each cluster. Shang\textit{ et al.} \cite{Shang:2015:ADP:2668930.2688052} also demonstrated that their approach is applicable to a system with injected performance regression. We use the same technique in our study to inject performance regression in the target system nonetheless the limitation of their study to perform their experiments in a virtual environment persists.

Belonging to the same background in the domain of performance engineering is pair-wise analysis. Nguyen \textit{et al.} \cite{Nguyen:2012:ADP:2188286.2188344} introduce the concept of using control charts in order to detect performance regression, providing a solution for keeping numerous counters, only to make the tasks like book keeping and data analysis tedious for the practitioners. Control charts use a predefined threshold to detect performance anomalies. However control charts assume that the output follow a uni-modal distribution which an inappropriate assumption for the performance load. Nguyen \textit{ et al.} propose an approach to scale the metrics accordingly. However, the experiments are only carried out on the virtual machines in contrast to our approach.

Model-based approached use the target counters(e.g. DISC I/O and CPU Utilization) to build models and these models are then used to detect performance regression int he system. What make the model-based approach celebrated is the inclusion and comparison of numerous performance counters at the same time. 
Xiong \textit{et al.} \cite{xiong2013vperfguard} proposed a model-driven approach to sofdetect software performance regression. The devised framework called \textit{vPerfGuard} helps in detecting performance anomalies in a cloud-environment. Xiong \textit{et al.} \cite{xiong2013vperfguard} only used virtual environments to test their framework, therefor our study is crucial to such state of the art research as it lays down the platform for putting confidence in such performance assurance practices.
Jiang \textit{et al. }\cite{jiang2011system} used an improved least square regression models to detect system faults.
Cohen \textit{et al. }\cite{cohen2004correlating} adapted an approach that includes fabricating probabilistic model, e.g. Tree-Augmented Bayesian Networks, to examine the causes that target the changes in the system's response time. Cohen \textit{et al. }\cite{Cohen:2005:CIC:1095810.1095821} also proposed that system faults can be detected by building statistical models based on performance metrics. The works of Cohen \textit{et al: }'s \cite{cohen2004correlating} \cite{Cohen:2005:CIC:1095810.1095821} were improved by Bodik \textit{et al.} \cite{bodik2008hilighter} by using logistic regression models.



%Model-based investigation fabricates a set number of models for an arrangement of target execution counters (e.g., CPU and memory) and influences the models to identify execution relapses. The model-based methodology offers us some assistance with dealing with the expansive number of execution counters and aids in comparing the connections between the different counters. 
%Latest research by Xiong et al: proposes a model driven system to help with performance determination in a cloud environment. Their system manufactures models between workload counters and an chosen performance counter, such as CPU. The models can be utilized to recognize workload changes, and additionally helps with distinguishing execution bottlenecks.
%Cohen et al propose a methodology that fabricates probabilistic models, for example, Tree-Augmented Bayesian Networks, to associate framework level counters and frameworks' response time. The methodology is utilized to comprehend the reason to changes on systems' execution time. Cohen et al: propose 
%that execution counters can be utilized to assemble measurable models for framework issues. Bodik et al: use logistic regression models to enhance Cohen et al's: work
%Jiang et al: propose a methodology that ascertains the relationship between execution counters by enhancing the. Conventional Least Squares relapse models and utilizing the model to identify deficiencies in a framework. 
%Current model-based methodologies still have their limitations. Execution experts frequently select the objective performance counters in light of their experience and hunch. 
%They frequently concentrate on a little arrangement of surely understood counters (e.g., CPU and memory). Such specially appointed determination of target counters 
%may prompt the inability to watch execution relapses 
%

Previous literature has its limitations as most of the performance regression testing and regression modelling  is performed in virtual environments. Through our work we validate the usage of virtual environments in the field of performance engineering. 




	
	 \section{A Motivating Example}
	 \label{sec:a_motivating_example}
	 After presenting the background and challenges in the previous section, we illustrate an example of conventional performance testing activity.

Ben is a performance engineer who is working on a large-scale distributed software system. This system is capable of generating thousands of web requests and serving a number of clients all around the globe. Before the software is deployed, or even after a modification, it is Ben's responsibility to investigate the performance of the software versus the performance benchmarks.

A classic performance testing scenario would include Ben using a virtual machine to set up the desired environment. A virtual machine is a complete segregated and secure copy of the underlying physical architecture \cite{sugerman2001virtualizing}. It can be tailored according the needs of required software hence saving resources. This is followed by deploying the system in a real world environment. He tests the system with a predefined load profile (for e.g. a set of request for logging in, browsing and logging out with x amount of users in y time) which is used to exercise the system extensively. During the testing phase, various counters are recorded. He will be comparing the results from a previously passed test in the virtual environment succeeded by the analysis of the results. \cite{5306331} Or alternatively, exercising the older version of the system with the same load and then quantitatively comparing the counters to examine any signs of performance regression \cite{sugerman2001virtualizing}.

Ben investigates if there exists performance regression by selecting counters based on his experience and intuition for e.g. \textit{disk reads/second}. Ben choses a model based approach, one of the techniques used to detect performance regression as discussed by \textit{Shang et al.} \cite{Shang:2015:ADP:2668930.2688052}. Through this approach, Ben will predict the load and then compare it to the actual load. Subsequently, he considers the load as the dependent variable and the selected performance counter as the independent variable. He uses his chosen variables to build a linear regression model. In this example, Ben uses his model for the prediction of the web requests/sec, as the load, for the system under test. He will observe if there are any unexpected deviance between the predicted and the actual load. This will be followed by comparing the errors to a predefined acceptable threshold. If he concludes that the system is not performing up to the mark, Ben will conclude that the system is exhibiting performance regression. 

Ben did not discover any performance regression and proceeds to pass this system, allowing it to be deployed in the dedicated environment. However, the performance metrics generated by the physical environment are contradictory to what they were in the virtual environment. The system's performance metrics for CPU utilization are different than expectations. Ben overlooked these as he only chose disk reads/sec for his regression model. Further investigation showed that if 50\% of the CPU is dedicated to the virtual machine, this will be considered 100\% by the performance counters in the virtual environment thus mismatching the behavior of the software system in the physical environment where it has access to 100\% of the underlying physical architecture. \cite{vmwareCPU}

The above mentioned example, we note that a system passing a performance test in the virtual environment is not bound to follow the same course in the physical environment. Ben, according to his experience, build his performance model based on his chosen performance metrics because of which he failed to take into account into the deviation of the metric responsible for CPU utilization. 



	
	 \section{Approach}
	 \label{sec:approach}
	 %\begin{figure}[t!]
%	\includegraphics[width=1\textwidth]{approach.pdf}
%	\centering
%	\caption{Approach overview}
%    \label{fig:approach}
%\end{figure} 
%%\includepdf][pages={1}][width=\columnwidth{approach.pdf}

\begin{figure}[!t]
	\centering
	\includegraphics[width=\textwidth]{approach.pdf}
	\caption{Approach Overview}
	\captionsetup{justification=centering}
	\label{fig:Approach}
	
\end{figure}

In this section we discuss the steps in our approach. The primary intention behind our experiment was to observe the performance of the system under test in a virtual environment. Additionally, to analyze whether the metric values depict a similar behavior when compared cross-environment. Once the performance metrics are recorded and processed, we then analyze our results using graphical techniques and regression models. Figure 1 shows the synopsis of our approach.


\subsection{System Setup}
We chose Dell DVD Store (DS2) \cite{delldvd} as our test database. The DVD store is an online e-commerce web application with a multitier architecture, consisting of scripts for MySQL Server, \textit{PHP} web pages and a load driver program written in \textit{C-sharp}. \cite{Shang:2015:ADP:2668930.2688052} \cite{Nguyen:2012:ADP:2188286.2188344} \cite{Nguyen:2012:ADP:2188286.2188344}. \textit{WAMP} \cite{wamp} web application server were used to deploy and exercise the system while the database was set up on MySQL Server 5.6 \cite{mysql}

Based on TPC-W standards \cite{tpcw}, a performance benchmark originally proposed by the Transaction Processing Performance Council, our second subject system under test was also an open source application by CloudScale project \cite{cloudscaleproject} called CloudStore \cite{cloudstore}. Just like DS2, CloudStore serves as online e-commerce website. An online book store, CloudStore is used as a standard in the domain of cloud computing. CloudStore's \textit{JSP} pages were deployed on \textit{Tomcat} \cite{tomcat} and MySQL Server 5.6 \cite{mysql} was used to set up the database. The workload generator was based on scripts written for Apache Jmeter. \cite{apachejmeter}

Our system setup included three machines in a lab environment, Pentium i5 each with an 8GB of memory. To make the systems' configuration identical prior to exercising the subject system, we chose 2 cores and 3GB of memory dedicated to each environment to avoid crashes on the guest operating system. We opted for single tenancy of the guest operating system to avoid any unwanted noise. The first machine was dedicated to the database server, the second machine was dedicated to the web server and the third machine was used to run the load driver.

%\subsection{Exercising the database}
Following the set up of our subject systems on the respective servers, the systems were exercised with an aid of drivers. These drivers generated multi-type web requests and simulated real-time user behavior depending on the input parameters provided by us. We ran our performance tests for numerous hours while recording all the performance metrics generated for varying load applied on our software systems.

The load variation per run was introduced by the number of threads. A higher number of threads represented a higher number of users which would resultantly increase the number of requests per minute and vice-versa. As our study was based on exercising our systems and recording the performance metrics, and not stress testing \cite{stresstesting}, the respected limits were chosen in order to avoid the under-performance of the physical machine and system failure of the virtual machine.

%\subsection{Metrics Collection}
\subsection{Data Transformation}
The values of all the available performance metrics were monitored, recorded every 10 seconds via \textit{perfmon} \cite{windowsperfmon}\cite{perfmon}. \textit{Perfmon} is a system performance monitor used to observe and record performance metrics like CPU utilization, Memory usage and disk IOs. For the reason that our web server and database server were on different machines, we recorded two data sets from two machines monitoring the application process resulting in a data set of 56 metrics. A set of every 6 recorded entries averaged out was labeled as a \textit{single run} representing a minute. This way we had accumulated data from more than 500 runs. We used web server access logs to extract the requests that represented our load. Access logs record every web request that is received by the web server accompanied by the information such as IP addresses, URLs and timestamps \cite{weinberg2003use}. With the help of a script we grouped the web requests per minute. The two data sets were then concatenated and mapped against requests according to the timestamps.

   %\begin{itemize}
   	%\item \textit{User Time}:  	
	 %  	Time the processor takes executing in the user mode.
   	%\item \textit{Privileged time}:
   	 %  	Time the processor takes executing in the privileged mode.
   	%\item \textit{IO Read Operations/Sec}: 
   	 %  	 The rate of issuing of read I/O operations by the process.
   	%\item \textit{IO Read Bytes/Sec}:
   	 %  	The rate at which the process reads bytes from I/O operations.
   	%\item \textit{IO Write Operations/Sec}:
	 %  	   The rate of issuing of write I/O operations by the process.	
   	%\item \textit{IO Write Bytes/sec}:
	 %  	The rate at which the process writes bytes to I/O operations.
   	%\item \textit{Working Set}:
	 %  	Set of memory of pages altered latterly by the process.
   	%\item \textit{Working Set - Private}:
   	%\item \textit{Private Byte}:
	 %  	Bytes allocated to a particular process that cannot be shared.
   %\end{itemize}


\subsection{Data Cleansing and Data Analysis}
Following the data transformation, we used R \cite{R} to derive conclusions for our research questions with the help of Spearman correlation, q-q plots \cite{qqplots}; a graphical technique to detect if two sets of data belong to the same population, Mann-Whitney \textit{U} test \cite{mannwhitney}, and generalized linear regression models (GLM) \cite{SanFranciscoStateUniversity} which are discussed later in detail in section 4. Spearman correlation is used to identify the level of association between two variables while Mann-Whitney \textit{U} test  test is used to verify the hypothesis that if the data sets belong to the same population.
%calculate the absolute deviation in a dataset from the median \cite{brownftest} \textcolor{red}{or instead brown forsythe test. enough already for rq1, discuss}. 
Both these methods do not strictly assume that the ordinal data is normally distributed \cite{spearman} \cite{manutest}. Due to the size of our data, heat maps \cite{heatmaps} were used to visualize the correlations. We chose model-based analysis as they can support the automatic selection and detection of performance abnormalities between heterogeneous environments. \cite{Shang:2015:ADP:2668930.2688052}\cite{Nguyen:2012:ADP:2188286.2188344}. Contrary to the usual practice of ad-hoc selection of certain target performance metrics \cite{heger2013automated} to detect performance regression, we included exhaustive set of all 56 metrics to shape our models. For our linear regression models, we processed our data in two steps. First, we remove any metrics that showed no variance by a R script, for example a metric which has less than 20 unique values \cite{rahm2000data}. Next, we also remove the metrics which have high correlation amongst themselves to remove any \textit{multicollinearity} present \cite{mansfield1982detecting} so that the models do not over fit based on similar predictors(performance metrics) for the load. This approach will also magnify the metric which actually contribute the most to the model. The rationale behind aforementioned steps was to clean up data that may not contribute significantly to the model \cite{Shihab:2010:UIC:1852786.1852792}. 


Once our data was organized our consequent footprint was to build the GLM using the physical and the virtual environment's performance metrics. Our models were built using the load as the dependent variable and the performance metrics as the independent variable i.e. the load is dependent on the change of values of the performance metrics. with the help. We, additionally, trained and tested our models for the same environment before using it cross-environments. This served as the benchmark for lower limit of the prediction percentage error. A model tested for the same environment was validated via ten-fold cross validation \cite{Cross_Validation} \cite{kohavi1995study}. Each fold is based on a random subset of values. The model is trained on 9 folds and tested on the $10^{th}$ fold. This is done 10 times with 10 different subsets of data each time. The accuracy of our models was determined by the percentage error of the predicted load versus the actual load values which was calculated as the absolute difference of the actual and predicted load values with respect to the actual load values.

Based on the results from the GLMs we then investigate about the metrics that are contributing the most and their respective correlation values using heat maps. Heat maps allow us to graphically view, in our case, the individual metrics and their correlation values. 
	
	 \section{Case Study Results}
	 \label{sec:results}
	 The objective of our case study is a comparative analysis of the two environments and to find out if there exists software performance regression between a dedicated server and the virtual environment. To analyze this, we divided our project into three research questions. Following the set up of the subject systems and data preprocessing, we used requests per minute as our independent variable and performance metrics as our dependent variables to answer our research questions. %Firstly, we observe if our models are transferable across the three chosen environments. Secondly, if the models can accurately predict counter values cross-servers. Lastly, if defect injection will generate similar performance metrics values compared   

\subsection{\textbf{Do performance metrics from physical and virtual environments belong to the same population?}}

\textbf{Motivation}: Performance assurance activities are not only bounded by the idea that performance regression can only be identified relative to different versions of the software. A software system might also regress once it is transferred from one environment to another. According to the norm, practitioners often rely on performance metrics generated by the virtual environment prior to releasing it on a disparate environment. We address this question by comparing the metric values between the physical and virtual server. 

\textbf{Approach}: As mentioned earlier in Section 3, after setting up the subject systems, we used the number of requests as our workload for the systems. A script was written to randomly alter the workload after every minute for CloudStore and every two minutes for DS2. The script was based on an underlying assumption that every unit of change in the performance metrics there will be a consequent unit of change in the workload which eventually introduced variety in our dataset. \cite{linearregression}. Subsequently, the requests per minute were mapped against the performance metrics recorded via \textit{perfmon}.

The numerical values for requests per minute were extracted out of the web server's access logs. These requests were recorded per second and reached up to thousands in a minute, depending on the workload. We then, with the aid of a script, grouped and extracted the requests per minute for as long as the test systems were exercised. The metrics, on the contrary, were recorded every 10 seconds. A set of six records was averaged out as a minute according to the timestamps. This was called a single \textit{run}. The metrics generated for the web and database server were concatenated. Ultimately, we had a dataset consisting of 500 runs where every record represented a \textit{run}; the metrics and requests generated for a given minute. This process was carried out for both of our environments. The two datasets, to analyze any discrepancies, were then analyzed using R's q-q plots and \textit{cor} function.

\begin{figure}[thb]
	\centering
	\includegraphics[width=0.9\columnwidth]{figures/ds2_qq.pdf}
	\caption{Q-Q plots: DS2}
%	\captionsetup{justification=centering}
	\label{fig:Results Table}
\end{figure}



\begin{figure}[thb]
	\centering
	\includegraphics[width=0.9\columnwidth]{figures/cloudstore_qq.pdf}
	\caption{Q-Q plots: CloudStore}
%	\captionsetup{justification=centering}
	\label{fig:Results Table}
\end{figure}



\begin{table}[t]
	\centering
			\caption{DS2: Correlation Values}
			\label{resultRQ3}
	\begin{tabular}{c|cc}
				\textbf{Performance Metrics}   & \textbf{Cor} & \textbf{p-value}\\  
			 
				\textbf{Web Servers' User Times} &  0.08 & 0.07\\
				\textbf{DB Servers' User Times} & -0.05 & 0.30\\
				\textbf{Web Servers' IO Data Ops/Sec}   & 0.25 & 0.000 \\
				\textbf{DB Servers' IO Data Ops/Sec} & -0.14 & 0.00\\
				\textbf{Web Servers' Memory Working Set} & 0.22 & 0.00\\
				\textbf{DB Servers' Memory Working Set} & 0.46 & 0.00\\
		\end{tabular}
\end{table}
	
\begin{table}[t]
		\caption{CloudStore: Correlation Values}
		\label{resultRQ3}
		\begin{tabular}{c|cc}
			\toprule
			\textbf{Performance Metrics}   & \textbf{Cor}& \textbf{p-value}\\
			\midrule 
			\midrule 
			\textbf{Web Servers' User Times} & \ 0.01& 0.87\\
			\textbf{DB Servers' User Times} & \ 0.20 & 0.00\\
			\midrule 
			\textbf{Web Servers' IO Data Ops/Sec}   & \ 0.17& 0.00 \\
			\textbf{DB Servers' IO Data Ops/Sec} & \ 0.18& 0.00\\
			\midrule 
			\textbf{Web Servers' Memory Working Set} &\ 0.69& 0.00\\
			\textbf{DB Servers' Memory Working Set} & -0.13 & 0.00\\
			\bottomrule             
		\end{tabular}
\end{table}

%\begin{figure}[t!]
%	\centering
%	\includegraphics[width=0.7\textwidth]{private_bytes.pdf}
%	\caption{QQ-plot of Private Bytes of Physical vs. Virtual Environment}
%	\captionsetup{justification=centering}
%	\label{fig:Results Table}
%\end{figure}

%\textbf{Approach}: 
%Following our data preprocessing, we decided to use Q-Q plot to identify if there exists a similarity between the distribution of our counters for the dedicated and the virtual server. We divided our metrics into three sub-categories namely \textit{processor, IO and memory}.

\textbf{Results}: Figure 2 and 3 show the results from our q-q plots. For the sake of brevity, we chose one q-q plot to be displayed from each subset of metrics i.e. CPU, IO and memory. If the distributions of our performance metrics were similar, we should see the plots closer to the line y=x. For the sake of reference, we named this line 'Z'. \cite{Cross_Validation}. We plotted the line Z on the same axes. 

As seen in Figure 2 for the subject system DS2, the CPU user times for the web servers are closer to the line Z however the plots database servers' CPU user times are highly deviated from the same line as it not visible in the same plot. The disk IO operations/sec for the web servers gradually deviate form the line Z whereas the database server is, again, highly deviated. The same can be concluded about the memory working sets for both of our environments.
Figure 3 shows the q-q plots for CloudStore. The web servers CPU user times are not congruent with the line Z. The database server CPU user times towards the tail of the plot are closer to the line Z however they still do not follow the line Z. The disk IO operations for both servers tends to follow the line Z initally but gradually moves away. Further on, the memory working set for both of our servers, as seen, are distant to the line Z.

Our Mann-Whitney \textit{U} Tests also concluded that for each of the performance metrics selected they do not belong to the same distributions with a p-value $<$ 0.05, except the web server's user times for DS2.

Table 1 and 2 shows the Spearman correlation values between the selected performance metrics. If the correlation value is closer to 1 the metrics have a strong correlation and if the value is closer to 0 the metrics have a weak correlation. A negative correlation means that if one of the metrics is increasing in value, the other is decreasing. For both of our subject systems we observe that the correlation values are mostly positive and are closer to 0 representing weak correlations. This conclusion, however, is not applicable to the user times from both environments as the p-value $>$ 0.05.

\subsection{\textbf{Is the correlation value between load and performance metrics same cross-environments?}}

\textbf{Motivation}: Building on the conclusion from the previous research question, we next address the change in metric correlation values amongst themselves and versus the load. Our goal was to explore whether the change in correlation cross-environments is identical for both of our subject systems. This way we will be able to conclude whether a certain type of discrepancy is always present between the two environments or is it the unstable nature of the performance of subject systems in different environments. 
%Before the software is deployed in a physical environment, the performance analyst relies on the heuristics generated from the virtual environment \cite{foo2010mining}. One of the approaches to detect performance regression is to compare every metric with the previously passed performance test \cite{Shang:2015:ADP:2668930.2688052}. As discussed in our previous research question, most of the performance metrics between two environments do not belong to the similar distribution. 

%As discussed in RQ1, we discovered that our comparison of performance metrics between our physical and virtual environments produced dissimilar results. Our next step was to explore what metrics changed the most 


%As performance testing spans our from handful of hours to several days , the reliability on such exercise is remarkably critical. A recent bug fix or a modification may require a reiteration of performance tests \cite{foo2010mining}. On the contrary, This led us to our next question, that whether the performance assurance activities run on virtual environments can be duplicated.

\textbf{Approach}: We looked for the top 5 metrics which are highly correlated with the load in each of our environments. We used R's \textit{cor} function to determine the Spearman value of the aforementioned associations. 
Next, we used heat maps to highlight the set of metrics which show a significant change in correlation values amongst themselves.
The correlation values between the metrics of physical server were stored in matrix A. The correlation values between the metrics of the virtual server were store in matrix B. Matrix Z was the absolute difference between these two matrices. For example the Spearman correlation value for CloudStore's physical server between web server's User Time and database server's IO write/Bytes sec is 0.94. The correlation value for the same pair of metrics in the virtual environment is 0.26. Then the Matrix Z will record a value which is the absolute difference of the aforementioned Spearman values i.e. 0.68. If the metric value has changed significantly, according to the legend, this will be denoted as a 'hot zone' in the heatmap denoted by a lighter gradient of color.

\textbf{Results}: Figure 4 and 5 are the heatmaps for the change in correlation values amongst the metrics of two environments. For DS2, figure 4, the hot spots are mostly prevalent between the IO operations, for both the web and database server, cross-environments. This means the correlations amongst IO operations in the physical environment are not the same as the correlation between IO operations in the virtual environment. While CloudStore's heatmap, figure 5, shows that the change in correlation values is not as similar to DS2. Most of the hot spots are scattered across the heatmaps, contrary to DS2 where we can see clusters around most of the IO operations. 

We also observed a similar change in correlation values between the processor times and other metrics. This trend may not be as strong as CloudStore's heatmap for DS2, however these changes in values can be found in both of our subject system.

Table 3-6 are the top five highly correlated metrics with the load. In DS2, most of the IO operations from the web driver are highly correlated with the load in the physical environment. However, the database server is highly correlated than any of the metrics from the web driver in the virtual environment.
Table 5 and 6 shows a much similar behavior of CloudStore in both the environments.

We primarily learned that the DS2 IO operations' behavior in one environment are not similar to that of the physical environment. We also learned that the change in nature of correlation cross-environments is non-uniform.
%Figure 4 and 5 are the heatmaps for the change in correlations' values between the metrics of two environments. For DS2, figure 4, 4he hot spots are mostly prevalent between the IO operations cross-environment while CloudStore's heatmap, figure 5, shows that the change in correlation values is not as similar to DS2. Most of the hot spots are scattered across the heatmaps, contrary to DS2 where we can see clusters around most of the IO operations.



\begin{figure}[tbh]
	\centering	
	\includegraphics[width=0.5\textwidth]{figures/ds2_heatmap.pdf}
	\caption{Heatmap: DS2}
%	\captionsetup{justification=centering}
	\label{fig:Results Table}
\end{figure}

\begin{figure}[tbh]
	\centering
	{\includegraphics[width=0.5\textwidth]{figures/cloudscale_heatmap.pdf}}
	\caption{Heatmap: CloudStore}
	%\captionsetup{justification=centering}
	\label{fig:Results Table}
\end{figure}


\begin{table}[tbh]
		\centering
		\caption{DS2: Top 5 highly correlated metrics with load (Physical Server)}
		\label{resultRQ3}
		\begin{tabular}{c}
			\toprule
			1. Web Server IO Other Operations/sec \\
			2. Web Server IO Other Bytes/sec \\
			3. Web Server IO Write Operations/sec \\
			4. Web Server IO Data Operations/sec \\
			5. Web Server IO Data Bytes/sec \\
			
			\bottomrule             
		\end{tabular}
\end{table}

\begin{table}[tbh]
	\centering
		\caption{DS2: Top 5 highly correlated metrics with load (Virtual Server)}
		\label{resultRQ3}
		\begin{tabular}{c}
			\toprule
			1. Database Server Handle Count \\
			2. Database Server Working Set-Peak \\
			3. Database Server Pool Paged Bytes \\
			4. Database Server IO Other Operations/sec \\
			5. Database Server Page File Bytes Peak \\
			\bottomrule             
		\end{tabular}
\end{table}

\begin{table}[tbh]
	\centering
		\caption{CloudStore: Top 5 highly correlated metrics with load (Physical Server)}
		\label{resultRQ3}
		\begin{tabular}{c}
			\toprule
			1. Database Server IO Other Bytes/sec  \\
			2. Database Server IO Read Operations/sec   \\
			3. Database Server IO Read Bytes/sec \\
			4. Database Server IO Data Operations/sec    \\
			5. Database Server IO Write Operations/sec \\
			\bottomrule             
		\end{tabular}
\end{table}


\begin{table}[tbh]
	\centering
		\caption{CloudStore: Top 5 highly correlated metrics with load (Virtual Server)}
		\label{resultRQ3}
		\begin{tabular}{c}
			\toprule
			1. Database Server IO Other Operations/sec  \\
			2. Database Server IO Write Operations/sec   \\
			3. Database Server IO Write Bytes/sec \\
			4. Database Server IO Data Bytes/sec    \\
			5. Database Server IO Read Bytes/sec \\
			\bottomrule             
		\end{tabular}
\end{table}


\subsection{\textbf{Do performance metrics from different environments impact performance modeling?}}

\textbf{Motivation}: As discussed in earlier work \cite{Shang:2015:ADP:2668930.2688052} \cite{Nguyen:2012:ADP:2188286.2188344}, performance tests require a large dedication of resources as it is carried out just before the system is on the brink of deployment. This gives insufficient time to the performance engineers, leaving them with minimal resources. As a result they leverage on the performance assurance activities carried out in the virtual environment. The motivation behind this research question is to investigate whether the performance models generated from one environment can be applied and held representatives of the other. In essence, this will help the practitioners conclude the reliability of the performance activities carried out in the virtual environment. 
% In practice, this may or may not be appropriate to assume. This also spawns the concept of including the entire set of performance metrics for analysis, which is slipshod and ineffectual. 

\textbf{Approach}: We first validated our models with an environment using 10-fold cross validation. \textit{K}-fold cross validation divides the data into \textit{k} parts which are known as folds. The model is trained on \textit{k}-1 folds and tested on the \textit{k}$^{th}$ fold. This process is iteratively repeated \textit{k} times \cite{10foldcross} \cite{kohavi1995study}.


We partitioned our results into two segments; the explanatory and the predictive part for our models, trailed by applying our model to predict load cross-environments. The explanatory part calculates the percentage of deviance explained by our models i.e. the fit of the model while the predictive part explains the error percentage between the actual and predicted load values. Both of these parts were addressed by building generalized linear models that only incorporated the metrics which were selected via R \textit{stepwise stepwise} or commonly knows as \textit{stepwise} function, from the complete set of metrics from the dataset.

\textit{Explanatory Power}

After removing any outliers for both of our subject systems, we built our GLM, initially, using all the physical server's metrics. We trained and tested our model on the same server i.e. physical. We then reduced our model, iteratively, using only the metrics that our contributing the most to the model. This was achieved using R's \textit{stepwise} function. The \textit{stepwise} function adds the independent variables one by one to the model to exclude any metrics that not contributing to the model \cite{RInAction}. Once the metrics were automatically selected out on the physical server, we used R's \textit{ANOVA}, or \textit{analysis of deviance} to rank the metrics according to their deviance value. Higher the deviance value for a given performance metric in the model, higher the contribution of the performance metric to the model. This approach was similarly applied to the set of metrics from the virtual server to build a generalized linear model. R's \textit{deviancepercentage} was used to determine the explanatory part or fit of our model. It is used to calculate the percentage of deviance for a given GLM model.

As a result, we had two scenarios per subject system:
\begin{description}
	\item[$\bullet$] Trained on physical, tested on physical.
	\item[$\bullet$] Trained on virtual, tested on virtual.
\end{description} 

\textit{Predictive Power}
%\begin{description}
	
	%\paragraph{Cross-Environment}
	
	The notion behind our predictive approach was to observe the percentage error between the actual and the predicted load. Our first set of predicted values were based on the model trained on the virtual environment's performance metrics and tested on the physical's server metrics. For the dataset of the virtual metrics, we wrote a script in R to remove the metrics that indicated practically zero fluctuation because else they will not have any impact on the GLM. This was trailed by the step to remove any highly correlated metrics and using \textit{stepwise} for every fold \cite{Shihab:2010:UIC:1852786.1852792}. 
	%From the statistical techniques available to validate our model, we used 10-fold cross \cite{kohavi1995study} \cite{10foldcross}.
	We used mean absolute percentage error (\textit{MAPE}) to measure the error between the predicted and actual load values \cite{mape}. MAPE serves as the percentage measurement of the deviance of our forecasted values from our real values which makes it easier to interpret our results. If the error is 3, we say the forecasted value are off by 3\%. 
	
	Based on our results, we then also scaled the virtual environment's metrics according to the physical environment. Our approach for scaling was based on by Nguyen \textit{et al.} previous work \cite{Nguyen:2012:ADP:2188286.2188344}:
	%We addressed the presence of a high percentage error by adjusting our virtual environment's metrics to the phsyiscal environment according to the following equation:
	%As explained in RQ1, we perceived that our distributions generated by the performance metrics in our environments are not the same. We used R's \textit{predict} to predict our desired set of load, based on the training and testing mentioned in the previous subsections. \textcolor{red}{we used 1-fold cross validation here, necesaary to mention?}
	
	\begin{equation*}
	Load_{physical}= \alpha_{physical} \times Counter_{physical} + \beta_{physical}
	\end{equation*}
	
	\begin{equation*}
	Load_{virtual}= \alpha_{virtual} \times Counter_{virtual} + \beta_{virtual}
	\end{equation*}
	
	\begin{equation}
	V_{scaledmetric} = (\frac{Load_{virtual}-\beta_{virtual}}{\alpha_{virtual}})\times\alpha_{physical}+\beta_{physical}
	\end{equation}
	
To scale our metrics from the virtual environment, we first build a GLM with the selected counter and the load in the each of the environment. The selected counter was chosen on the basis of counters selected by \textit{stepwise}, trained in virtual and tested in phyiscal environment. For every GLM there exists an intercept and a gradient value. We used the aforementioned values and applied them to the metric from the virtual environment as explained by equation 1.
We also assumed that for each of the metric there will be a gradient and intercept value.
 %The selection of metrics was based on their presence in the GLMs that were trained in the virtual environment and tested in the physical environment.


 
\textbf{Results}: Tables 7-10 show the results of our approach. In Table 7, for our subject system DS2, we see that the statistically significant metrics for the GLM from both of our environments are different. The significant of the performance metrics in an GLM is environment dependent. Due to automatic selection of metrics we see a high fit and lower \textit{MAPE} values for our environments. Table 8 shows us the results when the model from the virtual was applied to predict load for the physical server and vice-versa. We observe a an immensely high \textit{MAPE} value for the former while physical-virtual prediction is almost 50\%. This means that the predicted and actual values of the workload have a mean absolute error percentage of almost 50\%. When the same set of metrics from the virtual environment were scaled the MAPE value for DS2 reduces drastically.


Table 9 shows us the results from CloudStore. Again, the ranks of the metrics that prove to contribute most to the model are not the same compared to both environments. After the data validation, we see a \textit{MAPE} value of almost 16\% for physical server and almost 5\% for virtual. As the GLM was based on automatic selection via \textit{stepwise} regression, we see a high percentage fit for our models. 
Table 10 shows us the results of cross application of models. A model that was trained on virtual server and tested to predict the workload values for the physical server was off by almost 29\% whereas from physical applied to virtual it was off by 293\%. Similarly, when the metrics were scaled, instead of a an expected decrease in the MAPE value we see that the MAPE jumps from 28.95\% to 276.04\%.

Not only the \textit{MAPE} values are high when applying models cross environments, they are inconsistent between two environments. We conclude that for both of our subject systems, the models can not directly be applied to predict load for the other environment. We also observed that scaling according to the physical environment may or may not work. This is dependent on the selection of metrics in both of the environments. What might be significant in on of the models may not be applicable to the other model in another environment.




%We obtained the former equations with the models by training and testing on the respective environments. These equations are derived from the previous work of \cite{Nguyen:2012:ADP:2188286.2188344} Once, the model was constructed we extracted the \(\alpha\) and \(\beta\) values from each model and applied them to scale the virtual environment's metrics accordingly. Equation (1) shows the derived equation through which we scaled the metrics during the process of scaling the counters. 


	%As shown in Table 3, we concluded that most of the contributions to our models are by \textit{DISC IO} and the \textit{CPU}. Inclusion of the aforementioned set of performance metrics gave us a fit percentage closer to the models which included all of the performance metrics. Additionally, the models that were based on the statistically significant metrics ranked by the other environment showed a MAPE value of no more than 7\%. This helped us conclude the metrics that are contributing the most between are environments are overlapping.  
	%The unscaled transfer of metrics generated a \textit{MAPE} value \textbf{40\%} more than that of scaled counters. Our approach successfully identified that the performance metrics generated in the virtual environment can not be taken as the exact reflection of the performance of the system in the physical environment. We also deduced the diversity in environment does impact the regression models and the performance assurance activities. One of the solution to counters this is scaling, however, the error still remains relatively high as shown in Table 2.
	
	%\afterpage{%
	%    \clearpage% Flush earlier floats (otherwise order might not be correct)
	%    \thispagestyle{empty}% empty page style (?)
	%    \begin{landscape}% Landscape page
	%        \centering % Center table
	%        \begin{tabular}{llll}
	%          \begin{table}[thb!]
	%    \begin{center}
	%    \caption{Results}
	%    \label{tab:project_results}
	%            \begin{tabular}{c||cccc}
	%            \toprule
	%            \textbf{Training-Testing}   & \textbf{Physical - Physical} & \textbf{Physical - Virtual} & \textbf{Virtual - Virtual} & \textbf{Virtual - Physical} \\  
	%            \midrule 
	%            \textbf{Ranking}     &       1. IO Read Bytes/sec & 1.IO Read Bytes/sec & 1.User Time  & 1.IO Read Operations/sec \\
	%             &                               2. IO Data Operations/sec & 2. IO Data Operations/sec & 2. IO Read Bytes/sec & 2. IO Read Operations/sec \\
	%             & 3.IO Read Operations/sec & 3.IO Other Bytes/sec & 3. IO Data Operations/sec & 3. IO Other Bytes/sec \\
	%             & 4. IO Other Bytes/sec & 4. IO Read Operations/sec & 4. IO Other Bytes/sec & 4. IO Write Bytes/sec\\
	%             & 5.IO Data Bytes/sec & & 5.Elapsed Time &\\
	%             & 6.Page Faults/sec & & 6.IO Read Operations/sec &\\
	%             & & & 7. Thread Count &\\
	%             & & & 8. IO Write Bytes/sec &\\
	%             & & & 9.IO Write Operations/sec &\\
	%             & & & 10. Working Set &\\
	%             \midrule
	%             \textbf{Fit} & All metrics: 64.1\% & All metrics: 81\% & All metrics: 81\% & All metrics: 63.4\%\\
	%             & Statistically Significant Metrics: 63.5\% & Statistically Significant Metrics: 81\% & Statistically Significant Metrics: 81\% & Statistically Significant Metrics: 63.2\% \\
	%             \midrule
	%             \textbf{MAPE} & 6.17\% & 3.81\% & 3.89\% & 6.14\% \\
	%            \bottomrule             
	%        \end{tabular}
	%    \end{center}
	%\end{table}
	%\\
	%        \end{tabular}
	%        \captionof{table}{Table caption}% Add 'table' caption
	%    \end{landscape}
	%    \clearpage% Flush page
	%}
	
%\end{description}
	
%\begin{landscape}
	\begin{table}[tbh]
		\centering
			\caption{DS2: Ranking of Performance Counters and Prediction errors}
			\label{tab:resultRQ1}
			\resizebox{\columnwidth}{!}{%
			\begin{tabular}{c||cc}
				\toprule
				\textbf{Training-Testing} & \textbf{Physical - Physical} & \textbf{Virtual - Virtual} \\  
				\midrule 
				\textbf{Ranking}     & 1. Web Server Privileged Time & 1. Web Server IO Other Bytes/sec \\
				\\
				& 2. Database Server User Time & 2. Web Server Page Faults/sec \\
				\\
				& 3. Database Server IO Read Bytes/sec & 3.Database Server Working Set-Peak\\
				\\
				& 4. Web Server Page Faults/sec & 4. Web Server Handle Count \\
				\\
				& 5. Database Server Privileged Time & 5. Database Server IO Data Operations Bytes/sec \\
				\\
				& 6. Database Server IO Write Operations Bytes/sec &\\
				\\
				& 7. Database Server Pool Paged Bytes &\\
				\\
				& 7. Web Server Working Set-Private &\\
				\\
				\midrule
				\textbf{Fit} &  85.80\% &  67.10\% \\
				\midrule
				\textbf{MAPE} & 7.02\% & 10.49\% \\
				\bottomrule             
		\end{tabular}%
	}
	\end{table}
%\end{landscape}


	\begin{table}[tbh]
		\centering
			\caption{DS2: Prediction errors cross-environments}
			\label{resultRQ3}
			\begin{tabular}{c|c}
				\toprule
				\textbf{Training - Testing}   & \textbf{MAPE}\\  
				\midrule 
				Physical-Virtual       & 49.97\% \\
				Virtual - Physical        & 6033.16\% \\
				Virtual - Physical \textbf{(after scaling)}      & 16.04\% \\
				\bottomrule             
			\end{tabular}
	\end{table}
	
	
	\begin{table}[tbh]
		\centering
			\caption{CloudStore: Ranking of Performance Counters and Prediction errors}
			\label{tab:resultRQ1}
			\resizebox{\columnwidth}{!}{%
				\begin{tabular}{c||cc}
					\toprule
					\textbf{Training-Testing} & \textbf{Physical - Physical} & \textbf{Virtual - Virtual} \\  
					\midrule 
					\textbf{Ranking}     & 1. Web Server Privileged Time & 1. Web Server IO Data Bytes/sec \\
					\\
					& 2. Database Server Privileged Time & 2. Database Server User Time \\
					\\
					& 3. Web Server Page Faults/sec & 3.Database Server IO Other Bytes/sec\\
					\\
					& 4. Web Server Virtual Bytes & 4.Database Server Page Faults/sec\\
					\\
					& 5. Database Server Pool Nonpaged Bytes & \\
					\\
					& 6. Database Server Page Faults/sec & \\
					\\
					& 7. Database Server Page File Bytes &\\
					\\
					& 8. Database Server Working Set &\\
					\\
					\midrule
					\textbf{Fit} &  85.20\% &  89.10\% \\
					\midrule
					\textbf{MAPE} & 15.75\% & 4.70\% \\
					\bottomrule             
				\end{tabular}%
			}
	\end{table}
	%\end{landscape}
	
	
	\begin{table}[tbh]
		\centering
			\caption{CloudStore: Prediction errors cross-environments}
			\label{resultRQ3}
			\begin{tabular}{c|c}
				\toprule
				\textbf{Training - Testing}  & \textbf{MAPE}\\  
				\midrule 
				Physical-Virtual       & 293.62\% \\
				Virtual - Physical      & 28.95\% \\
				Virtual - Physical \textbf{(after scaling)} & 276.04\% \\
				\bottomrule             
			\end{tabular}
	\end{table}
	
	
	
	%
	%\begin{table*}[thb!]
	%    \begin{center}
	%    \caption{Results}
	%    \label{tab:project_results}
	%            \begin{tabular}{l| c c r c || c c c || c c c}
	%            \toprule
	%            \textbf{}   & \thead{Release}  & \thead{\# of classes}   & \thead{SLOC} & \thead{\# of \\contributors}  & \thead{\# of \\comments}   & \thead{\# of \\comments \\after filtering} & \thead{\# of \\TD \\comments} & \thead{\% of \\Design \\Debt} & \textbf{\% of Requirement Debt} & \thead{\% of \\Other \\Debts}\\ 
	%            \midrule 
	%            \textbf{Ranking}            & 1.7.0    & 1,475 & 115,881 & 74  & 21,587 &   4,137 &    131 &  72.51  & 09.92  & 17.55 \\
	%            ArgoUML        & 0.34     & 2,609 & 176,839 & 87  & 67,716 &   9,548 &  1,413 &  56.68  & 29.08  & 14.22 \\
	%            Columba        & 1.4      & 1,711 & 100,200 & 9   & 33,895 &   6,478 &    204 &  61.76  & 21.07  & 17.15 \\
	%            EMF            & 2.4.1    & 1,458 & 228,191 & 30  & 25,229 &   4,401 &    104 &  75.00  & 15.38  & 09.61 \\
	%            Hibernate      & 3.3.2 GA & 1,356 & 173,467 & 226 & 11,630 &   2,968 &    472 &  75.21  & 13.55  & 11.22 \\
	%            JEdit          & 4.2      &   800 &  88,583 & 57  & 16,991 &  10,322 &    256 &  76.56  & 05.46  & 17.96 \\
	%            JFreeChart     & 1.0.19   & 1,065 & 132,296 & 19  & 23,474 &   4,423 &    209 &  88.03  & 07.17  & 04.78 \\
	%            Jmeter         & 2.10     & 1,181 &  81,307 & 33  & 20,084 &   8,162 &    374 &  84.49  & 05.61  & 09.89 \\
	%            JRuby          & 1.4.0    & 1,486 & 150,060 & 328 & 11,149 &   4,897 &    622 &  55.14  & 17.68  & 27.17 \\ 
	%            SQuirrel       & 3.0.3    & 3,108 & 215,234 & 46  & 27,474 &   7,230 &    286 &  73.07  & 17.48  & 09.44 \\ 
	%            \bottomrule             
	%        \end{tabular}
	%    \end{center}
	%\end{table*}
	
	


%We retested our virtual and physical environments with a same set of load. As this experiment was a replica of our previous research question, our approach was based on the same lines. After obtaining the performance metrics, we preprocessed our data where every record represented the load and performance metrics generated for a minute. We removed the first thirty entries, which represented first thirty minutes in our tests to remove any noise.  After the preprocessing phase we had three set of metrics for every environment. We labeled one of them as the \textit{baseline metrics} and the rest as \textit{reference metrics 1} and \textit{reference metrics 2}. Succeedingly, these six datasets were imported in R for further analysis.
%To draw conclusion and validate the findings, firstly we removed any metric that showed no variance. This is pursued by removing metrics which are highly correlated with other metrics in the same data set, for example \textit{Processor Time} and \textit{Privileged Time}. We, then, train our data bolstered by the generalized linear model. \textit{Stepwise Regression} is used to filter the metrics which do not contribute to our model. The metrics are added one by one to until no further contribution is made to the model.  \cite{RInAction}. We trained our model with the set of \textit{baseline metrics} and tested our model by predicting the load for the \textit{reference metrics 1} and \textit{reference metrics 2} . We trained iteratively on the rest of the sets of metrics and tested for the remaining two sets of metrics. We repeated this approach for both of our environments.
%We calculated the accuracy of our approach by the \textit{mean absolute percentage error} where the absolute difference of real and predicted values is divided by the real values followed by taking the mean. \cite{perfmon}

%\textbf{Results}:  We observed that \textit{MAPE} values for majority of our experiments run on physical environment fell under \textbf{30\%} while for the virtual the values were more than 200\%. 
%We also concluded that because of the presence of the noise and the irregular behavior of the virtual environment, the experiments did not produce exact similar set of results. The difference in behavior was based on inconsistent \textit{User Time} and \textit{DISC IO} operations. This lead our to our next question that whether the performance testing done in virtual environment can be calibrated and applied to the physical environment, as profoundly practiced in the field of performance testing.

	
	

	
	 \section{Discussion}
     \label{sec:discussion}
     
In the previous section, we find that there is a discrepancy between performance testing results from the virtual and physical environments. However, such discrepancy can also be due to other factors such 1) the instability of the virtual environments, 2) the virtual machine that we used or 3) the different hardware resources on the virtual environments. Therefore, in this section, we examine the impact of such factors to better understand our results. 


\subsection{Instability of virtual environment}

\subsubsection{Motivation}

A major challenge in our case studies was to make our virtual environment stable. If the performance of virtual machines are unstable, the observed discrepancy in Section~\ref{sec:results} may be due to the instability of virtual environment. We address the hypothesis that if our virtual environment was unstable, then the performance testing results should not be repeatable and congruent between different runs in the same virtual environment. 

\subsubsection{Approach}

In order to study whether the virtual environment is stable, we repeat the same performance tests, twice, on the virtual environments for both subject systems. In total, we had results from three performance tests. We perform the data analysis in Section~\ref{sec:model} by building statistical models using performance metrics. %Table~\ref{tab:stabilityvm} shows the median absolute percentage error from building a model using one virtual environment and testing on another virtual environment. 
As the previously mentioned approach, we build a model based on one of the runs, serving as our training data for the model, and tested it on another run. In this case, we define external validation when a model is trained on a different run than it is tested on. We validate our model by predicting the throughput of a different run.  
 
 \subsubsection{Results}
Values(prediction error) closer to 0 indicate that our model was able to successfully explain the variation of the throughput of a different run. This also means that the external validation error closer to 1 or higher depicts an instability of the environment. We find that external validation error (0.04 and 0.13 for CloudStore and DS2) is almost as low as the internal validation error (0.03 and 0.09 for CloudStore and DS2). Such low error shows that the performance testing results from the virtual environments are rather stable. 

\subsection{Virtual machine software for the virtual environment}

\subsubsection{Motivation}

We also investigated the impact of choosing different virtual machine software on our experimental results. It can be argued that our chosen subject systems behave differently in another environment. In order to address the aforementioned hypothesis, we set up another virtual environment using VMWare (version 12) with the same allocated computing resources as when we set up Virtual Box.

\subsubsection{Approach}


We repeat the performance tests for both subject systems. We train statistical models on the performance testing results from VMWare and test on the results from both the original virtual environment data (Virtual Box) and the results from the physical environments. We could not apply the normalization by deviance for the data from VMWare since some of the significant metrics in the model have a median absolute deviance of 0, making the normalized metric value to be infinite (see Equation~\ref{equ:mad}). We only apply the normalization by load. 

 \subsubsection{Results}
The low percentage error when our model was tested on Virtual Box in Table~\ref{tab:vmware} shows that the performance testing results from the two different virtual machine software is similar. In addition, the high error when predicting with physical environment agrees with the results when testing with the performance testing results from the Virtual Box (see Table~\ref{tab:errors}). Such results show that the discrepancy observed during our experiment also exits with the virtual environments that are set up with VMWare.

\begin{table}[tbh]
	\centering
	\caption{Median absolute percentage error from building a model using VMWare data.}
	\label{tab:vmware}
		\begin{tabular}{|c||c|c|}
			\hline
			\multirow{2}{*}{\textbf{Validation type}} & \multicolumn{2}{c|}{\textbf{Median absolute percentage error}} \\ \cline{2-3} 
			& \textbf{CloudStore} & \textbf{DS2} \\ %\hline
			\midrule
			\midrule
			External validation with Virtual Box results& 0.07&0.10\\ \hline
%			External validation with physical normalization by deviance & 0.07 &0.06 \\ \hline
			External validation with physical normalization by load & 7.52& 1.63 \\ \hline
		\end{tabular}
\end{table}

\subsection{Resource Allocation}


\subsubsection{Motivation}

The third hypothesis about our study may be the result of our analysis is based on the resources allocated and the configuration of the virtual environment. As we already were working on an optimal point for our vritual environment to avoid system crashes, we only could increase the allocated resources in order to ensure the execution of the subject systems.


\subsubsection{Approach}
We increase the computing resources allocated to the virtual environments by increasing the CPU to be 3 cores and increasing the memory to be 5GB. We cannot allocate more resource to the virtual environment since we need to keep resources for the hosting OS. We train statistical models on the new performance testing results and tested it on the performance testing results from the physical environment. 


\subsubsection{Results}
Similar to the results shown in Table~\ref{tab:errors}, the prediction error is high when we normalize by load as per Equation~\ref{equ:mad} (1.57 for DS2 and 1.25 for CloudStore), while normalizing based on deviance can significantly reduce the error (0.09 for DS2 and 0.07 for CloudStore). By altering the resource allocated, such results show the minimal impact to our findings. Moreover, our results demonstrate the ability of reducing the discrepancy in performance testing results by using normalization based on deviance. 

     
     	
  	\section{Threats to validity}
   	\label{sec:threats_to_validity}
   	This sections chalks out out the threats to our validity.
\subsection{External validity}
\textcolor{red}{discuss}
We chose two subject system for our study and VirtualBox as our virutal environment. All of the former mentioned entities have a credible history when it comes to the research are of performance engineering. \cite{5306331} \cite{Nguyen:2012:ADP:2188286.2188344}. Also, we made sure that our virtual environment is set up exactly the same as our physical environment by keeping a constant checks aided by scripts. Having said that, this study can be boosted by additional subject systems being tested on more than one type of virtual environment. 

\subsection{Internal Validity}

All of our models are dependent on the performance metrics' accuracy. Which means if the load on the server is beyond the capacity of the system to handle and builds up a queue, there is a possibility of noise sneaking in the recording process of the performance metrics. 
The scaling approach we have adopted assumes that the for a particular metric, there exists alpha and beta values. However, if the metric value is constant it not possible to have an alpha and beta value associated with it in the model. Hence looking for an alternate scaling process is necessary for such a performance metric.
We build performance regression models to compare the metrics from both of our environments. This models are accurate if there exists a high number of records for the performance counters. Additionally, we also assume that none of our dependent variable is correlated to the independent variable or vice-versa. 

	
	 \section{Conclusion and Future work}
	 \label{sec:conclusion}
	 Performance assurance activities are vital in ensuring software reliability. Virtual environments are often used to conduct performance tests. However, the discrepancy between performance testing results in virtual and physical environments are never evaluated. In this paper, we evaluate such discrepancy by conducting performance tests on two open source systems (DS2 and CloudStore) in both virtual and physical environments. By examine the performance testing results, we find that there exist discrepancy between performance testing results in virtual and physical environments when examining individual performance metrics, relationship among performance metrics and building statistical models from performance metrics, even after we normalize performance metrics across different environments. The major contribution of this paper includes: 
%\vspace{-0.15cm}
\begin{itemize} \itemsep -0.8pt 
	\item Our paper is the first research attempt to evaluate the discrepancy between performance testing results in virtual and physical environments.
	\item We find that relationships among I/O related metrics have large differences between virtual and physical environments.
	\item We find that normalizing performance metrics based on deviance may reduce the discrepancy. Practitioners may exploit such normalization techniques when analyzing performance testing results from virtual environments.
\end{itemize}
%\vspace{-0.15cm}
Our results highlight the needs of awareness of discrepancy between performance testing results in virtual and physical environments, for both practitioners and researchers. Future research effort may focus on minimizing such discrepancy in order to improve the use of virtual environments in performance engineering and reliability assurance activities



%Our paper magnifies the impact of performance in difference environments. Additionally, we observed that the performance metrics from different environment do not belong to the same distribution. We tried to mitigate this impact by using scaled metrics in our performance regression models. As a result, the percentage error dropped drastically for one of our subject systems are increased for the other. We concluded that scaling may not apply to every subject system. We plan to see in what ways we can leverage the metrics from the virtual environment and use them for prediction of the metrics in the physical environment. We also plan to investigate the injection of regression in the system being hosted in a virtual environment will behave similar to that under regression in a physical environment. 
	
	\bibliography{mybibfile}
\end{document}