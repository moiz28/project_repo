\documentclass[8pt]{letter} % default is 10 pt
%\usepackage{newcent}   % uses new century schoolbook postscript font 
% the following commands control the margins:
\topmargin=-1in    % Make letterhead start about 1 inch from top of page
\textheight=8.8in  % text height can be bigger for a longer letter
\oddsidemargin=0pt % leftmargin is 1 inch
\textwidth=6.5in   % textwidth of 6.5in leaves 1 inch for right margin

\usepackage{booktabs}
\usepackage{letterbib}

\usepackage{natbib}
%\usepackage{natbib}



%\usepackage[pdftex]{graphicx}
%\usepackage{textcomp}
%\usepackage{booktabs, tabularx}
%\usepackage{rotating}
%\usepackage[dvipsnames, gray]{xcolor}
%\usepackage[dvipsnames]{xcolor}
%\usepackage[round,sort]{natbib}
% \usepackage{multirow}


\begin{document}

\let\raggedleft\raggedright                % needed to get date flush left

\newcommand{\response}[1]{{\bf Response.} #1}

\newcounter{commentCounter}
\newcommand{\comment}[2]{
  \stepcounter{commentCounter}
  \vspace{2em}
  {\bf Comment R#1.\arabic{commentCounter}.} {\em #2}
  }
 
\begin{letter}{}

\address{
Weiyi Shang\\
EV 3.129, 1515 Ste. Catherine Street West\\
Department of Computer Science and Software Engineering, Concordia University\\ 
Montreal, QC, Canada H3G 1M8 \\
1 514-848-2424 ext. 7801\\
} 



\opening{Dear Editor and Reviewers:} 
 
\noindent Thank you for your insightful feedback and comments, both positive and constructive, and for allowing us the opportunity to improve our manuscript. We have taken each of your comments into consideration and made the appropriate changes and extensions to our manuscript. 

In the end, as suggested by the reviewers, we removed the study results of topics in user reviews. We also made several structural changes throughout the paper, improved paper presentation, and improved the discussion of implications. We feel that the paper is now much stronger.


%significantly expanded our threats to validity section with more discussion of our approach, and experiments to support our heuristics for identifying test files. 

%Below, we repeat each of the reviewer's comments and provide our responses. 
Below, we include a description of the changes that we made to our manuscript with respect to each of the reviewers comments. We denote the reviewers comments in italic typeface, and our responses follow below each reviewer comment.


%We thank the reviewers for their valuable feedback. This document includes a list of changes and rebuttal to the comments that were raised by the reviewers. Below, we include a description of the changes made to our manuscript with respect to each of the reviewersÕ comments. We denote the reviewers comments in italic typeface. Our responses follow below each reviewer comment.


%%%%%%%%%%%%%%%%%%%%%%%%%%%%%%%%%%%%%%%%%%%%%%%
%%%%%%%%%%%%%%%%%%%%%%%%%%%%%%%%%%%%%%%%%%%%%%%
\vspace{2em}

\noindent {\Large {\bf Reviewer 4 Comments}}

\comment{4}{
While the results are interesting, the authors should describe more in depth the implications of their findings for practitioners and researchers analyzing user reviews in mobile applications platforms, i.e why is the consideration of length difference/submission frequency important? what should practitioners and researchers re-consider when automatically processing user reviews in view of the findings presented in this work?
}

\response{Thanks for the suggestion. We now elaborate and highlight the implications of our findings in the result section. }

\comment{4}{
Additionally, I would recommend removing the content discussion from the appendix as there is no in depth finding. It is unclear why the authors decided to run LDA on the reviews from all of the applications against the option of running on the reviews of each application separately. I believe that doing so would help to better answer their research question, as the individual topics of each application would be more apparent and avoid the dominance — in the topic generation — of the applications with most reviews.
}

\response{Based on the feedback by both reviewers and due to limited space, we removed the appendix.}

\vspace{2em}

\noindent {\Large {\bf Reviewer 3 Comments}}

\comment{3}{
According to the authors, the key point is that different app stores need to be considered differently.
In my opinion, this is not yet well reflected in the structure and argumentation line of the manuscript.

Rather, the manuscript has become an even more detailed study about reviews in the Google Play Store. The discussion of the topics, which has been added in this revision, are an example for this direction.
Personally, I think that this is not necessarily a bad thing. But it does not really seem to fit the format of a typical CACM article. E.g., the word count limit does not allow the authors to go into more detail and make more comparisons between the different other studies about app reviews. And yet there is no discussion about more practical implications of the study rather than the point that different stores have different characteristics.

I suggest that the authors take a step back and think again about their goal for this article. Do they want to publish a meta-study or a study of the Google Play Store vs. Apple App Store? Do they maybe need to publish a more in-depth article on the Google Play Store first, without any word count limit, so that they can place an overview over the differences in CACM? Did they maybe try to satisfy all reviewers' comments and lost a bit of the original direction?
}

\response{Thanks based on the feedback of both reviewers, we removed the appendix. We agree that it is not inline with the paper's main message.}

\comment{3}{
I think the work the authors have done has good potential. If I was the author, I would make two publications out of it. One about the study itself, allowing myself to express the depth of my results, and one meta study in CACM.

If the authors decide to continue focusing on the difference between their results for the Google Play Store to the Apple App Store, probably the argumentation line, structure, and focus of the current revision have to be adjusted.
}

\response{Thanks for the suggestion, we rephrase the argumentation line, re-structure the paper in order to fit better with the theme of focusing on the differences between the Google Play Store and the Apple App Store.}



\vspace{5mm}

\noindent Again, we thank all of you for your valuable feedback, which has made this a stronger manuscript. We look forward to hearing your feedback on the updated manuscript.

 \vspace{2mm}
 
\noindent{Sincerely,}\\
\noindent{Stuart McIlroy, Weiyi Shang, Nasir Ali, Ahmed E. Hassan}\\
%\noindent{\tt{\{tsehsun, sthomas, hemmati, mei, ahmed\}@cs.queensu.ca}}
\end{letter}
 
%\bibliographystyle{natbib}      % basic style, author-year citations
%\bibliographystyle{spmpsci}      % mathematics and physical sciences
%\bibliographystyle{spphys}       % APS-like style for physics
%\bibliographystyle{plain}  
%\bibliography{responses1} 


\end{document}




