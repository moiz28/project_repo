% -*- root: cuthesis_masters.tex -*-



\begin{landscape}

\begin{figure}[thb]
	%\centering
	%\vfil
	\vspace*{15ex}\includegraphics[width=1.5\textwidth]{figures/overview_ls}
	\caption{Overview of our case study setup.}
	%\captionsetup{justification=centering}
	\label{fig:Approach}
\end{figure}

\end{landscape}

\section{Approach}
\label{chap3:sec:approach}


The goal of our case study is to evaluate the discrepancy between performance testing results from virtual and physical environments. We deploy our subject systems in two identical environments (physical and virtual) with the same hardware. A load driver is used to exercise our subject systems. After the collection and processing of the performance metrics we analyze and draw conclusions based on: 1) single performance metric 2) relationship between performance metrics and 3) statistical models based on the performance metrics. An overview of our case study setup is shown in Figure~\ref{fig:Approach}.


\subsection{Subject Systems}
Dell DVD Store (DS2)~\cite{delldvd} is an online multi-tier e-commerce web application that is widely used in performance testing and prior performance engineering research~\cite{Shang:2015:ADP:2668930.2688052,Nguyen:2012:ADP:2188286.2188344, jackicsm2009}. We deploy DS2 (SLOC $>$ 3,200) on an Apache (Version 3.0.0) web application server with MySQL 5.6 database server~\cite{mysql}. CloudStore~\cite{cloudstore}, our second subject system, is an open source application based on the TPC-W benchmark~\cite{tpcw}. CloudStore (SLOC $>$ 7,600) is widely used to evaluate the performance of cloud computing infrastructure when hosting web-based software systems and is leveraged in prior research~\cite{tarekmsr16}. We deploy CloudStore on \textit{Apache Tomcat}~\cite{tomcat} (version 7.0.65) with MySQL 5.6 database server~\cite{mysql}.


\subsection{Environmental Setup}

The performance tests of the two subject systems are conducted on three machines in a lab environment. Each machine has an Intel i5 4690 Haswell Quad-Core 3.50 GHz CPU, with 8 GB of memory, 100GB SATA storage and connected to a local gigabyte ethernet. The first machine hosts the application servers (Apache and Tomcat). The second machine hosts the MySQL 5.6 database server. The load drivers were deployed on the third machine. We separate the load driver, the web/application server and the database server on different machines in order to mimic the real world scenario and avoid interference among these processes. For example, isolating the application and database driver would ensure that the processor is not overused. The operating systems on the three machines are Windows 7. We disable all other processes and unrelated system services to minimize their performance impact. Since our goal is to compare performance metrics in virtual and physical environments, we setup the two different environments, as follows:
\\

\noindent \textbf{Virtual environment.} We install one Virtual Box (version 5.0.16) and create only one virtual machine on one physical machine to avoid any interference between virtual machines. For each virtual machine, we allocate two cores and three gigabytes of memory, which is well below capacity to make sure we are not topping out and pushing our configuration for unrealistic results. Virtual machines typically have an option of using disk pass-through\cite{whatisdiskpassthrough}. However, disk pass-through prevents the quick deployment of an existing virtual machine image that's designed for performance testing and quick execution of performance tests~\cite{diskpassthrough}. Hence, we opt to disable disk pass-through since it is unlikely to be used in practice. The network of the virtual machine is set up based on network address translation (NAT) configuration\cite{NAT_config}. The network traffic of the workload was generated on a dedicated load machine to keep our experiments as close to the real-world as possible.
\\


\noindent \textbf{Physical environment.} We used the same hardware as the virtual environment to set up our physical environments. To make the physical environment similar to the virtual environment, we only enable two cores and three gigabytes of memory for each machine for the physical environment. 

\subsection{Performance tests}

DS2 is released with a dedicated load driver program that is designed to exercise DS2 for performance testing. We used the load driver to conduct performance testing on DS2. We used Apache JMeter~\cite{apachejmeter} to generate a workload to conduct the performance tests on CloudStore. For both subject systems, the workload of the performance tests is varied randomly and periodically in order to avoid bias from a consistent workload. The variation was identical across environments. The workload variation was introduced by the number of threads. A higher number of threads represents a higher number of users accessing the system. Each performance test is run after a 15 minute warming up period of the system and lasts for 9 hours. We chose to run the test 9 hours ensuring that our sample sizes have enough data points for our results to be statistically significant.
The nature of our performance tests was based on our related studies mentioned in section 2.2. To ensure the consistency between the performance tests, we restored the environments followed by a restart of the systems after every test.


\subsection{Data collection and preprocessing}

\noindent \textbf{Performance metrics.} We used \textit{PerfMon}~\cite{perfmon} to record the values of performance metrics. \textit{PerfMon} is a performance monitoring tool used to observe and record performance metrics such as CPU utilization, memory usage and disk IOs. We run \textit{PerfMon} on each of the application server and database server machines. We record all the available performance metrics that can be monitored on a single process by \emph{PerfMon}. In order to minimize the influence of \textit{Perfmon}, we monitor only the performance of the two processes of the application server and database server on the two dedicated machines. We recorded the performance metrics with an interval of 10 seconds. In total, we recorded 44 performance metrics.
\\

\noindent \textbf{System throughput.} We used the application server's access logs from Apache and Tomcat to calculate the throughput of the system by measuring the number of requests per minute. The two datasets were then concatenated and mapped against requests using their respective timestamps.

Since an end user will consider a system as a whole, we combine the performance datasets from our application and database servers. In order to combine the two datasets of performance metrics and system throughput, and to minimize noise of the performance metric recording, we calculate the mean values of the performance metrics every minute. Then, we combine the datasets of performance metrics and system throughput based on the time stamp on a per minute basis. A similar approach has been applied to address mining performance metrics challenges~\cite{foo2010mining}.

The goal of our study is to evaluate the discrepancy between performance testing results from virtual and physical environments, particularly considering the impact of discrepancy on the analysis of such results. Our experiments are set in the context of analyzing performance testing data, based on the related work. Shown in Section~\ref{sec:related}, prior research and practitioners examine performance testing results in three types of approaches: 1) examining a single performance metric, 2) examining the relationship between performance metrics and 3) building statistical models using performance metrics. Therefore, our experiments are designed to answer three research questions, where each questions corresponds to one of the types of analysis above.



\subsection{Are the trend and distribution of a single performance metric similar across environments?}
\label{sec:individual}

\noindent \textbf{Motivation.}
The most intuitive approach of examining performance testing results is to examine every single performance metric. As shown in Section~\ref{sec:relatedindividual}, prior studies propose different approaches that typically compare the distribution or trend of each performance metric from different tests. Due to influences from testing environments, performance testing results are not expected to be identical in raw values. However, the shape of the distribution and the trend of the metrics should be similar. For example, if in one environment, we observe that Memory has an increasing trend while the increasing trend is not seen in another environment, we observe a discrepancy. In addition, the distribution differences between two test results should not be statistically significant. Therefore, we use quantile-quantile (Q-Q) plot and normalized Kolmogorov-Smirnov (KS) tests to examine the differences in trends and shape of the distributions. 
\\

\noindent \textbf{Approach.} 
After running and collecting the performance metrics, we compare every single performance metric between the virtual and physical environments. Since the performance tests are conducted in different environments, intuitively the scales of performance metrics are not the same. For example, the virtual environment may have higher CPU usage than the physical environment. Therefore, instead of comparing the values of each performance metric in both environments, we study whether the performance metric follows the same shape of the distribution and the same trend in virtual and physical environments. 

First, we plot a quantile-quantile (Q-Q) plot~\cite{qqplots} for every performance metric in two environments. A Q-Q plot is a plot of the quantiles of the first data set against the quantiles of the second data set. We also plot a 45-degree reference line on the Q-Q plots. If the performance metrics in both environments follow the same shape of the distribution, the points on the Q-Q plots should fall approximately along this reference (i.e., 45-degree) line. A large departure from the reference line indicates that the performance metrics in the virtual and physical environments come from populations with different shapes of distributions, which can lead to a different set of conclusions. For example, the virtual environment has a CPU's utilization spike at a certain time, but the spike is absent in the physical environment. 

Second, to quantitatively measure the discrepancy, we perform a Kolmogorov-Smirnov test~\cite{kstest} between every performance metric in the virtual and physical environments. Since the scales of each performance metric in both environments are not the same, we first normalize the metrics based on their median values and their median absolute deviation: 
\begin{equation}
\label{equ:mad}
M_{normalized}=\frac{M-\tilde{M}}{MAD(M))}		
\end{equation}

where $M_{normalized}$ is the normalized value of the metric, $M$ is the original value of the metric, $\tilde{M}$ is the median value of the metric and $MAD(M)$ is the median absolute deviation of the metric~\cite{walker1929studies}. The Kolmogorov-Smirnov test gives a p-value as the test outcome. A p-value $\leq$ 0.05 means that the result is statistically significant, and we may reject the null hypothesis (i.e., two populations are from the same distribution). By rejecting the null hypothesis, we can accept the alternative hypothesis, which tells us the performance metrics in virtual and physical environments do not have the same distribution. We choose to use the Kolmogorov-Smirnov test since it does not have any assumption on the distribution of the metrics.

Finally, we calculate Spearman's rank correlation between every performance metric in the virtual environment and the corresponding performance metric in the physical environment, in order to assess whether the same performance metrics in two environments follow the same trend during the test. Intuitively, two sets of performance testing results without discrepancy should show a similar trend, i.e., when memory keeps increasing in the physical environment (like memory leak), the memory should also increase in the virtual environment. We choose Spearman's rank correlation since it does not have any assumption on the distribution of the metrics. 
\\

\noindent \textbf{Results.}
\noindent \textbf{Most performance metrics do not follow the same shape of the distribution in virtual and physical environments.} Figure~\ref{fig:qqds2} and \ref{fig:qqcs} show the Q-Q plots by comparing the quantiles of performance metrics from virtual and physical environments. We only present Q-Q plots for CPU user time, IO data operations/sec and memory working set for both application sever and database server. For complete results please refer Chapter \ref{appendix}.



\begin{figure}[tp]]
	\centering
	\includegraphics[width=0.9\columnwidth]{figures/DS2_qq.pdf}
	\caption{Q-Q plots for DS2.}
	\label{fig:qqds2}
\end{figure}



\begin{figure}[tp]
	\centering
	\includegraphics[width=0.9\columnwidth]{figures/CS_qq.pdf}
	\caption{Q-Q plots for CloudStore.}
	\label{fig:qqcs}
\end{figure}



%\footnote{The complete results, data and scripts are shared online at http://das.encs.concordia.ca/members/moiz-arif/}. 
The results show that the lines on the Q-Q plot are not close to the 45-degree reference line. By looking closely on the Q-Q plots we find that the patterns of each performance metric from different subject systems are different. For example, the application (web) server's CPU user time for DS2 in the virtual environment shows higher values than in the physical environment at the median to high range of the distribution; while the Q-Q plot of CloudStore shows the application (web) server's CPU user time with higher values at the low range of the distribution. In addition, the lines of the Q-Q plots for database memory working set show completely different shapes in DS2 and in CloudStore. The results imply that the discrepancies between virtual and physical environments are present between the subject systems. The impact of the subject systems warrants its own study.

The majority of the performance metrics had statistically significantly different distributions (p-values lower than 0.05 in Kolmogorov-Smirnov tests). Only 13 and 12 metrics (out of 44 for each environment) have p-values higher than 0.05, for DS2 and CloudStore, respectively, showing statistically in-significant difference between the distribution in virtual and physical environments. By looking closely at such metrics, we find that these metrics either do not highly relate to the execution of the subject system (e.g., application server CPU privileged time in DS2), or highly relate to the workload. Since the workload between the two environments is similar, it is expected that the metrics related to the workload follow the same shape of the distribution. For example, the I/O operations are highly related with the workload. The metrics related to I/O operations may show statistically in-significant differences between the distributions in the virtual and physical environments (e.g., application server I/O write operations per second in DS2). %\emad{should we list the metrics}


\noindent \textbf{Most performance metrics do not have the same trend in virtual and physical environments.} Table~\ref{tab:correlationrq1} shows the Spearman's rank correlation coefficient and corresponding p-value between the selected performance metrics for which we shared the Q-Q plots. We find that for the application server memory working set in CloudStore and the database server memory working set in DS2, there exists strong (0.69) to moderate (0.46) correlation between the virtual and physical environments, respectively. By examining the metrics, we find that both metrics have an increasing trend that may be caused by a memory leak. Such increasing trend may be the cause of the moderate to strong correlation. Instead of showing the selected metrics as the Q-Q plots, Table~\ref{tab:correlationall} shows a summary of the Spearman's rank correlation of all the performance metrics. Most of the correlations have an absolute value of 0 to 0.3 (low correlation), or the correlation is not statistically significant (p-val\textgreater0.05).

\noindent \textbf{Impact on the interpretation of examining single performance metric.} Practitioners often plot the trend of each important performance metrics, identify when the outliers exist or calculate the median or mean value of the metric to understand the performance of the system in general. However, based on our findings in this RQ, such analysis results may not be useful if they are from a virtual environment. For example, shown in Figures~\ref{fig:qqds2} and~\ref{fig:qqcs} many differences between the two distribution are in the lower and higher ends of the plots, which correspond to the high and low values of the metrics. Such values are often treated as outliers. However, if such outliers are due to the virtual environment rather than the system itself, the results may be misleading. In addition, since the distribution of the metrics are statistically different, the mean and median value of the metrics may also be misleading. 

\noindent\fbox{%
	\parbox{\textwidth}{%
		\textbf{Findings: }Performance metrics typically do not follow the same distribution in virtual and physical environments.\\ 
		\textbf{Actionable implications: }Practitioners cannot assume a straightforward overhead from the virtual environment nor compare single performance metric after applying a simple scaling factor to the metric.  
	}%
}




\begin{table}[thb]
	\centering
	\caption{Spearman's rank correlation coefficients and p-values of the highlighted performance metrics.}
	\label{tab:correlationrq1}
	\begin{tabular}{|c||c|c|c|c|}
		\hline
		\multirow{2}{*}{\textbf{Performance Metrics}} & \multicolumn{2}{c|}{\textbf{DS2}} & \multicolumn{2}{c|}{\textbf{CloudStore}} \\ \cline{2-5} 
		& \textbf{coef.} & \textbf{p-value} & \textbf{coef.} & \textbf{p-value} \\ %\hline
		\midrule 
		\midrule 
		Web Servers' User Times & 0.08 & 0.07 & -0.04 & 0.33 \\ \hline
		DB Servers User Times & -0.05 & 0.30 & 0.10 & 0.02 \\ \hline
		Web Servers' IO Data Ops/sec & 0.25 & 0.00 & 0.13 & 0.00 \\ \hline
		DB Servers' IO Data Ops/sec & -0.14 & 0.00 & 0.13 & 0.00 \\ \hline
		Web Servers' Memory Working Set & 0.22 & 0.00 & 0.69 & 0.00 \\ \hline
		DB Servers' Memory Working Set & 0.46 & 0.00 & -0.16 & 0.00 \\ \hline
	\end{tabular}
\end{table}

\begin{table}[tbh]
	\centering
	\caption{Summary of Spearman's rank correlation p-values and absolute coefficients of all the performance metrics in DS2 and CloudStore. The numbers in the table are the number of metrics that fall into each category.}
	\label{tab:correlationall}
	\begin{threeparttable}
		
		\begin{tabular}{|c||c|c|c|c|c|}
			\hline
			\multirow{3}{*}{\textbf{System}} & \multirow{3}{*}{\textbf{p-value\textgreater0.05}} & \multicolumn{4}{c|}{\textbf{p-value\textless0.05}} \\ \cline{3-6} 
			&  & \textbf{0.0$\sim$0.3} & \textbf{0.3$\sim$0.5} & \textbf{0.5$\sim$0.7} & \textbf{0.7$\sim$1} \\ %\hline
			\midrule 
			\midrule 
			\textbf{DS2} & 8 & 28 & 4 & 0 & 1 \\ \hline
			\textbf{CloudStore} & 5 & 25 & 4 & 4 & 3 \\ \hline
		\end{tabular}%
		\begin{tablenotes}
			\item Three metrics are constant. Therefore, we do not calculate the correlation on those metrics.
		\end{tablenotes}
	\end{threeparttable}
	
\end{table}



\subsection{To what extent does the relationship between the performance metrics change across environments?}

\label{sec:relation}

\noindent \textbf{Motivation.}
The relationship between two performance metrics may significantly change between two environments, which may be a hint of performance issues or system regression. 
%We have used the following approach to examine the relationships between the performance metrics. 
As found by Cohen \emph{et al$.$}~\cite{cohen2004correlating}, combinations of performance metrics are significantly more predictive towards performance issues than a single metric. A change in these combinations can reflect the discrepancy of performance and can help a practitioner identify the behavioral changes of a system between the two environments. For instance, in one release of the system, the CPU may be highly correlated with I/O while (e.g., when I/O is high, CPU is also high); while on a new release of the system, the correlation between CPU and I/O may become low. Such change to the correlation may expose a performance issue (e.g., the high CPU without I/O operation may be due to a performance bug). However, if there is a significant difference in correlations simply due to the platform being used, i.e., virtual vs. physical, then practitioners may need to be warned that a correlation discrepancy may be false. Therefore, we examine whether the relationship among performance metrics has a discrepancy between the virtual and physical environments. 
\\

\noindent \textbf{Approach.} 
We calculate Spearman's rank correlation coefficients among all the metrics from each performance test in each environment. Then we study whether such correlation coefficients are different between the virtual and physical environments. 

First, we compare the changes in correlation between the performance metrics and the system throughput. For example, in one environment, the system throughput may be highly correlated with CPU; while in another environment, such correlation is low. In such a case, we consider there to be a discrepancy in the correlation coefficient between CPU and the system throughput. Second, for every pair of metrics, we calculate the absolute difference between the correlation in two environments. For example, if CPU and Memory have a correlation of $0.3$ in the virtual environment and $0.5$ in the physical environment, we report the absolute difference in correlation as $0.2$ ($|0.3-0.5|$). Since we have 44 metrics in total, we plot a heatmap in order to visualize the 1,936 absolute difference values between every pair of performance metrics. The lighter the color for each block in the heatmap, the larger the absolute difference in correlation between a pair of performance metrics. With the heatmap, we can quickly spot the metrics that have large discrepancy in correlation coefficients. 
\\

\noindent \textbf{Results.}
\noindent \textbf{The correlations between system throughput and performance metrics change between virtual and physical environments.} Tables~\ref{tab:top10ds2p} and~\ref{tab:top10csp} present the top ten metrics with the highest correlations to system throughput in the physical environment for DS2 and CloudStore, respectively. We chose system throughput to be our criterion as it was kept identical between the environments.  We find that for these top ten metric sets, the difference in correlation coefficients in virtual and physical environments is up to \textbf{0.78} and the rank changes from \#9 to \#40 in DS2 and \#1 to \#10 in CloudStore.

\noindent \textbf{There exist differences in correlation among the performance metrics from virtual and physical environments.} Figures~\ref{fig:heatmap} and~\ref{fig:heatmap_cs} present the heatmap showing the changes in correlation coefficient among the performance metrics from virtual and physical environments. By looking at the heatmap, we find hotspots (with lighter color), which have larger correlation differences. For the sake of brevity, we do not show all the metric names in our heatmaps. Instead, we enlarge the heatmap by showing one of the hotspots for each subject system in Figures~\ref{fig:heatmap} and~\ref{fig:heatmap_cs}. We find that the hotspots correspond to the changes in correlation among I/O related metrics. Prior research on virtual machines has similar findings about I/O overhead in virtual machines~\cite{menon2005diagnosing,kraft2011io}. In such a situation, when practitioners observe that the relationship between I/O metrics and other metrics change, the change may not indicate a performance regression, but rather the change may be due to the use of a virtual environment.

\begin{figure}[tp]
	\centering
	{\includegraphics[width=1.0\textwidth]{figures/heatmap_DS2}}
	\caption{Heatmap of correlation changes for DS2.}
	%\captionsetup{justification=centering}
	\label{fig:heatmap}
\end{figure}


\begin{figure}[tp]
	\centering
	{\includegraphics[width=1.0\textwidth]{figures/heatmap_CS}}
	\caption{Heatmap of correlation changes for CloudStore.}
	%\captionsetup{justification=centering}
	\label{fig:heatmap_cs}
\end{figure}




\noindent \textbf{Impact on the interpretation of examining correlations between performance metric.} When a system is reported to have performance issues, correlations between metrics are often used in practice, as describe in the motivation of this RQ. However, since such correlation can be inconsistent in virtual and physical environment, existing knowledge of assumptions of correlation may not exist or new correlation may emerge, due to the use of virtual environment. For example, practitioners of a database-centric system may have the knowledge that I/O traffic is correlated with CPU and system throughput. Examining these three metrics together can help diagnose performance issues, while if no such correlation exists in the virtual environment, these three metrics together may not be as useful in performance issue diagnosis.


\noindent\fbox{%
	\parbox{\textwidth}{%
		\textbf{Findings: }The correlations between performance metrics and system load may change considerably between virtual and physical environments. The correlation among performance metrics may also change considerably between virtual and physical environments. The correlations that are related with I/O metrics have the largest discrepancy.\\
		\textbf{Actionable implications: }Practitioners should always verify whether the inconsistency of correlations between performance metrics (especially I/O metrics) are due to virtual environments.
	}%
}






\begin{table}[tbh]
	\centering
	\caption{Top ten metrics with highest correlation coefficient to system throughput in the physical environment for DS2. }
	\label{tab:top10ds2p}
	\begin{threeparttable}
		
		\begin{tabular}{|c||c|c|c|c|}
			\hline
			\textbf{Rank} & \textbf{Performance } & \textbf{Coef. } & \textbf{Coef. } & \textbf{Rank in} \\ %\hline
			& \textbf{ Metrics} & \textbf{PE} & \textbf{VE} & \textbf{VE} \\ %\hline
			\midrule
			\midrule
			1 & Web IO Other Ops/sec & 0.91 & 0.62 & 10 \\ \hline
			2 & Web IO Other Bytes/sec & 0.91 & 0.62 & 12 \\ \hline
			3 & Web IO Write Ops/sec & 0.91 & 0.63 & 9 \\ \hline
			4 & Web IO Data Ops/sec & 0.91 & 0.63 & 8 \\ \hline
			5 & Web IO Write Bytes/sec & 0.90 & 0.62 & 11 \\ \hline
			6 & Web IO Data Bytes/sec & 0.90 & 0.61 & 13 \\ \hline
			7 & DB IO Other Ops/sec & 0.84 & 0.75 & 3 \\ \hline
			8 & DB IO Data Ops/sec & 0.83 & 0.07 & 41 \\ \hline
			9 & DB IO Other Bytes/sec & 0.83 & 0.15 & 40 \\ \hline
			10 & DB IO Read Ops/sec & 0.82 & 0.15 & 39 \\ \hline
		\end{tabular}%
		\begin{tablenotes}
			\item PE in the table is short for physical environment; while VE is short for virtual environment.
		\end{tablenotes}
	\end{threeparttable}
	
	
\end{table}

\begin{table}[tbh]
	\centering
	\caption{Top ten metrics with highest correlation coefficient to system throughput in the physical environment for CloudStore}
	\label{tab:top10csp}
	\begin{threeparttable}
		
		\begin{tabular}{|c||c|c|c|c|}
			\hline
			\textbf{Rank} & \textbf{Performance } & \textbf{Coef. } & \textbf{Coef. } & \textbf{Rank in} \\ %\hline
			& \textbf{ Metrics} & \textbf{PE} & \textbf{VE} & \textbf{VE} \\ %\hline
			\midrule
			\midrule
			1 & DB Server IO Other Bytes/sec & 0.98 & 0.73 & 10 \\ \hline
			2 & DB Server IO Read Ops/sec & 0.98 & 0.84 & 7 \\ \hline
			3 & DB Server IO Read Bytes/sec & 0.98 & 0.93 & 5 \\ \hline
			4 & DB Server IO Write Ops/sec & 0.98 & 0.97 & 2 \\ \hline
			5 & DB Server IO Data Ops/sec & 0.98 & 0.92 & 6 \\ \hline
			6 & DB Server IO Data Bytes/sec & 0.98 & 0.96 & 4 \\ \hline
			7 & DB Server IO Write Bytes/sec & 0.98 & 0.96 & 3 \\ \hline
			8 & Web Server IO Other Bytes/sec & 0.98 & 0.68 & 16 \\ \hline
			9 & DB Server IO Other Ops/sec & 0.98 & 0.98 & 1 \\ \hline
			10 & Web Server IO Other Ops/sec & 0.98 & 0.70 & 14 \\ \hline
		\end{tabular}%
		\begin{tablenotes}
			\item PE in the table is short for physical environment; while VE is short for virtual environment.
		\end{tablenotes}
	\end{threeparttable}
	
	
\end{table}


\subsection{Can statistical performance models be applied across virtual and physical environments?}
\label{sec:model}


\noindent \textbf{Motivation.}
As discussed in the last research question (see Section~\ref{sec:relation}), the relationship among performance metrics is critical for examining performance testing results (see Section~\ref{sec:relatedrelation}). However, thus far we have only examined the relationships between two performance metrics. In order to capture the relationship among a large number of performance metrics, more complex modeling techniques are needed. Hence, we use statistical modeling techniques to examine the relationship among a set of performance metrics~\cite{xiong2013vperfguard,cohen2004correlating}. Moreover, some performance metrics do not have any impact with system performance, which are still examined. For example, for a software system that is CPU intensive, I/O operations may be irrelevant. Such performance metrics may expose large discrepancies between virtual and physical environments while not contributing to the examination of performance testing results. It is necessary to remove such performance metrics that are not contributing or impacting the results of the performance analysis. To address the above issues, modeling techniques are proposed to examine performance testing results (see Section~\ref{sec:relatedmodel}). In this step, we examine whether the modeling among performance metrics can apply across virtual and physical environments and whether we can minimize such discrepancy between performance models.
\\

\noindent \textbf{Approach.}
We follow a model building approach that is similar to the approach from prior research~\cite{Shang:2015:ADP:2668930.2688052,Cohen:2005:CIC:1095810.1095821,xiong2013vperfguard}. We first build statistical models using performance metrics from one environment, then we test the accuracy of our performance model with the metric values from the same environment and also from a different environment. For example, if the model was built in a physical environment it was tested in both, physical and virtual environments.

\subsubsection{B-1: Reducing metrics}

Mathematically, performance metrics that show little or no variation do not contribute to the statistical models hence we first remove performance metrics that have constant values in the test results. We then perform a correlation analysis on the performance metrics to remove multicollinearity based on statistical analysis~\cite{cor_R}. We used the Spearman's rank correlation coefficient among all performance metrics from one environment. We find the pair of performance metrics that have a correlation higher than 0.75, as 0.75 is considered to be a high correlation~\cite{Syer2016}. From these two performance metrics, we remove the metric that has a higher average correlation with all other metrics. We repeat this step until there exists no correlation higher than 0.75.

We then perform redundancy analysis on the performance metrics. The redundancy analysis would consider a performance metric redundant if it can be predicted from a combination of other metrics~\cite{harrell2001regression}. We use each performance metric as a dependent variable and use the rest of the metrics as independent variables to build a regression model. We calculate the $R^2$ of each model. $R^2$, or the coefficient of multicollinearity, is used to analyze how a change in one of the variables (e.g. predictor) can be explained by the change in the second variable (e.g. response) \cite{rsquare}. We consider multicollinearity to be present if more than one predictor variable can explain the change in the response variable. If the $R^2$ is larger than a threshold (0.9)\cite{Syer2016}, the current dependent variable (i.e., performance metric) is considered redundant. We then remove the performance metric with the highest $R^2$ and repeat the process until no performance metric can be predicted with $R^2$ higher than the threshold. For example, if CPU can be linearly modeled by the rest of the performance metrics with $R^2\textgreater0.9$, we remove the metric for CPU.

Not all the metrics in the model are statistically significant. Therefore in this step, we only keep the metrics that have a statistically significant contribution to the model. We leverage the \textit{stepwise} function that adds the independent variables one by one to the model to exclude any metrics that are not contributing to the model~\cite{RInAction}. 

\subsubsection{B-2: Building statistical models}

In the second step, we build a linear regression model~\cite{freedman2009statistical} using the performance metrics that are left after the reduction and removal of statistically insignificant metrics in the previous step as independent variables and use the system throughput as our dependent variable. We chose the linear regression model over other models because of its simple explanation. Hence, it is easier to interpret the discrepancy that is illustrated by the model. Similar models have been built in prior research~\cite{Cohen:2005:CIC:1095810.1095821,xiong2013vperfguard,Shang:2015:ADP:2668930.2688052}.

%\subsubsection{B-3: Finalizing statistical models}
After removing all the insignificant metrics, we have all the metrics that significantly contribute to the model. We use these metrics as independent variables to build the final model.

\subsubsection{V-1: Validating model fit}

Before we validate the model with internal and external data, we first examine how good the model fit is. If the model has a poor fit to the data, then our findings from the model may be biased by the noise from the poor model quality. We calculate the $R^2$ of each model to measure fit. If the model perfectly fits the data, the $R^2$ of the model is 1, while a zero $R^2$ value indicates that the model does not explain the variability of the response data. We would also like to estimate the impact that each independent variable has on the model fit. We follow a ``drop one'' approach~\cite{Chambers1990}, which measures the impact of an independent variable on a model by measuring the difference in the performance of models built using: (1) all independent variables (the full model), and (2) all independent variables except for the one under test (the dropped model). A Wald statistic is reported by comparing the performance of these two models ~\cite{harrell2001regression}. A larger Wald statistic indicates that an independent variable has a larger impact on the model's performance, i.e., model fit. A similar approach has been leveraged by prior research in~\cite{mcintosh2015emse}. We then rank the independent variables by their impact on model fit. 


\subsubsection{V-2: Internal validation}

We validate our models with the performance testing data that is from the same environment. We leverage a standard 10-fold cross validation process, which starts by partitioning the performance data into 10 partitions. We take one partition (fold) at a time as the test set, and train on the remaining nine partitions~\cite{10foldcross,kohavi1995study}, similar to prior research~\cite{haroon}. For every data point in the testing data, we calculate the absolute percentage error. For example, for a data point with a throughput value of 100 requests per minute, if our predicted value is 110 requests per minute, the absolute percentage error is $0.1$ ($\frac{|110-100|}{100}$). After the ten-fold cross validation, we have a distribution of absolute percentage error (\textit{MAPE}) for all the data records.



\subsubsection{V-3: External validation}
To evaluate whether the model built using performance testing data in one environment (e.g., virtual environment) can apply to another environment (e.g., physical environment), we test the model using the data from the other environment.

Since the performance testing data is generated from different environments, directly applying the data on the model would intuitively generate large amounts of error. We adopt two approaches in order to normalize the data in different environments: (1) \textbf{Normalization by deviance.} The first approach we use is the same when we compare the distribution of each single performance metric shown in Equation~\ref{equ:mad} from Section~\ref{sec:individual} by calculating the relative deviance of a metric value from its median value. (2) \textbf{Normalization by load.} The second approach that we adopt is an approach that is proposed by Nguyen \textit{et al.}~\cite{Nguyen:2012:ADP:2188286.2188344}. The approach uses the load of the system to normalize the performance metric values across different environments. As there are varying inputs for the performance tests that we carried out, normalization by load helps in normalizing the multi-modal distribution that might be because of the trivial tasks like background processes(bookkeeping).



To normalize our metrics, we first build a linear regression model with the one metric as an independent variable and the throughput of the system as the dependent variable. With the linear regression model in one environment, the metric values can be represented by the system throughput. Then we normalize the metric value by the linear regression from the other environment. The details of the metric transformation are shown as follows:

\begin{equation*}
throughput_{p}= \alpha_{p} \times M_{p} + \beta_{p}
\end{equation*}
\vspace{-0.4cm}
\begin{equation*}
throughput_{v}= \alpha_{v} \times M_{v} + \beta_{v}
\end{equation*}
\vspace{-0.4cm}
\begin{equation*}
M_{normalized} = \frac{(\alpha_{v} \times M_{v})+\beta_{v}-\beta_{p}}{\alpha_{p}}
\end{equation*}
where $throughput_{p}$ and $throughput_{v}$ are the system throughput in the physical and virtual environment, respectively. $M_{p}$ and $M_{v}$ are the performance metrics from both environments, while $M_{normalized}$ is the metric after normalization. $\alpha$ and $\beta$ are the coefficient and intercept values for the linear regression models. After normalization, we calculate the absolute percentage error for every data record in the testing data.




\subsubsection{Identifying model discrepancy}
In order to identify the discrepancy between the models built using data from the virtual and physical environments, we compare the two distributions of absolute percentage error based on our internal and external validation. If the two distributions are significantly different (e.g., the absolute percentage error from internal validation is much lower than that from external validation), the two models are considered to have a discrepancy. To be more concrete, in total for each subject system, we ended up with four distributions of absolute percentage error: 1) modeling using the virtual environment and testing internally (on data from the virtual environment), 2) modeling using the virtual environment and testing externally (on data from the physical environment), 3) modeling using the physical environment and testing internally (on data from the physical environment), 4) modeling using the physical environment and testing externally (on data from the virtual environment). We compare distributions 1) and 2) and we compare distributions 3) and 4). Since normalization based on deviance will change the metrics values to be negative when the metric value is lower than median, such negative values cannot be used to calculate absolute percentage error. We perform a min-max normalization on the metric values before calculating the absolute percentage error. In addition, if the observed throughput value after normalization is zero (when the observed throughput value is the minimum value of both the observed and predicted throughput values), we cannot calculate the absolute percentage error for that particular data record. Therefore, we remove the data record if the throughput value after normalization is zero. In our case study, we only removed one data record when performing external validation with the model built in the physical environment. 
\\

\noindent \textbf{Results.}
\noindent \textbf{The statistically significant performance metrics leveraged by the models in virtual and physical environments are different.} Tables~\ref{tab:modelsummaryds2} and \ref{tab:modelsummarycs} show the summary of the statistical models built for the virtual and physical environments for the two subject systems. We find that all the models have a good fit (66.9\% to 94.6\% $R^2$ values). However, some statistically significant independent variables in one model do not appear in the other model. For example, Web Server Virtual Bytes ranks \#4 for the model built from the physical environment data of CloudStore, while the metric is not significant in the model built from the virtual environment data. In fact, none of the significant variables in the model built from the virtual environment are related to the application server's memory (see Table~\ref{tab:modelsummarycs}). We do observe some performance metrics that are significant in both models even with the same ranking. For example, Web Server IO Other Bytes/sec is the \#1 significant metric for both models built from the virtual and physical environment data of DS2 (see Table~\ref{tab:modelsummaryds2}). 

\noindent \textbf{The prediction error illustrates discrepancies between models built in virtual and physical environments.} Although the statistically significant independent variables in the models built by the performance testing results in the virtual and physical environments are different, the model may have similar prediction results due to correlations between metrics. However, we find that the external prediction errors are higher than internal prediction errors for all four models from the virtual and physical environments for the two subject systems. In particular, Table~\ref{tab:errors} shows the prediction errors using normalization based on load is always higher than that of the internal validation. For example, the median absolute percentage error for CloudStore using normalization by load is 632\% and 483\% for the models built in the physical environment and virtual environment, respectively; while the median absolute percentage error in internal validation is only 2\% and 10\% for the models built in the physical and virtual environments, respectively. However, in some cases, the normalization by deviance can produce low absolute percentage error in external validation. For example, the median absolute percentage error for CloudStore can be reduced to 9\% using normalization by deviance.\\ 
\indent One possible reason is that the normalization based on load performs better, even though it is shown to be effective in prior research~\cite{Nguyen:2012:ADP:2188286.2188344}, assumes a linear relationship between the performance metric and the system load. However, such an assumption may not be true in some performance testing results. For example, Table~\ref{tab:top10ds2p} shows that some I/O related metrics do have low correlation with the system load in virtual environments. On the other hand, the normalization based on deviance shows much lower prediction error. We think the reason is that the virtual environments may introduce metric values with high variance. Normalizing based on the deviance controls such variance, leading to lower prediction errors.



\begin{table}[tbh]
	\centering
	\caption{Summary of statistical models built for DS2. The metrics listed in the table are the significant independent variables.}
	\label{tab:modelsummaryds2}
	\resizebox{\columnwidth}{!}{%
		\begin{tabular}{|c||c|c|}
			\hline
			\textbf{Environment} & \textbf{Physical} & \textbf{Virtual} \\ %\hline
			\midrule
			\midrule
			\textbf{1} & Web Server IO Other Bytes/sec & Web Server IO Other Bytes/sec \\ \hline 
			\textbf{2} & Web Server Page Faults/sec & DB server Working Set - Peak \\ \hline
			\textbf{3} & DB Server Page Faults/sec & Web Server Virtual Bytes \\  \hline
			\textbf{4} & DB Server IO Write Bytes/sec & Web Server Page Faults/sec \\ \hline
			\textbf{5} & Web Server IO Read Bytes/sec & DB Server Page Faults/sec \\ \hline
			\textbf{6} & DB Server User Time & DB Server IO Data Ops/sec \\ \hline
			\textbf{7} & DB Server Pool Paged Bytes & -  \\ \hline
			\textbf{8} & DB Server Privileged Time &  - \\ \hline
			\midrule
			\textbf{$R^2$}  & 94.6\% & 66.90\% \\ \hline
			%			\textbf{MAPE} & 4.00\% & 10.52\% \\ \hline
		\end{tabular}%
	}
\end{table}


\begin{table}[tbh]
	\centering
	\caption{Summary of statistical models built for CloudStore. The metrics listed in the table are the significant independent variables.}
	\label{tab:modelsummarycs}
	\resizebox{\columnwidth}{!}{%
		\begin{tabular}{|c||c|c|}
			\hline
			\textbf{Environment} & \textbf{Physical} & \textbf{Virtual} \\ %\hline
			\midrule
			\midrule
			\textbf{1} & Web Server Privileged Time & Web Server IO Write Ops/sec \\ \hline
			\textbf{2} & DB Server Privileged Time & DB Server IO Read Ops/sec \\ \hline
			\textbf{3} & Web Server Page Faults/sec &  Web Server Privileged Time \\ \hline
			\textbf{4} & Web Server Virtual Bytes & DB Server Privileged Time \\ \hline
			\textbf{5} & Web Server Page File Bytes Peak &  DB Server IO Other Bytes/sec \\ \hline
			\textbf{6} & DB Server Pool Nonpaged Bytes & DB Server Pool Nonpaged Bytes \\ \hline
			\textbf{7} & DB Server Page Faults/sec & -  \\ \hline
			\textbf{8} & DB Server Working Set & - \\ %\hline
			\midrule
			\textbf{$R^2$} & 85.30\% & 90.20\% \\ \hline
			%		\textbf{MAPE} & 15.63\% & 3.65\% \\ \hline
		\end{tabular}%
	}
\end{table}
\begin{table}[tbh]
	\centering
	\caption{Internal and external prediction errors for both subject systems.}
	\label{tab:errors}
	\resizebox{\textwidth}{!}{%
		\begin{tabular}{|c||c|c|c|c|c|c|c|c|c|}
			\hline
			\multicolumn{9}{|c|}{\textbf{DS2}} \\ \hline
			\textbf{Model Built} & \multicolumn{2}{c|}{\textbf{Validation}} & \textbf{Min.} & \textbf{1st Quart.} & \textbf{Median} & \textbf{Mean} & \textbf{3rd Quart.} & \textbf{Max}\\ %\hline
			\midrule
			\midrule
			\multirow{3}{*}{\textbf{Physical}} & \multicolumn{2}{c|}{\textbf{Internal Validation}} & 0.00 & 0.01 & 0.02 & 0.03 & 0.05 & 0.30 \\ \cline{2-9} 
			& \multirow{2}{*}{\textbf{External Validation}} & \textbf{Normalization by Deviance} & 0.00 & 0.08 & 0.25 & 0.36 & 0.49 & 13.65 \\ \cline{3-9} 
			&  & \textbf{Normalization by Load} & 0.00 & 0.34 & 0.44 & 0.48 & 0.56 & 1.56  \\ \hline
			\multirow{3}{*}{\textbf{Virtual}} & \multicolumn{2}{c|}{\textbf{Internal Validation}} & 0.00 & 0.04 & 0.09 & 0.11 & 0.15 & 0.54 \\ \cline{2-9} 
			& \multirow{2}{*}{\textbf{External Validation}} & \textbf{Normalization by Deviance} & 0.00 & 0.09 & 0.20 & 0.27 & 0.34 & 2.82  \\ \cline{3-9} 
			&  & \textbf{Normalization by Load} & 0.00 & 0.06 & 0.13 & 0.17 & 0.23 & 0.92 \\ \hline
		\end{tabular}%
	}
	
	
	\vspace{2ex}
	
	\centering
	\resizebox{\textwidth}{!}{%
		\begin{tabular}{|c||c|c|c|c|c|c|c|c|}
			\hline
			\multicolumn{9}{|c|}{\textbf{CloudStore}} \\ \hline
			\textbf{Model Built} & \multicolumn{2}{c|}{\textbf{Validation}} & \textbf{Min.} & \textbf{1st Quart.} & \textbf{Median} & \textbf{Mean} & \textbf{3rd Quart.} & \textbf{Max}\\ \midrule
			\midrule
			\multirow{3}{*}{\textbf{Physical}} & \multicolumn{2}{c|}{\textbf{Internal Validation}} & 0.00 & 0..05 & 0.10 & 0.16 & 0.18 & 2.68\\ \cline{2-9} 
			& \multirow{2}{*}{\textbf{External Validation}} & \textbf{Normalization by Deviance} & 0.00 & 0.04 & 0.09 & 0.17 & 0.17 & 2.29  \\ \cline{3-9} 
			&  & \textbf{Normalization by Load} & 2.90 & 5.14 & 6.32 & 7.75 & 8.08 & 51.33 \\ \hline
			\multirow{3}{*}{\textbf{Virtual}} & \multicolumn{2}{c|}{\textbf{Internal Validation}} & 0.00 & 0.01 & 0.03 & 0.04 & 0.05 & 0.50  \\ \cline{2-9} 
			& \multirow{2}{*}{\textbf{External Validation}} & \textbf{Normalization by Deviance} & 0.00 & 0.03 & 0.07 & 0.11 & 0.13 & 1.00  \\ \cline{3-9} 
			&  & \textbf{Normalization by Load} & 4.07 & 4.64 & 4.83 & 5.13 & 5.10 & 33.36  \\ \hline
		\end{tabular}%
	}
	\vspace{-0.3cm}
\end{table}

\noindent \textbf{Impact on the interpretation of examining statistical performance models.} Statistical performance models are often used to interpret relationships among many system performance metrics. For example, what are the significant metrics that are associated with system load and what performance metrics are redundant. Since the statistical performance models have large discrepancy, even after applying normalization techniques that is proposed by prior research, we cannot directly use the performance models built in the virtual environment. Even though our results show that normalizing by deviance can reduce the discrepancy, practitioners should still be aware of it when examining the performance models.

\noindent\fbox{%
	\parbox{\textwidth}{%
		\textbf{Findings: }We find that the statistical models built by performance testing results in an environment cannot advocate for the other environment due to discrepancies present. Normalization technique for heterogeneous environments and workloads that is proposed by prior research may not work for virtual and physical environment.\\
		\textbf{Actionable implications: }Normalizing the performance metrics by deviance may minimize such discrepancy and should be considered by practitioners before examining performance testing results. 
	}%
}

In Chapter~\ref{chapter4}, we further validate our findings of Chapter~\ref{chapter3} by looking at external factors that may have affected the nature of our performance tests and the subsequent analysis.