\documentclass[12pt]{report}


%\usepackage{color}
\usepackage{url}
\usepackage{xspace}
\usepackage{cuthesis}
\usepackage{booktabs}
\usepackage{makecell}
\usepackage{csquotes}
%\usepackage{pdflscape}
\usepackage{subfigure}
%\usepackage[framemethod=TikZ]{mdframed}
\usepackage{amssymb,amsmath}
\usepackage{multirow}
\usepackage{graphicx}
\usepackage{algorithm}
\usepackage{algorithmic}
\usepackage{fancybox}
\usepackage[flushleft]{threeparttable}
\usepackage{framed}
\usepackage{breakcites}


\linespread{1.6}

\author{Muhammad Moiz Arif}
\title {An Empirical Study on the Discrepancy between Performance Testing Results from Virtual and Physical Environments}
\degree{Master of Applied Science in Software Engineering}
\dept  {Computer Science and Software Engineering}

%%%%%%%%%%%%%%%%%%%%%%%%%%%%%%%%%%%%%%%%%%%%%%%%%%%%%%%%%
% define commands for frequently used words or phrases
%%%%%%%%%%%%%%%%%%%%%%%%%%%%%%%%%%%%%%%%%%%%%%%%%%%%%%%%%
% \newcommand{\SATD}{self-admitted technical debt\xspace}

%%%%%%%%%%%%%%%%%%%%%%%%%%%%%%%%%%%%%%%%%%%%%%%%%%%%%%%%%
% define commands for your RQs
%%%%%%%%%%%%%%%%%%%%%%%%%%%%%%%%%%%%%%%%%%%%%%%%%%%%%%%%%
\newcommand{\chapterIIIrqi}{\textbf{RQ:}}
\newcommand{\chapterIVrqi}{\textbf{RQ1. \\}}
\newcommand{\chapterIVrqii}{\textbf{RQ2. \\}}
\newcommand{\chapterIVrqiii}{\textbf{RQ3. \\}}


%%%%%%%%%%%%%%%%%%%%%%%%%%%%%%%%%%%%%%%%%%%%%%%%%%%%%%%%
% define command to create a conclusion box to answer your RQs
%%%%%%%%%%%%%%%%%%%%%%%%%%%%%%%%%%%%%%%%%%%%%%%%%%%%%%%%
\newcommand{\conclusionbox}[1]{%
    \vspace{6mm}
    \framebox[0.95\textwidth][c]{%
        \parbox[b]{0.92\textwidth}{%
            {\it #1}
        }
    }
    \vspace{6mm}
}


\begin{document}

\begin{abstract}

Large software systems often undergo performance tests to ensure their capability to handle expected loads. These performance tests often consume large amounts of computing resources and time since heavy loads need to be generated. Making it worse, the ever evolving field requires frequent updates to the performance testing environment. In practice, virtual machines (VMs) are widely exploited to provide flexible and less costly environments for performance tests. However, the use of VMs may introduce extra overhead (e.g., a higher than expected memory utilization) to the testing environment and lead to unrealistic performance testing results. Yet, little or no research has studied the overhead and impact on test results of using VMs in performance testing activities. 

In this dissertation we evaluate the discrepancy between the performance testing results from virtual and physical environments. We perform a case study on two open source systems - namely Dell DVD Store and CloudStore. We conduct the same performance tests in both virtual and physical environments and compare the performance testing results based on the three aspects that are typically examined for performance testing results: 1) individual performance metrics (e.g. CPU Time from virtual environment vs. CPU Time from physical environment), 2) the relationship among performance metrics (e.g. correlation between CPU and I/O) and 3) performance models that are built to predict system performance. Our results show that 1) individual performance metrics from virtual and physical environments do not follow the same distribution hence practitioners cannot simply use a scaling factor to compare the performance between environments,  2) correlations among performance metrics in virtual environments are different from those in physical environments 3) statistical models built based on the performance counters from virtual environments are different from the models built from physical environments suggesting that practitioners cannot use the performance testing results across virtual and physical environments. In order to assist the practitioners leverage performance testing results in both environments, we also investigate ways to transform results from virtual and physical environments and performance metrics based on deviance may reduced the discrepancy between performance metrics. Overall, we recommend that practitioners should not simply assume that performance testing results done on virtual environments will be the same in physical environments.

\end{abstract}

\begin{acknowledgments}

Thesis acknowledgments.

\end{acknowledgments}

\begin{publications}

The following publications are related to this thesis:

\begin{enumerate}

\item \textbf{...} 

%\item \textbf{Moiz}, 
\end{enumerate}

\end{publications}

\chapter{Introduction}
\label{introduction}
% -*- root: cuthesis_masters.tex -*-  

At the core of any software system is the software team who develop it...

\section{Research Hypothesis}

\conclusionbox{}  

\section{Thesis Overview}

\section{Thesis Contributions}

\chapter{Literature Review}
\label{literature_review}
% -*- root: cuthesis_masters.tex -*-

In this chapter we present related work on ...


\chapter{Testing For The Ubiety Of Discrepancy}
\label{chapter3}
% -*- root: cuthesis_masters.tex -*-

\section{Introduction}
\label{chap3:sec:introduction}


\section{Related Work}
\label{chap3:sec:related_work}

\section{Approach}
\label{chap3:sec:approach}

\begin{figure*}[thb!]
  \centering
  \includegraphics[width=1\textwidth]{figures/chapter3/approach.pdf}
  \caption{Approach Figure example}
  \label{chap3:fig:approach}
\end{figure*}

\section{Case Study Results}
\label{chap3:sec:results}

\vspace{3mm}
\noindent\chapterIIIrqi
\vspace{3mm}

\noindent\textbf{Motivation:} 

\vspace{1mm}
\noindent\textbf{Approach:} 

\vspace{1mm}
\noindent\textbf{Results:} 

\conclusionbox{Anwser of the Reserach Question}

\section{Threats to Validity}
\label{chap3:sec:threats_to_validity}

\section{Conclusion and Future Work}
\label{chap3:sec:conclusion}

\chapter{Studying The Impact Of Modifying The Virtual Environment}
\label{chapter4}

In the previous section, we find that there is a discrepancy between performance testing results from the virtual and physical environments. However, such discrepancy can also be due to other factors such 1) the instability of the virtual environments, 2) the virtual machine that we used or 3) the different hardware resources on the virtual environments. Therefore, in this section, we examine the impact of such factors to better understand our results. 


\subsection{Investigating the stability of virtual environments}

Thus far, we perform our case studies in one virtual environment and compare the performance metrics to the physical environment. However, the stability of the results obtained from the virtual environment need to be validated, in particular since VMs tend to be highly sensitive to the environment that they run in \cite{leitner}.

%A major challenge in our case studies was to make our virtual environment stable. If the performance of virtual machines are unstable, the observed discrepancy in Section~\ref{sec:results} may be due to the instability of virtual environment. We address the hypothesis that if our virtual environment was unstable, then the performance testing results should not be repeatable and congruent between different runs in the same virtual environment. 

In order to study whether the virtual environment is stable, we repeat the same performance tests, twice, on the virtual environments for both subject systems. In total, we had results from three performance tests. We perform the data analysis in Section~\ref{sec:model} by building statistical models using performance metrics. %Table~\ref{tab:stabilityvm} shows the median absolute percentage error from building a model using one virtual environment and testing on another virtual environment. 
As the previously mentioned approach, we build a model based on one of the runs, serving as our training data for the model, and tested it on another run. In this case, we define external validation when a model is trained on a different run than it is tested on. We validate our model by predicting the throughput of a different run.  
 
Prediction error values (see section 4.3.5) closer to 0 indicate that our model was able to successfully explain the variation of the throughput of a different run. This also means that the external validation error closer to 1 or higher depicts instability of the environment. We find the external validation error to be 0.04 and 0.13 for CloudStore and DS2, respectively. The internal validation error is 0.03 and 0.09 for CloudStore and DS2, respectively. Such low error values show that the performance testing results from the virtual environments are rather stable. 

\subsection{Investigating the Impact of Specific Virtual Machine Software}

In all of our experiments, we used the Virtual Box software to setup our virtual environment. However, there exist a plethora of VM software (i.e., it can be argued that our chosen subject systems behave differently in another environment). The question that arises then is whether the choice of VM software impacts our findings. In order to address the aforementioned hypothesis, we set up another virtual environment using VMWare (version 12) with the same allocated computing resources as when we set up Virtual Box.

To investigate this phenomenon, we repeat the performance tests for both subject systems. We train statistical models on the performance testing results from VMWare and test on the results from both the original virtual environment data (Virtual Box) and the results from the physical environments. We could not apply the normalization by deviance for the data from VMWare since some of the significant metrics in the model have a median absolute deviance of 0, making the normalized metric value to be infinite (see Equation~\ref{equ:mad}). We only apply the normalization by load. 

Table~\ref{tab:vmware} shows that the performance testing results from the two different virtual machine software is similar, as supported by the low percentage error when our model was tested on Virtual Box. In addition, the high error when predicting with physical environment agrees with the results when testing with the performance testing results from the Virtual Box (see Table~\ref{tab:errors}). Such results show that the discrepancy observed during our experiment also exits with the virtual environments that are set up with VMWare.

\begin{table}[tbh]
	\centering
	%\resizebox{\textwidth}{!}
	\caption{Median absolute percentage error from building a model using VMWare data.}
	\label{tab:vmware}
	\resizebox{\textwidth}{!}{\begin{tabular}{|c||c|c|}
			\hline
			\multirow{2}{*}{\textbf{Validation type}} & \multicolumn{2}{c|}{\textbf{Median absolute percentage error}} \\ \cline{2-3} 
			& \textbf{CloudStore} & \textbf{DS2} \\ %\hline
			\midrule
			\midrule
			External validation with Virtual Box results& 0.07&0.10\\ \hline
			%			External validation with physical normalization by deviance & 0.07 &0.06 \\ \hline
			External validation with physical normalization by load & 7.52& 1.63 \\ \hline
		\end{tabular}}
	\end{table}
	
	
\subsection{Investigating the Impact of Allocated Resources}

Another aspect that may impact our results is the resources allocated and the configuration of the virtual environment. We did not decrease the system resources as decreasing the resources may lead to crashes in the testing environment.

To investigate the impact of the allocated resources, we increase the computing resources allocated to the virtual environments by increasing the CPU to be 3 cores and increasing the memory to be 5GB. We cannot allocate more resource to the virtual environment since we need to keep resources for the hosting OS. We train statistical models on the new performance testing results and tested it on the performance testing results from the physical environment. 

Similar to the results shown in Table~\ref{tab:errors}, the prediction error is high when we normalize by load as per Equation~\ref{equ:mad} (1.57 for DS2 and 1.25 for CloudStore), while normalizing based on deviance can significantly reduce the error (0.09 for DS2 and 0.07 for CloudStore). We conclude that our findings still hold when the allocated resources are changed and this change has minimal impact on the results of our case studies.


\subsection{External validity.}
We chose two subject systems, CloudStore and DS2 for our study and two virtual machine software, VirtualBox and VMware. The two subject systems have years of history and prior performance engineering research has studied both systems~\cite{jackicsm2009,Nguyen:2012:ADP:2188286.2188344,tarekmsr16}. The virtual machine software that we used is widely used in practice. Nevertheless more case studies on other subject systems in other domains with other virtual machine software are needed to evaluate our findings. We also present our results based on our subject systems only and do not generalize for all the virtual machines.

%Also, we made sure that our virtual environment is set up exactly the same as our physical environment by keeping a constant checks aided by scripts. Having said that, this study can be boosted by additional subject systems being tested in other types of virtual environment. 

\subsection{Internal Validity.}
Our approach is based on the recorded performance metrics. The quality of recorded performance metrics can impact the internal validity of our study. Replicating our study by other performance monitoring tools, such as psutil~\cite{psutil} may address this threat. Even though we build a statistical model using performance metrics and system throughput, we do not assume that there is causal relationship. The use of statistical models merely aims to capture the relationship among multiple metrics. Similar approaches have been used in the prior studies~\cite{Cohen:2005:CIC:1095810.1095821, Shang:2015:ADP:2668930.2688052, xiong2013vperfguard}. 


%All of our models are dependent on the performance metrics' accuracy. Which means if the load on the server is beyond the capacity of the system to handle and builds up a queue, there is a possibility of noise sneaking in the recording process of the performance metrics. 
%The scaling approach we have adopted assumes that the for a particular metric, there exists alpha and beta values. However, if the metric value is constant it not possible to have an alpha and beta value associated with it in the model. Hence looking for an alternate scaling process is necessary for such a performance metric.
%We build performance regression models to compare the metrics from both of our environments. This models are accurate if there exists a high number of records for the performance counters. Additionally, we also assume that none of our dependent variable is correlated to the independent variable or vice-versa. 

\subsection{Construct Validity.}
We monitor the performance by recording performance metrics every 10 seconds and combine the performance metrics for every minute together as an average value. There may exist unfinished system requests when we record the system performance, leading to noise in our data. We choose a time interval (10 seconds) that is much higher than the response time of the requests (less than 0.1 second), in order to minimize the noise. Repeating our study by choosing other time interval sizes would address this threat. We exploit two approaches to normalize performance data from different environments. We also see that our {$R^2$} value is high. Although a higher {$R^2$} determines our model is accurate but it may also be an indication of overfit. There may exist other advance approaches to normalize performance data from heterogeneous environment. We plan to extend our study on other possible normalization approaches. There may exist other ways of examining performance testing results. We plan to extend our study by evaluating the discrepancy of using other ways of examining performance testing results in virtual and physical environments.



%We compared our distributions using Mann-Whitney \textit{U} Tests. As this test is commonly used to compare two distributions, other tests like T-test can also be used for this purpose. 
%We used R's \textit{step()} function to build our generalized linear regression models. This function automatically selects the variables contributing most to the models however the models might be different if they are only based on p-values. To address this, we plan to build models with a predefined threshold for the p-values.


\chapter{Summary, Contributions and Future Work}
\label{conclusion}
Performance assurance activities are vital in ensuring software reliability. Virtual environments are often used to conduct performance tests. However, the discrepancy between performance testing results in virtual and physical environments are never evaluated. In this paper, we evaluate such discrepancy by conducting performance tests on two open source systems (DS2 and CloudStore) in both virtual and physical environments. By examine the performance testing results, we find that there exist discrepancy between performance testing results in virtual and physical environments when examining individual performance metrics, relationship among performance metrics and building statistical models from performance metrics, even after we normalize performance metrics across different environments. The major contribution of this paper includes: 
%\vspace{-0.15cm}
\begin{itemize} \itemsep -0.8pt 
	\item Our paper is the first research attempt to evaluate the discrepancy between performance testing results in virtual and physical environments.
	\item We find that relationships among I/O related metrics have large differences between virtual and physical environments.
	\item We find that normalizing performance metrics based on deviance may reduce the discrepancy. Practitioners may exploit such normalization techniques when analyzing performance testing results from virtual environments.
\end{itemize}
%\vspace{-0.15cm}
Our results highlight the needs of awareness of discrepancy between performance testing results in virtual and physical environments, for both practitioners and researchers. Future research effort may focus on minimizing such discrepancy in order to improve the use of virtual environments in performance engineering and reliability assurance activities



%Our paper magnifies the impact of performance in difference environments. Additionally, we observed that the performance metrics from different environment do not belong to the same distribution. We tried to mitigate this impact by using scaled metrics in our performance regression models. As a result, the percentage error dropped drastically for one of our subject systems are increased for the other. We concluded that scaling may not apply to every subject system. We plan to see in what ways we can leverage the metrics from the virtual environment and use them for prediction of the metrics in the physical environment. We also plan to investigate the injection of regression in the system being hosted in a virtual environment will behave similar to that under regression in a physical environment. 

%\addcontentsline{toc}{chapter}{Bibliography}
\bibliography{bibliography.bib}  
\bibliographystyle{alpha}
\end{document}