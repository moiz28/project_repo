\documentclass[12pt]{report}


%\usepackage{color}
\usepackage{url}
\usepackage{xspace}
\usepackage{cuthesis}
\usepackage{booktabs}
\usepackage{makecell}
\usepackage{csquotes}
%\usepackage{pdflscape}
\usepackage{subfigure}
%\usepackage[framemethod=TikZ]{mdframed}
\usepackage{amssymb,amsmath}
\usepackage{multirow}
\usepackage{graphicx}
\usepackage{algorithm}
\usepackage{algorithmic}
\usepackage{fancybox}
\usepackage[flushleft]{threeparttable}
\usepackage{framed}
\usepackage{breakcites}
\usepackage{microtype}
\usepackage{pdflscape}


\linespread{1.6}

\author{Muhammad Moiz Arif}
\title {An Empirical Study on the Discrepancy between Performance Testing Results from Virtual and Physical Environments}
\degree{Master of Applied Science in Software Engineering}
\dept  {Computer Science and Software Engineering}

%%%%%%%%%%%%%%%%%%%%%%%%%%%%%%%%%%%%%%%%%%%%%%%%%%%%%%%%%
% define commands for frequently used words or phrases
%%%%%%%%%%%%%%%%%%%%%%%%%%%%%%%%%%%%%%%%%%%%%%%%%%%%%%%%%
% \newcommand{\SATD}{self-admitted technical debt\xspace}

%%%%%%%%%%%%%%%%%%%%%%%%%%%%%%%%%%%%%%%%%%%%%%%%%%%%%%%%%
% define commands for your RQs
%%%%%%%%%%%%%%%%%%%%%%%%%%%%%%%%%%%%%%%%%%%%%%%%%%%%%%%%%
\newcommand{\chapterIIIrqi}{\textbf{RQ:}}
\newcommand{\chapterIVrqi}{\textbf{RQ1. \\}}
\newcommand{\chapterIVrqii}{\textbf{RQ2. \\}}
\newcommand{\chapterIVrqiii}{\textbf{RQ3. \\}}


%%%%%%%%%%%%%%%%%%%%%%%%%%%%%%%%%%%%%%%%%%%%%%%%%%%%%%%%
% define command to create a conclusion box to answer your RQs
%%%%%%%%%%%%%%%%%%%%%%%%%%%%%%%%%%%%%%%%%%%%%%%%%%%%%%%%
\newcommand{\conclusionbox}[1]{%
    \vspace{6mm}
    \framebox[0.95\textwidth][c]{%
        \parbox[b]{0.92\textwidth}{%
            {\it #1}
        }
    }
    \vspace{6mm}
}


\begin{document}

\begin{abstract}
Large software systems often undergo performance tests to ensure their capability to handle expected loads. These performance tests often consume large amounts of computing resources and time since heavy loads need to be generated. Making it worse, the ever evolving field requires frequent updates to the performance testing environment. In practice, virtual machines (VMs) are widely exploited to provide flexible and less costly environments for performance tests. However, the use of VMs may introduce confounding overhead (e.g., a higher than expected memory utilization with unstable I/O traffic) to the testing environment and lead to unrealistic performance testing results. Yet, little research has studied the impact on test results of using VMs in performance testing activities. 

To evaluate the discrepancy between the performance testing results from virtual and physical environments, we perform a case study on two open source systems -- namely Dell DVD Store (DS2) and CloudStore. We conduct the same performance tests in both virtual and physical environments and compare the performance testing results based on the three aspects that are typically examined for performance testing results: 1) single performance metric (e.g. CPU Time from virtual environment vs. CPU Time from physical environment), 2) the relationship among performance metrics (e.g. correlation between CPU and I/O) and 3) performance models that are built to predict system performance. Our results show that 1) A single metric from virtual and physical environments do not follow the same distribution, hence practitioners cannot simply use a scaling factor to compare the performance between environments,  2) correlations among performance metrics in virtual environments are different from those in physical environments 3) statistical models built based on the performance metrics from virtual environments are different from the models built from physical environments suggesting that practitioners cannot use the performance testing results across virtual and physical environments. In order to assist the practitioners leverage performance testing results in both environments, we investigate ways to reduce the discrepancy. We find that such discrepancy can be reduced by normalizing performance metrics based on deviance. Overall, we suggest that practitioners should not use the performance testing results from virtual environment with the simple assumption of straightforward performance overhead. Instead, practitioners should consider leveraging normalization techniques to reduce the discrepancy before examining performance testing results from virtual and physical environments.


\end{abstract}

\begin{acknowledgments}

Thesis acknowledgments.

\end{acknowledgments}

\begin{dedication}
	
	Dedication
	
\end{dedication}

\begin{publications}

The following publications are related to this thesis:

\begin{enumerate}

\item \textbf{...} 

%\item \textbf{Moiz}, 
\end{enumerate}

\end{publications}

\chapter{Introduction}
\label{introduction}
% -*- root: cuthesis_masters.tex -*-  

At the core of any software system is the software team who develop it...

\section{Research Hypothesis}

\conclusionbox{}  

\section{Thesis Overview}

\section{Thesis Contributions}

\chapter{Background and Literature Review}
\label{background_and_literature_review}
% -*- root: cuthesis_masters.tex -*-

%In this chapter we present related work on ...

Software systems are expected to serve millions of concurrent requests~\cite{arlitt2000workload}. However, the systems are first tested to ensure that they are working correctly under a certain load(s). This load is also known as the rate at which the system is processing the requests. 
In this chapter, we discuss the motivation behind our work and similar studies in the field of performance engineering. We later survey the state-of-the-art literature that is related to our work.

\section{Background}

Generally, performance assurance activities are carried out by analyzing the system responses on a variety of workloads. For example, to detect performance issues, the system performance is analyzed after applying a workload. This workload profile depicts the normal workload of the system once it is functional in the field~\cite{464549}.


\subsection{Performance Testing}

Although there may be some similarities present between performance and load testing, performance testing is mainly focused to detect primarily performance related performance with the software system. For example, response time, throughput and resource utilization.~\cite{Barna:2011,6032540,Gorton}

Performance test is detailed than load testing as it may be used to cover designs ~\cite{csurgay1999performance,denaro2004early,denaro2005performance}, algorithms ~\cite{cangussu2009segment,cangussu2007reducing}, and system configurations ~\cite{hoskins2005software,pozin2011models,sopitkamol2005method}. The goal behind performance testing may be to test performance requirements~\cite{pozin2011models} or exploratory. For example, to answer questions such as how do various configurations impact the performance of the system ~\cite{Menasce:2000,Menasce:1994,Menasce:2001,pozin2011models}.

\subsubsection{What is a test execution?}



Before the phase of text execution, practitioners have to develop the test for the system. It is based on either the realistic usage of the system when functional or with the goal to uncover problems \cite{jiang2015survey}.
Once the test is developed, it is followed by its execution. 

\begin{figure*}[thb!]
	\centering
	\includegraphics[width=1\textwidth]{figures/bck1.pdf}
	\caption{Test execution phases}
	\label{fig:test_phases}
\end{figure*}

As shown in figure \ref{fig:test_phases}, the lifecycle of a typical performance test is made up of four phases. 

\subsubsection{Setup and Testing}

~\textit{Setup}, that is system and test execution setup. ~\textit{Testing}, that is the actual phase where the test is run and terminated at the end of the time frame. ~\textit{Data Collection}, that is the phase where the performance metrics and executions logs are collected and further analyzed in the fourth phase \cite{jiang2015survey}.

The term~\textit{setup} is further divided into two sub-terms. The setup for the system and the setup for the text execution system. The former deals with making sure that the system or~\textit{SUT} is operational and fully functional. This may also include setting up servers and other functionalities attached to the system for example, database servers. The latter engulfs configuring and using the load test drivers (for example, WebLoad~\cite{webload}, HP LoadRunner~\cite{loadrunner}, Apache Jmeter ~\cite{apachejmeter}) in addition to setting up the testing environment. In order to record the performance metrics, performance monitoring tools are set up, e.g. Perfmon~\cite{perfmon}, Psutil~\cite{psutil}, etc. 

Following the setup, the system is tested by applying load. The results are concurrently recorded. Practitioners may terminate the load based on the following techniques:

\begin{itemize}
	\item Continuous: Testing until it is topped manually by practitioners~\cite{4017687}.
	\item Timer-Based: The test runs for a specific duration of time~\cite{4017687}.
	\item Counter-Based: The load is stopped after a certain number of requests sent or received~\cite{4017687}.
	\item Statistic-Based: A comparatively newer technique where the metric of interest is captures till it is statistically stable. This serves a high confidence level when analyzing the metric~\cite{mansharamani2010performance,snellman2011towards}.
\end{itemize}

\subsubsection{Data and Metrics Collection}

During the course of loading the system and running the tests, the data is monitored and recored. The data collected is in the form of logs and performance metrics. These performance metrics may be higher-level (e.g. throughput, response time, etc.) or lower-level (e.g. CPU utilization, memory usage, etc.) It is necessary that the monitoring or recording tool does not induce extra overhead on the system. This may lead to biased results~\cite{mytkowicz2010evaluating}.

\subsubsection{Data Analysis}
We discuss in detail the approaches to analyze data in section \ref{sec:related}.


\subsection{What are the differences between load testing, stress testing and performance testing?}

Although there are similarities between with these three types of testing techniques, for example all of these are carried out after functional testing, we now differentiate between the application of these tests.

\subsubsection{Load Testing}

The rate at which which the requests are submitted to a system is called the load~\cite{Beizer:1984}. This system is more commonly known as system under test or~\textit{SUT}. Load testing is carried out later than conventional testing in a software's life. This may be done on a prototype or a working system than a design or a model. Load testing is used to detect load related problems. For example, deadlocks, buffer overflows, high response times and low throughput~\cite{464549,Barna:2011,6032540}. 

In some exceptional cases where the non-functional requirements are not present, one of ways to determine if the test has passed is by comparing it to the results of the previous version. This is also known as the "no-worse-than-before" principle. As the name suggests, the version being tested should be, if not better, equal to the previous version.~\cite{Dumke:2001}

\subsubsection{Stress Testing}

Stress testing is testing the application under "stress" or extreme load. This can be used to detect how resilient the system or to detect further load-related problems. For example, memory leaks and deadlocks. These "stressful" conditions for the system can be either load related(normal ~\cite{zhang2002automated,kalita2011investigation,chakravarty2010stress} or heavy load ~\cite{Dillenseger2009,kalita2011investigation,huebner2001performance}) or limited resources allocated/failures (for example, disk or database failures) ~\cite{acharya2009mining}. We noted that in some cases it may also be used to detect the competency of the SUT \cite{garousi2010genetic,garousi2008empirical,garousi2006traffic,garousi2008traffic}.


Having stated all of the above, there are instances where the interest of a performance test may overlap with load testing pr stress testing and vice-versa. For example, to check robustness of the system when put in extreme conditions, or to check how an algorithm works when handling large files. We observe that the terms performance testing~\cite{Dillenseger2009,Menasce02loadtesting,Menasce:2002}, load testing ~\cite{536461,Bayan:2008,perf_load_stress,perf_web} and stress testing~\cite{Bayan:2008,Yang:1996,4020172} are also used interchangeably. 

We focus on performance testing, which is to detect performance related behavior of our~\textit{SUT}.


\section{Literature Review}
\label{sec:related}
In this section, we discuss the motivation and related work of this thesis in broadly three subsections: 1) analyzing performance metrics from performance testing, 2) analysis of VM overhead and 3) performance testing and bug detection.


\subsection{Analyzing performance testing results} 

Prior research has proposed a slew of techniques to analyze performance testing results, i.e. performance metrics. Such techniques typically examine three different aspects of the metrics: 1) single performance metric, 2) the relationship between performance metrics, and 3) statistical modeling based on performance metrics.


\subsubsection{Single performance metric}
\label{sec:relatedindividual}
Nguyen \textit{et al$.$}~\cite{Nguyen:2012:ADP:2188286.2188344} introduce the concept of using control charts~\cite{shewhart1931economic} in order to detect performance regressions. Control charts use a predefined threshold to detect performance anomalies. However control charts assume that the output follows a uni-model distribution, which may be an inappropriate assumption for performance. Nguyen \textit{ et al$.$} propose an approach to normalize performance metrics between heterogeneous environments and workloads in order to build robust control charts. %However, the experiments are only carried out on the virtual machines in contrast to our approach.

Malik \emph{et al$.$}~\cite{Malik:2010:ACL:1955601.1955936, haroon} propose approaches that cluster performance metrics using Principal Component Analysis (PCA). Each component generated by PCA is mapped to performance metrics by a weight value. The weight value measures how much a metric contributes to the component. For every performance metric, a comparison is performed on the weight value of each component to detect performance regressions.

Heger \emph{et al$.$}~\cite{DBLP:conf/wosp/HegerHF13} present an approach that uses software development history and unit tests to diagnose the root cause of performance regressions. In the first step of their approach, they leverage Analysis of Variance (ANOVA) to compare the response time of the system to detect performance regressions. Similarly, Jiang \emph{et al$.$}~\cite{jackicsm2009} extract response time from system logs. Instead of conducting statistical tests, Jiang \emph{et al$.$} visualize the trend of response time during performance tests, in order to identify performance issues.


\subsubsection{Relationship between performance metrics}
\label{sec:relatedrelation}

Malik \emph{et al$.$}~\cite{5635038} leverage Spearman's rank correlation to capture the relationship between performance metrics. The deviance of correlation is calculated in order to pinpoint which subsystem should take responsibility of the performance deviation.

Foo \emph{ et al$.$}~\cite{foo2010mining} propose an approach that leverages association rules in order to address the limitations of manually detecting performance regressions in large scale software systems. Association rules capture the historical relationship among performance metrics and generate rules based on the results of prior performance tests. Deviations in the association rules are considered signs of performance regressions.

Jiang \emph{et al$.$}~\cite{5270324} use normalized mutual information as a similarity measure to cluster correlated performance metrics. Since metrics in one cluster are highly correlated, the uncertainty among metrics in the cluster should be low. Jiang \emph{et al$.$} leverage entropy from information theory to monitor the uncertainty of each cluster. A significant change in the entropy is considered as a sign of a performance fault. 


\subsubsection{Statistical modeling based on performance metrics}
\label{sec:relatedmodel}

Xiong \textit{et al$.$}~\cite{xiong2013vperfguard} proposed a model-driven approach named \textit{vPerfGuard} to detect software performance regressions in a cloud-environment. The approach builds models between workload metrics and a performance metric, such as CPU. The models can be used to detect workload changes and assists in identifying performance bottlenecks. Since the usage of \emph{vPerfGuard} is typically in a virtual environment, our study may help the future evaluation of \textit{vPerfGuard}. Similarly, Shang \textit{ et al.}~\cite{Shang:2015:ADP:2668930.2688052} propose an approach of including only a limited number of performance metrics for building statistical models. The approach leverages an automatic clustering technique in order to find the number of models to be build for the performance testing results. By building statistical models for each cluster, their approach is applicable to detect injected performance regressions. 

Cohen \textit{et al$.$}~\cite{cohen2004correlating} propose an approach that builds probabilistic models, such as Tree-Augmented Bayesian Networks, to examine the causes that target the changes in the system's response time. Cohen \textit{et al$.$}~\cite{Cohen:2005:CIC:1095810.1095821} also proposed that system faults can be detected by building statistical models based on performance metrics. The approaches of Cohen \textit{et al$.$}~\cite{cohen2004correlating, Cohen:2005:CIC:1095810.1095821} were improved by Bodik \textit{et al.}~\cite{bodik2008hilighter} by using logistic regression models.

Jiang \emph{et al$.$}~\cite{Jiang:2009:SMM:1555228.1555233} propose an approach that improves the Ordinary Least Squares regression models that are built from performance metrics and use the model to detect faults in a system. The authors conclude that their approach is more efficient in successfully detecting the injected faults than the current linear-model approach.

On one hand, none of the prior research discusses the impact of their approaches results in virtual and physical environments, which motivates the empirical study that is conducted in this thesis. On the other hand, since there are hardly two identical performance testing results, we do no compare the raw data of performance testing results from virtual and physical environments. Instead, we conduct our case study in the context of all the above three types of analyses, in order to see the impact when practitioners use such analyses on performance testing results. Our findings can help better evaluate and understand the findings from the aforementioned research. 



\subsection{Analysis of VM overhead}

Kraft \textit{et al$.$}~\cite{kraft2011io} discuss the issues related to disk I/O in a virtual environment. They examine the performance degradation of disk request response time by recommending a trace-driven approach. Kraft \textit{et al.} emphasize on the latencies existing in virtual machine requests for disc IO due to increments in time associated with request queues. 

Aravind \textit{et al$.$}~\cite{menon2005diagnosing} audit the performance overhead in Xen virtual machines. They uncover the origins of overhead that might exist in the network I/O causing a peculiar system behavior. However, there study is limited to Xen virtual machine only while mainly focusing on network related performance overhead.

Brosig \textit{et al$.$}~\cite{brosig2013evaluating} predict the performance overhead of virtualized environments using Petri-nets in Xen server. The authors focused on the visualization overhead with respect to queuing networks only. The authors were able to accurately predict server utilization but had significant errors for multiple VMs.


Huber \textit{et al$.$}~\cite{huber2011evaluating} present a study on cloud-like environments. The authors compare the performance of virtual environments and study the degradation between the two environments. Huber \textit{et al$.$} further categorize factors that influence the overhead and use regression based models to evaluate the overhead. However, the modeling only considers CPU and memory.


Luo \textit{et al$.$}~\cite{Luo:2016:MPR:2901739.2901765} converge the set of inputs that may cause software regression. They apply genetic algorithms to detect such combinations. Netto \textit{et al$.$} \cite{netto2011evaluating} present a similar study to compare performance metrics generated via load tests between the two environments. However, the author did not analyse the results from a statistical perspective.

Prior research focused on the overhead of virtual environments without considering the impact of such overhead on performance testing and assurance activities. In this thesis, we evaluate the discrepancy between virtual and physical environments by focusing on the impact of performance testing results analyses and investigate whether such impact can be reduced in practice.


\subsection{Performance testing and bug detection}

There exists much research on performance testing and bug detection. Nistor \textit{et al$.$}~\cite{Nistor} detect the presence of functional and loop-related performance bugs with the help of their developed tool. Jin \textit{et al$.$} \cite{Jin} present a study on a wide range of performance bugs. The authors examined real-world performance bugs and developed rule-based performance bug detection tools. Nistor \textit{et al$.$}~\cite{nistor_2} in another study highlight that automated tool based performance bug detection is limited. The authors also comment that performance bugs are mostly detected by code reasoning rather than seeing the effects of the system by the end users. Tsakiltsidis \textit{et al$.$}~\cite{Tsakiltsidis} use prediction models to detect and predict performance bugs based on extraction from source code repositories. Malik \textit{et al$.$}~\cite{h_malik_p_bugs} present a study to uncover functional bugs via load testing. The authors propose an approach to reduce the large amount of performance metrics at the end of a load test by principal component analysis. Zaman \textit{et al$.$}~\cite{zaman_p_bugs} study the tracking and fixing of performance bugs.

However, none of the above mentioned performance bug detection approach has been applied in different environments. In most of the cases, the environment is not explicitly mentioned. Hence, to generalize the findings across environments remains an open topic.


\chapter{Testing For The Ubiety Of Discrepancy}
\label{chapter3}
% -*- root: cuthesis_masters.tex -*-

\section{Introduction}
\label{chap3:sec:introduction}


\section{Related Work}
\label{chap3:sec:related_work}

\section{Approach}
\label{chap3:sec:approach}

\begin{figure*}[thb!]
  \centering
  \includegraphics[width=1\textwidth]{figures/chapter3/approach.pdf}
  \caption{Approach Figure example}
  \label{chap3:fig:approach}
\end{figure*}

\section{Case Study Results}
\label{chap3:sec:results}

\vspace{3mm}
\noindent\chapterIIIrqi
\vspace{3mm}

\noindent\textbf{Motivation:} 

\vspace{1mm}
\noindent\textbf{Approach:} 

\vspace{1mm}
\noindent\textbf{Results:} 

\conclusionbox{Anwser of the Reserach Question}

\section{Threats to Validity}
\label{chap3:sec:threats_to_validity}

\section{Conclusion and Future Work}
\label{chap3:sec:conclusion}

\chapter{Studying The Impact Of Modifying The Virtual Environment}
\label{chapter4}

In the previous section, we find that there is a discrepancy between performance testing results from the virtual and physical environments. However, such discrepancy can also be due to other factors such 1) the instability of the virtual environments, 2) the virtual machine that we used or 3) the different hardware resources on the virtual environments. Therefore, in this section, we examine the impact of such factors to better understand our results. 


\subsection{Investigating the stability of virtual environments}

Thus far, we perform our case studies in one virtual environment and compare the performance metrics to the physical environment. However, the stability of the results obtained from the virtual environment need to be validated, in particular since VMs tend to be highly sensitive to the environment that they run in \cite{leitner}.

%A major challenge in our case studies was to make our virtual environment stable. If the performance of virtual machines are unstable, the observed discrepancy in Section~\ref{sec:results} may be due to the instability of virtual environment. We address the hypothesis that if our virtual environment was unstable, then the performance testing results should not be repeatable and congruent between different runs in the same virtual environment. 

In order to study whether the virtual environment is stable, we repeat the same performance tests, twice, on the virtual environments for both subject systems. In total, we had results from three performance tests. We perform the data analysis in Section~\ref{sec:model} by building statistical models using performance metrics. %Table~\ref{tab:stabilityvm} shows the median absolute percentage error from building a model using one virtual environment and testing on another virtual environment. 
As the previously mentioned approach, we build a model based on one of the runs, serving as our training data for the model, and tested it on another run. In this case, we define external validation when a model is trained on a different run than it is tested on. We validate our model by predicting the throughput of a different run.  
 
Prediction error values (see section 4.3.5) closer to 0 indicate that our model was able to successfully explain the variation of the throughput of a different run. This also means that the external validation error closer to 1 or higher depicts instability of the environment. We find the external validation error to be 0.04 and 0.13 for CloudStore and DS2, respectively. The internal validation error is 0.03 and 0.09 for CloudStore and DS2, respectively. Such low error values show that the performance testing results from the virtual environments are rather stable. 

\subsection{Investigating the Impact of Specific Virtual Machine Software}

In all of our experiments, we used the Virtual Box software to setup our virtual environment. However, there exist a plethora of VM software (i.e., it can be argued that our chosen subject systems behave differently in another environment). The question that arises then is whether the choice of VM software impacts our findings. In order to address the aforementioned hypothesis, we set up another virtual environment using VMWare (version 12) with the same allocated computing resources as when we set up Virtual Box.

To investigate this phenomenon, we repeat the performance tests for both subject systems. We train statistical models on the performance testing results from VMWare and test on the results from both the original virtual environment data (Virtual Box) and the results from the physical environments. We could not apply the normalization by deviance for the data from VMWare since some of the significant metrics in the model have a median absolute deviance of 0, making the normalized metric value to be infinite (see Equation~\ref{equ:mad}). We only apply the normalization by load. 

Table~\ref{tab:vmware} shows that the performance testing results from the two different virtual machine software is similar, as supported by the low percentage error when our model was tested on Virtual Box. In addition, the high error when predicting with physical environment agrees with the results when testing with the performance testing results from the Virtual Box (see Table~\ref{tab:errors}). Such results show that the discrepancy observed during our experiment also exits with the virtual environments that are set up with VMWare.

\begin{table}[tbh]
	\centering
	%\resizebox{\textwidth}{!}
	\caption{Median absolute percentage error from building a model using VMWare data.}
	\label{tab:vmware}
	\resizebox{\textwidth}{!}{\begin{tabular}{|c||c|c|}
			\hline
			\multirow{2}{*}{\textbf{Validation type}} & \multicolumn{2}{c|}{\textbf{Median absolute percentage error}} \\ \cline{2-3} 
			& \textbf{CloudStore} & \textbf{DS2} \\ %\hline
			\midrule
			\midrule
			External validation with Virtual Box results& 0.07&0.10\\ \hline
			%			External validation with physical normalization by deviance & 0.07 &0.06 \\ \hline
			External validation with physical normalization by load & 7.52& 1.63 \\ \hline
		\end{tabular}}
	\end{table}
	
	
\subsection{Investigating the Impact of Allocated Resources}

Another aspect that may impact our results is the resources allocated and the configuration of the virtual environment. We did not decrease the system resources as decreasing the resources may lead to crashes in the testing environment.

To investigate the impact of the allocated resources, we increase the computing resources allocated to the virtual environments by increasing the CPU to be 3 cores and increasing the memory to be 5GB. We cannot allocate more resource to the virtual environment since we need to keep resources for the hosting OS. We train statistical models on the new performance testing results and tested it on the performance testing results from the physical environment. 

Similar to the results shown in Table~\ref{tab:errors}, the prediction error is high when we normalize by load as per Equation~\ref{equ:mad} (1.57 for DS2 and 1.25 for CloudStore), while normalizing based on deviance can significantly reduce the error (0.09 for DS2 and 0.07 for CloudStore). We conclude that our findings still hold when the allocated resources are changed and this change has minimal impact on the results of our case studies.


\subsection{External validity.}
We chose two subject systems, CloudStore and DS2 for our study and two virtual machine software, VirtualBox and VMware. The two subject systems have years of history and prior performance engineering research has studied both systems~\cite{jackicsm2009,Nguyen:2012:ADP:2188286.2188344,tarekmsr16}. The virtual machine software that we used is widely used in practice. Nevertheless more case studies on other subject systems in other domains with other virtual machine software are needed to evaluate our findings. We also present our results based on our subject systems only and do not generalize for all the virtual machines.

%Also, we made sure that our virtual environment is set up exactly the same as our physical environment by keeping a constant checks aided by scripts. Having said that, this study can be boosted by additional subject systems being tested in other types of virtual environment. 

\subsection{Internal Validity.}
Our approach is based on the recorded performance metrics. The quality of recorded performance metrics can impact the internal validity of our study. Replicating our study by other performance monitoring tools, such as psutil~\cite{psutil} may address this threat. Even though we build a statistical model using performance metrics and system throughput, we do not assume that there is causal relationship. The use of statistical models merely aims to capture the relationship among multiple metrics. Similar approaches have been used in the prior studies~\cite{Cohen:2005:CIC:1095810.1095821, Shang:2015:ADP:2668930.2688052, xiong2013vperfguard}. 


%All of our models are dependent on the performance metrics' accuracy. Which means if the load on the server is beyond the capacity of the system to handle and builds up a queue, there is a possibility of noise sneaking in the recording process of the performance metrics. 
%The scaling approach we have adopted assumes that the for a particular metric, there exists alpha and beta values. However, if the metric value is constant it not possible to have an alpha and beta value associated with it in the model. Hence looking for an alternate scaling process is necessary for such a performance metric.
%We build performance regression models to compare the metrics from both of our environments. This models are accurate if there exists a high number of records for the performance counters. Additionally, we also assume that none of our dependent variable is correlated to the independent variable or vice-versa. 

\subsection{Construct Validity.}
We monitor the performance by recording performance metrics every 10 seconds and combine the performance metrics for every minute together as an average value. There may exist unfinished system requests when we record the system performance, leading to noise in our data. We choose a time interval (10 seconds) that is much higher than the response time of the requests (less than 0.1 second), in order to minimize the noise. Repeating our study by choosing other time interval sizes would address this threat. We exploit two approaches to normalize performance data from different environments. We also see that our {$R^2$} value is high. Although a higher {$R^2$} determines our model is accurate but it may also be an indication of overfit. There may exist other advance approaches to normalize performance data from heterogeneous environment. We plan to extend our study on other possible normalization approaches. There may exist other ways of examining performance testing results. We plan to extend our study by evaluating the discrepancy of using other ways of examining performance testing results in virtual and physical environments.



%We compared our distributions using Mann-Whitney \textit{U} Tests. As this test is commonly used to compare two distributions, other tests like T-test can also be used for this purpose. 
%We used R's \textit{step()} function to build our generalized linear regression models. This function automatically selects the variables contributing most to the models however the models might be different if they are only based on p-values. To address this, we plan to build models with a predefined threshold for the p-values.


\chapter{Summary, Contributions and Future Work}
\label{conclusion}
Performance assurance activities are vital in ensuring software reliability. Virtual environments are often used to conduct performance tests. However, the discrepancy between performance testing results in virtual and physical environments are never evaluated. In this paper, we evaluate such discrepancy by conducting performance tests on two open source systems (DS2 and CloudStore) in both virtual and physical environments. By examine the performance testing results, we find that there exist discrepancy between performance testing results in virtual and physical environments when examining individual performance metrics, relationship among performance metrics and building statistical models from performance metrics, even after we normalize performance metrics across different environments. The major contribution of this paper includes: 
%\vspace{-0.15cm}
\begin{itemize} \itemsep -0.8pt 
	\item Our paper is the first research attempt to evaluate the discrepancy between performance testing results in virtual and physical environments.
	\item We find that relationships among I/O related metrics have large differences between virtual and physical environments.
	\item We find that normalizing performance metrics based on deviance may reduce the discrepancy. Practitioners may exploit such normalization techniques when analyzing performance testing results from virtual environments.
\end{itemize}
%\vspace{-0.15cm}
Our results highlight the needs of awareness of discrepancy between performance testing results in virtual and physical environments, for both practitioners and researchers. Future research effort may focus on minimizing such discrepancy in order to improve the use of virtual environments in performance engineering and reliability assurance activities



%Our paper magnifies the impact of performance in difference environments. Additionally, we observed that the performance metrics from different environment do not belong to the same distribution. We tried to mitigate this impact by using scaled metrics in our performance regression models. As a result, the percentage error dropped drastically for one of our subject systems are increased for the other. We concluded that scaling may not apply to every subject system. We plan to see in what ways we can leverage the metrics from the virtual environment and use them for prediction of the metrics in the physical environment. We also plan to investigate the injection of regression in the system being hosted in a virtual environment will behave similar to that under regression in a physical environment. 

\chapter{Appendix}
\label{appendix}
Following is the complete set of Q-Q plots and heatmaps for both our subject systems.


%
\begin{figure}[tbh]
	\centering
	{\includegraphics[width=1.0\textwidth]{figures/appendix/qq_plots/DS2/Web_Server/First_six.pdf}}
	\caption{Q-Q plots for DS2's Web Server}
	%\captionsetup{justification=centering}
\end{figure}

\begin{figure}[tbh]
	\centering
	{\includegraphics[width=1.0\textwidth]{figures/appendix/qq_plots/DS2/Web_Server/Second_six.pdf}}
	\caption{Q-Q plots for DS2's Web Server}
	%\captionsetup{justification=centering}
\end{figure}

\begin{figure}[tbh]
	\centering
	{\includegraphics[width=1.0\textwidth]{figures/appendix/qq_plots/DS2/Web_Server/Third_six.pdf}}
	\caption{Q-Q plots for DS2's Web Server}
	%\captionsetup{justification=centering}
\end{figure}

\begin{figure}[tbh]
	\centering
	{\includegraphics[width=1.0\textwidth]{figures/appendix/qq_plots/DS2/Web_Server/Fourth_four.pdf}}
	\caption{Q-Q plots for DS2's Web Server}
	%\captionsetup{justification=centering}
\end{figure}
%%%%%%%%%%%%%%%%%%%%%%%%%%%%%%

\begin{figure}[tbh]
	\centering
	{\includegraphics[width=1.0\textwidth]{figures/appendix/qq_plots/DS2/DB_Server/First_six.pdf}}
	\caption{Q-Q plots for DS2's DB Server}
	%\captionsetup{justification=centering}
\end{figure}

\begin{figure}[tbh]
	\centering
	{\includegraphics[width=1.0\textwidth]{figures/appendix/qq_plots/DS2/DB_Server/Second_six.pdf}}
	\caption{Q-Q plots for DS2's DB Server}
	%\captionsetup{justification=centering}
\end{figure}

\begin{figure}[tbh]
	\centering
	{\includegraphics[width=1.0\textwidth]{figures/appendix/qq_plots/DS2/DB_Server/Third_six.pdf}}
	\caption{Q-Q plots for DS2's DB Server}
	%\captionsetup{justification=centering}
\end{figure}

\begin{figure}[tbh]
	\centering
	{\includegraphics[width=1.0\textwidth]{figures/appendix/qq_plots/DS2/DB_Server/Fourth_four.pdf}}
	\caption{Q-Q plots for DS2's DB Server}
	%\captionsetup{justification=centering}
\end{figure}





%%%%%%%%%%%%%%%%%%%%%%%%%%%%%%%%%%%%%%%%%%%%%%%%


\begin{figure}[tbh]
	\centering
	{\includegraphics[width=1.0\textwidth]{figures/appendix/qq_plots/CloudStore/Web_Server/First_six.pdf}}
	\caption{Q-Q plots for CloudStore's Web Server}
	%\captionsetup{justification=centering}
\end{figure}

\begin{figure}[tbh]
	\centering
	{\includegraphics[width=1.0\textwidth]{figures/appendix/qq_plots/CloudStore/Web_Server/Second_six.pdf}}
	\caption{Q-Q plots for CloudStore's Web Server}
	%\captionsetup{justification=centering}
\end{figure}

\begin{figure}[tbh]
	\centering
	{\includegraphics[width=1.0\textwidth]{figures/appendix/qq_plots/CloudStore/Web_Server/Third_six.pdf}}
	\caption{Q-Q plots for CloudStore's Web Server}
	%\captionsetup{justification=centering}
\end{figure}

\begin{figure}[tbh]
	\centering
	{\includegraphics[width=1.0\textwidth]{figures/appendix/qq_plots/CloudStore/Web_Server/Fourth_four.pdf}}
	\caption{Q-Q plots for CloudStore's Web Server}
	%\captionsetup{justification=centering}
\end{figure}
%%%%%%%%%%%%%%%%%%%%%%%%%%%%%%

\begin{figure}[tbh]
	\centering
	{\includegraphics[width=1.0\textwidth]{figures/appendix/qq_plots/CloudStore/DB_Server/First_six.pdf}}
	\caption{Q-Q plots for CloudStore's DB Server}
	%\captionsetup{justification=centering}
\end{figure}

\begin{figure}[tbh]
	\centering
	{\includegraphics[width=1.0\textwidth]{figures/appendix/qq_plots/CloudStore/DB_Server/Second_six.pdf}}
	\caption{Q-Q plots for CloudStore's DB Server}
	%\captionsetup{justification=centering}
\end{figure}

\begin{figure}[tbh]
	\centering
	{\includegraphics[width=1.0\textwidth]{figures/appendix/qq_plots/CloudStore/DB_Server/Third_six.pdf}}
	\caption{Q-Q plots for CloudStore's DB Server}
	%\captionsetup{justification=centering}
\end{figure}

\begin{figure}[tbh]
	\centering
	{\includegraphics[width=1.0\textwidth]{figures/appendix/qq_plots/CloudStore/DB_Server/Fourth_four.pdf}}
	\caption{Q-Q plots for CloudStore's DB Server}
	%\captionsetup{justification=centering}
\end{figure}



%%%%%%%%%%%%%%%%%%%%%%%%%%%%

%\begin{landscape}

\begin{figure}[tbh]
	\centering
	{\includegraphics[width=1.1\textwidth]{figures/appendix/ds2_heatmap_complete.pdf}}
	\caption{Heatmap (complete): DS2}
	%\captionsetup{justification=centering}
\end{figure}

%\end{landscape}

\begin{figure}[tbh]
	\centering
	{\includegraphics[width=0.8\textwidth]{figures/appendix/cloudstore_heatmap_complete.pdf}}
	\caption{Heatmap (complete): CloudStore}
	%\captionsetup{justification=centering}
\end{figure}



%\addcontentsline{toc}{chapter}{Bibliography}
\bibliography{bibliography.bib}  
\bibliographystyle{alpha}
\end{document}