
%\documentclass[10pt, conference, compsocconf]{IEEEtran}


%\hyphenation{op-tical net-works semi-conduc-tor}


%\begin{document}
%
% paper title
%
% can use linebreaks \\ within to get better formatting as desired
%\title{Studying the Impact of Logging Practices on Code Quality of Systems Software}


% author names and affiliations
% use a multiple column layout for up to two different
% affiliations

%\author{\IEEEauthorblockN{Weiyi Shang, Meiyappan Nagappan, Ahmed E. Hassan, Emad Shihab}
%\IEEEauthorblockA{Software Analysis and Intelligence Lab (SAIL)\\ Queen's University\\
%Kingston, Ontario, Canada\\
%\{swy, mei, ahmed, emad\}@cs.queensu.ca}
%}

% use for special paper notices
%\IEEEspecialpapernotice{(Invited Paper)}

% make the title area
%\maketitle




\RequirePackage{fix-cm}
%
%\documentclass{svjour3}                     % onecolumn (standard format)
%\documentclass[smallcondensed]{svjour3}     % onecolumn (ditto)
\documentclass[smallextended]{svjour3}       % onecolumn (second format)
%\documentclass[twocolumn]{svjour3}          % twocolumn
%
% ------------- Packages Used --------------------------------------------------
% They help us to produce a better looking document ;-)
% ------------------------------------------------------------------------------

% Comment the following line before compiling the final version
%\synctex

\usepackage{url}
\usepackage{listings}
\usepackage{color}
\usepackage{balance}
\usepackage{graphicx}
\usepackage[justification=centering]{caption}
\definecolor{dkgreen}{rgb}{0,0.6,0}
\definecolor{gray}{rgb}{0.5,0.5,0.5}
\definecolor{mauve}{rgb}{0.58,0,0.82}
\usepackage{multirow}
\usepackage{booktabs}
\usepackage{amsmath}
%\usepackage{mathtools}
\usepackage[flushleft]{threeparttable}

\lstset{frame=tb,
  language=Java,
  aboveskip=3mm,
  belowskip=3mm,
  showstringspaces=false,
  columns=flexible,
  basicstyle={\small\ttfamily},
  numbers=none,
  numberstyle=\tiny\color{gray},
  keywordstyle=\color{blue},
  commentstyle=\color{dkgreen},
  stringstyle=\color{mauve},
  breaklines=true,
  breakatwhitespace=true,
  tabsize=3
}

\usepackage{multirow}
\usepackage{balance}
\usepackage{graphicx}
\usepackage{alltt}
\usepackage{relsize}
%\usepackage{xspace}
\usepackage{booktabs}
\usepackage{array}
\usepackage{amsmath}
%\usepackage{multirow}
%\usepackage{array}
\usepackage{verbatim}

%\usepackage{subfigure}
\usepackage[tight,footnotesize]{subfigure}

%\usepackage{capt-of}
%\usepackage{pifont}
\usepackage{amsfonts}
\usepackage{amssymb}
%\usepackage[latin1]{inputenc}
%\usepackage{times}
%\usepackage{colortbl}
\usepackage{boxedminipage}
\usepackage{float}
\usepackage{cite}
\usepackage{fancyvrb}
\usepackage[dvipsnames]{xcolor}
%\usepackage{hyperref}
\usepackage{balance}
\usepackage{url}
\usepackage{fancybox}%for \hypobox
\usepackage{listings}
\usepackage{array}
\usepackage{multicol}
%\usepackage{chngpage}
%\usepackage{booktabs}

%\usepackage{textcomp}
%\usepackage{latexsym}
%\usepackage{amssymb}
%\usepackage{stmaryrd}
%\usepackage{euscript}
%\usepackage{wasysym}
%\usepackage{pifont}
%\usepackage{manfnt}
%\usepackage{undertilde}
%\usepackage{ifsym}
%\usepackage{tipa}
%\usepackage{txfonts}
%\usepackage{skak}
%\usepackage{skull}
%\usepackage{eurosym}
%\usepackage{yfonts}
%\usepackage{mathdots}
%\usepackage{trsym}
%\usepackage{upgreek}
%\usepackage{chemarr}
%\usepackage{accents}
%\usepackage{nicefrac}
%\usepackage{bm}


%\usepackage{pdfsync}

%   ACM Style
%\usepackage{lcsect}
% ------------------------------------------------------------------------------






% ------------ Color Definitions -----------------------------------------------
% Whatever colors we need
% ------------------------------------------------------------------------------
\definecolor{mygray}{rgb}{0.7,0.7,0.7}
% ------------------------------------------------------------------------------






% ---------- Special commands for annotating the paper's text ------------------
\let\mymarginpar\marginparm
\marginparwidth=1cm
\marginparsep=5pt
\newcommand{\todo}[1]{\textcolor{red}{\textbf{[[#1]]}}}
\def\TODO#1{\noindent\colorbox{yellow}{\bf \textcolor{red}{TODO: #1}}}
\newcommand{\hint}[1]{\textcolor{blue}{\textbf{#1}}}
\def\fig#1{Figure~\ref{#1}}
\def\tab#1{Table~\ref{#1}}
\def\eqn#1{Equation~\ref{#1}}
\def\sec#1{Section~\ref{#1}}

\pagenumbering{arabic}
\newcommand{\reviewer}[1]{\textcolor{DeepPink1}{{\it [Reviewer says: #1]}}}
\newcommand{\mei}[1]{\textcolor{red}{{\it [Mei says: #1]}}}
\newcommand{\hammam}[1]{\textcolor{red}{{\it [Hammam says: #1]}}}
\newcommand{\ian}[1]{\textcolor{blue}{{\it [Ian says: #1]}}}
\newcommand{\nic}[1]{\textcolor{WildStrawberry}{{\it [Nico says: #1]}}}
\newcommand{\ahmed}[1]{\textcolor{green}{{\it [Ahmed says: #1]}}}
\newcommand{\myfoot}[1]{\footnote{\scriptsize #1}}
\newcommand{\myurl}[1]{\myfoot{\url{#1}}}
\newcommand{\Ra}{{$\Rightarrow$}}
\newcommand{\ra}{{$\rightarrow$}}
\newcommand{\La}{{$\Leftarrow$}}
\newcommand{\la}{{$\leftarrow$}}
\newcommand{\lra}{{$\leftrightarrow$}}
\newcommand{\LRa}{{$\Leftrightarrow$}}
%\newcommand{\todo}{\bram{todo}}

\newenvironment{myindentpar}[1]%
{\begin{list}{}%
         {\setlength{\leftmargin}{#1}}%
         \item[]%
}
{\end{list}}

% \AtBeginDocument{%
%    \renewcommand{\figurename}{Figure}%
%    \newcommand{\subfigureautorefname}{\figureautorefname}%for using subfig
% %   \renewcommand{\tablename}{TABLE}%
%    \renewcommand{\tablename}{Table}%
%    \renewcommand{\subsectionautorefname}{Section}%
%    \renewcommand{\sectionautorefname}{Section}%
% }

% Hypothesis box
% ------------------------------------------------------------------------------
\newcommand{\hypobox}[1]{\begin{center}%
        \noindent\thicklines\setlength{\fboxsep}{6pt}%
        \cornersize{0}\Ovalbox{\begin{minipage}{3.3in}%
        \vspace{-0.15cm}
        \textit{#1}
        \vspace{-0.2cm}
        \end{minipage}} \end{center}}
% ------------------------------------------------------------------------------




% ------------------------- SYMBOLS OF SELF NAMES OFTEN USED -------------------
\newcommand{\APACHE}{{\small APACHE}\xspace}
\newcommand{\BUGZILLA}{{\small BUGZILLA}\xspace}
\newcommand{\ECLIPSE}{{\small ECLIPSE}\xspace}
\newcommand{\ASPECTJ}{{\small ASPECTJ}\xspace}
\newcommand{\JDT}{{\small JDT}\xspace}
\newcommand{\GNU}{{\small GNU}\xspace}
\newcommand{\MOZILLA}{{\small MOZILLA}\xspace}
\newcommand{\THUNDERBIRD}{{\small THUNDERBIRD}\xspace}
\newcommand{\JAVA}{{\small Java}\xspace}
\newcommand{\GNOME}{{\small GNOME}\xspace}
\newcommand{\PG}{{\small PostgreSQL}\xspace}
\newcommand{\SIM}{{\small SimScan}\xspace}

% Anything else, e.g., \NAME{MICROSOFT}
\newcommand{\NAME}[1]{{\small #1}\xspace}
% ------------------------------------------------------------------------------




% ----------------------- Computer Science lol ---------------------------------
% Variable, function, and program names
% ------------------------------------------------------------------------------
\newcommand{\smalltt}[1]{\ifmmode{\mbox{\smaller\texttt{#1}}}\else{\smaller\tt #1}\fi}
\newcommand{\code}[1]{\smalltt{#1}}
\newcommand{\var}[1]{\code{#1}}
\newcommand{\func}[1]{\code{#1}}
\newcommand{\proc}[1]{\code{#1}}
\newcommand{\prog}[1]{\code{#1}}
\newcommand{\type}[1]{\code{#1}}
\newcommand{\progpt}[1]{\code{#1}}

\newcommand{\mypar}[1]{\vspace{.1cm}\noindent \textbf{#1}}
\newcommand{\myxpar}[1]{\vspace{.1cm}\noindent \textbf{#1}\newline}
% ------------------------------------------------------------------------------





% ----------- Things to remember -----------------------------------------------
\newenvironment{mynote}%
{ \medskip
  \noindent
  \let\emph=\textbf
  \begin{boxedminipage}{\columnwidth}\em}%
{ \end{boxedminipage}}
% ------------------------------------------------------------------------------






% -------------------- Use bars ------------------------------------------------
% These macros are for advanced presentation of results by shaded bars only!
% © Tom Zimmermann, 2008
% ------------------------------------------------------------------------------
\newdimen\qdx
\newdimen\qda
\newdimen\qdb
\def\rrrr#1#2#3#4{\newdimen\qd\qd=#4 % length of bar for 1.0
\qdx=\qd\multiply\qdx by 5\divide\qdx by 4
\qda=\qd
\qdb=\qd
\multiply\qda by #1\divide\qda by #3\multiply\qdb by #2\divide\qdb by #3\advance\qdb by -\qda
    \leavevmode\hbox to \qdx{\hfil\vbox{%
    \hbox{\vrule\vbox{\hrule\hbox to 1\qd
            {\vrule depth0pt height0.7ex width \qda\color{mygray}%
 \vrule depth0pt height0.7ex width \qdb\hfill}\hrule}\vrule}
    }\hfil}}
\def\rrr#1#2#3{\rrrr{#1}{#2}{#3}{0.8cm}}
% -------------------------------------------------------------------------






% ------------ Graphics Hacks --------------------------------------------------
% Use these settings if Figures tend to get their one separate pages
% ------------------------------------------------------------------------------
% \renewcommand{\topfraction}{0.85}
% \renewcommand{\textfraction}{0.1}
% \renewcommand{\floatpagefraction}{0.75}
% ------------------------------------------------------------------------------





% ------------------ Biblio Hack -----------------------------------------------
% Use these settings if the References take too much space
% ------------------------------------------------------------------------------
\let\oldthebibliography=\thebibliography
  \let\endoldthebibliography=\endthebibliography
  \renewenvironment{thebibliography}[1]{%
       \vspace{-0.3cm}
    \begin{oldthebibliography}{#1}%
       \vspace{-0.2cm}
       \setlength{\parskip}{0ex}%
       \setlength{\itemsep}{0ex}%
%       \bibfont
  }%
  {%
    \end{oldthebibliography}%
  }
% ------------------------------------------------------------------------------





% -------------- Floats Redefined ----------------------------------------------
% Use these settings to change the whitespace between floats and text
% ------------------------------------------------------------------------------

% \setlength\dblfloatsep{1pt}
% \setlength\floatsep{1pt}
% \setlength\textfloatsep{5pt}
% \setlength\dbltextfloatsep{5pt}

% \renewcommand\floatpagefraction{.9}
% \renewcommand\topfraction{.9}
% \renewcommand\bottomfraction{.9}
% \renewcommand\textfraction{.1}
% \setcounter{totalnumber}{50}
% \setcounter{topnumber}{50}
% \setcounter{bottomnumber}{50}
% ------------------------------------------------------------------------------


\newcommand{\specialcell}[2][c]{%
        \begin{tabular}[#1]{@{}c@{}}#2\end{tabular}}


\usepackage{subfigure}
%\usepackage{subfig}

\smartqed  % flush right qed marks, e.g. at end of proof
%
\usepackage{graphicx}
\usepackage{balance}
\usepackage{amssymb,amsmath}
\usepackage{wrapfig}
\usepackage{multirow}
\usepackage{graphicx}
\usepackage{algorithm}
\usepackage{algorithmic}
\usepackage{times}
\usepackage{cite}
\usepackage{fancybox}
\usepackage{color}
\usepackage{array}
\usepackage{subfigure}
\usepackage{epstopdf}
\usepackage{booktabs}
\usepackage{caption,fixltx2e}
\usepackage[flushleft]{threeparttable}
%\usepackage{subfig}
%\usepackage{xspace}
\usepackage[hyphens]{url}
\usepackage[hidelinks]{hyperref}
\hypersetup{breaklinks=true}
\urlstyle{same}
\def\UrlBreaks{\do\/\do-} %to break the urls, all in one line
\newcommand{\ian}[1]{\textcolor{blue}{{\it [Ian: #1]}}}
\newcommand{\emad}[1]{\textcolor{green}{{\it [Emad: #1]}}}
\newcommand{\todo}[1]{\textcolor{red}{{\it [TODO: #1]}}}


\newcommand{\conclusionbox}[1]{%
	\vspace{2mm}
	\framebox[0.45\textwidth][c]{%
		\parbox[b]{0.42\textwidth}{%
			{\it #1}
		}
	}
	\vspace{2mm}
}

%
% \usepackage{mathptmx}      % use Times fonts if available on your TeX system
%
% insert here the call for the packages your document requires
%\usepackage{latexsym}
% etc.
%
% please place your own definitions here and don't use \def but
% \newcommand{}{}
%
% Insert the name of "your journal" with
% \journalname{myjournal}
%
\begin{document}

\title{Empirical Study on the Discrepancy between Performance Testing Results from Virtual and Physical Environments}

%\subtitle{Do you have a subtitle?\\ If so, write it here}

\titlerunning{Discrepancy between Performance Testing Results between diff. environments}        % if too long for running head

\author{Muhammad Moiz Arif \and Weiyi Shang \and Emad Shihab
}

%\authorrunning{Short form of author list} % if too long for running head

\institute{Muhammad Moiz Arif , Weiyi Shang, Emad Shihab\at
              Department of Computer Science and Software Engineering\\
		Concordia University\\
			Montreal, Quebec, Canada\\
              \email{\{mo\_ari, shang, eshihab\}@cse.concordia.ca}           %  \\
%             \emph{Present address:} of F. Author  %  if needed
%          \and
%           Meiyappan Nagappan \at
%		              Software Analysis and Intelligence Lab (SAIL)\\
%					Queen's University\\
%					Kingston, Ontario, Canada\\
%		              Tel.: +1 613-533-6802\\
%		              \email{mei@cs.queensu.ca}
%					   \and
%			 Ahmed E. Hassan\at
%	          	Software Analysis and Intelligence Lab (SAIL)\\
%					Queen's University\\
%					Kingston, Ontario, Canada\\
%		             Tel.: +1 613-533-6802\\
%		            \email{ahmed@cs.queensu.ca}
%		   \and
%	          Emad Shihab\at
%				Software Analysis and Intelligence Lab (SAIL)\\
%					Queen's University\\
%					Kingston, Ontario, Canada\\
%			Tel.: +1 613-533-6802\\
%		\email{emad@cs.queensu.ca}
}

\date{Received: date / Accepted: date}


\maketitle

\makeatletter
\def\ps@headings{%
\def\@oddfoot{\scriptsize \hfill \thepage }%
\def\@evenfoot{\scriptsize \thepage \hfill}}
\makeatother
\pagestyle{headings}

\begin{abstract}
Large software systems often undergo performance tests to ensure their capability to handle expected loads. These performance tests often consume large amounts of computing resources and time since heavy loads need to be generated. Making it worse, the ever evolving field requires frequent updates to the performance testing environment. In practice, virtual machines (VMs) are widely exploited to provide flexible and less costly environments for performance tests. However, the use of VMs may introduce extra overhead (e.g., a higher than expected memory utilization) to the testing environment and lead to unrealistic performance testing results. Yet, little or no research has studied the overhead and impact on test results of using VMs in performance testing activities. 

To evaluate the discrepancy between the performance testing results from virtual and physical environments, we perform a case study on two open source systems - namely Dell DVD Store and CloudStore. We conduct the same performance tests in both virtual and physical environments and compare the performance testing results based on the three aspects that are typically examined for performance testing results: 1) individual performance metrics (e.g. CPU Time vs CPU Time), 2) the relationship among performance metrics (one-to-many relationship) and 3) performance models that are built to predict future performance. Our results show that 1) individual performance metrics from virtual and physical environments do not follow the same distribution hence practitioners can not simply use a scaling factor to compare the performance between environments,  2) correlations among performance metrics in virtual environments are different from the correlations in physical environments which implies that the aforementioned change in correlations may not be just because of a performance discrepancy and 3) statistical models built based on the performance counters from virtual environments are different from the models built from physical environments suggesting that practitioners need to normalize the performance metrics. We also investigate ways to transform results from virtual and physical environments ang performance metrics based on deviance may reduced the discrepancy between performance metrics. Overall, we recommend that practitioners should not simply assume that performance testing results done on virtual environments will be the same in physical environments.

%be aware of such discrepancy and may leverage normalization techniques when analyzing performance testing results from the virtual environments.




%	activities, including performance modeling and performance regression detection. The two activities are conducted in both environments that consists of physical machines or VMs. We observed that the comparison between the virtual and physical environments is not straight forward. We also observed that the performance of our subject systems between two environments is not identical.

%	\textcolor{red}{(insert implications)}

%	\textit{Keywords--Performance engineering; Performance testing; Performance modeling; Performance analysis}	
\end{abstract}

%\begin{IEEEkeywords}
%Software maintenance, Mining software repositories, Software logs, Software quality
%\end{IEEEkeywords}

%\IEEEpeerreviewmaketitle
%\vspace{-0.2cm}


\section{Introduction}
%\vspace{-0.1cm}
\label{sec:intro}
%Cloud computing has eliminated the need for hardware decisions. It is now a significant part of the IT industry incorporating not only the leading names like \textit{Facebook, Amazon, Google} etc but also small-scaled business ventures that are migrating on to a cloud based environment. According to a survey \textit{'survey'}, up to 70\% of the companies are moving from conventional physical data servers? to cloud based servers. The primary concerns when migrating are, usually, costs, availability performance etc of the system. 

%Software Performance Engineering(SPE) incorporates all the software engineering activities carried out during the project's lifecycle required to meet the software's performance requirements.
%These performance requirements ensure that the user requests are served are in a timely manner and the software's response time does not degrade with the increase in workload i.e. users. In the midst of todays large-scale software systems, (e.g.Amazon and Google's Gmail), SPE plays a vital role as most of the failures are associated with performance than with feature bugs. A second's downtime of amazon.com would cost millions of dollars. One of the recent examples in this regard was the roll-out failure of healthcare.gov. Another example was the crash of Facebook which caused NASDAQ a high monetary misfortune. 

%To avoid performance failures, software performance engineers use performance testing. The tests are used to exercise the subject system which is under test. To exercise this system, it is subjected to different workloads. These tests are designed to uncover performance bottlenecks or a test objective like maximum operational capacity. 

%The goal of performance testing is to test how the system responds are realistic workloads. Therefore performance test are composted of test workload and configuration of SUT which represents a field like environment. 

%The goal of this study is to compare the performance of a SUT in heterogeneous environments. Due to lack of resources and constant requirement of evolution of the testing environment practitioners often rely on testing done in virtual environments. However, the road that leads to the reliability of performance activities done in virtual environments still remains undiscovered. 

%Performance is a big deal
Software performance assurance activities play a vital role in the development of large software systems. These activities ensure that the software meets the desired performance requirements~\cite{futureofspe}. All too often, failures in large software systems are due to performance issues rather than functional bugs~\cite{tailatscale, foo2010mining}. Such failures lead to eventual decline in the quality of the system with reputational issues and monetary losses~\cite{costofdowntime}. For instance, Amazon estimates that a one-second page-load slowdown can cost up to \$1.6 billion~\cite{amazononesec}. 

%The impact from performance issues on system reliability would also lead to serious reputational issues.

%an interruption in the Amazon Web Service lead to the disruption of Quora, Reddit, Foursquare and numerous other web sites~\cite{amazondown}. 

%In the software ecosystem, performance assurance activities play a vital role \cite{Shang:2015:ADP:2668930.2688052}. In essence, these activities ensure a consistent software functionality. The trend of dedicating a large chunk of costs, in some cases even exceeding the cost of development \cite{bertolino2007software} to such performance assurance activities is now not unusual. In fact, most of the problems in the field are due to performance related issues \cite{foo2010mining} . A failure here would not only include an eventual decline in the quality of the software but also monetary and temporal losses. That is why companies like \textit{Facebook, Amazon and Google} are committed to achieve excellence in this regard. \cite{jackson2010performance}

% Practitioners use performance tests
In order to mitigate performance issues and ensure software reliability, practitioners often conduct performance tests~\cite{futureofspe}. Performance tests apply a workload (e.g., mimicking users' behavior in the field) on the software system~\cite{ranjanbook,Syer2016}, and monitor performance metrics, such as CPU usage, that are generated from the tests. Practitioners use such metrics to gauge the performance of the software system and identify potential performance issues (such as memory leaks~\cite{markicsm2013} and throughput bottlenecks~\cite{5635038}).

 %Performance tests are subjected to highlight system's performance which are congruent to a field-like load~\cite{Shang:2015:ADP:2668930.2688052, Syer2016}. For example, to investigate the performance bottlenecks, the maximum throughput of the system~\cite{syer2014maintenance} or other the non-functional performance requirements.

%The objective behind performance regression testing is to identify if there exists a lapse in performance for the newer version of the software compared to the previous versions. The system is tested by applying a fixed load which is congruent to a field-like load \cite{Shang:2015:ADP:2668930.2688052} \cite{foo2010mining} \cite{5306331}. The performance analysts then look for deviations between metrics values compared to the earlier versions. Examples of factors causing performance lapse may be because of high CPU utilization or a memory leak. \cite{5306331}. As there are no benchmarks for measuring the software performance cross environments, and with little or no time dedicated to performance assurance activities practitioners often find it hard to test and analyze the results of regression testing.

%The need for performance testing environments to advance and evolve is continually augmenting and so is the cost associated with the environment~\cite{stpmag, bertolino2007software}. 

%One of challenges practitioners face is the lack of available resources for performance testing. For instance

%People perform things on virtual environments
Since performance tests are often performed on large-scale software systems, the performance tests often require many resources. Moreover, performance tests often need to run for a long period of time in order to build statistical confidence on the results~\cite{ranjanbook}. In addition, such testing environment needs to be easily configurable such that a specific environment can be mimicked, reducing false performance issues. For example, issues that are related to the environment. Therefore, to address such challenges, virtual environments (i.e., VMs) are often leveraged for performance testing~\cite{whyvirtualisbetter, vmwarehighcost, whyvirtualisbetter}. The flexibility of virtual environments enables practitioners to easily prepare, customize, use and update performance testing environments in an efficient manner.

%Making it worse, the diversified and ever-changing users' behaviour forces the testing environments to be frequently customized and updated~\cite{Syer2016}.

%More importantly, performance testing is often last stage of the software development lifecycle which forces the managers to dedicate a minimal time for performance testing which can even span out to days. That's why practitioners prefer testing the system in a virtual environment~\cite{whyvirtualisbetter, vmwarehighcost}. The choice of running the performance assurance activities in a virtual environment is also based on the complexity of the large scale software systems. This enforces a virtual set up of the environment which saves resources and is easier to set up according to the desired needs~\cite{VMWarePowerCLIBlog, seetharaman2006test}. 

% No one looked at the applicability across platforms
%However, a major question that lingers is that \textbf{are performance tests executed in a virtual environment representative of what happens in the physical environment?}. This question is particularly important since 1) virtual environments are highly leveraged in practice~\cite{Nguyen:2012:ADP:2188286.2188344,xiong2013vperfguard} and 2) prior work has shown that using virtual machines imposes a hidden overhead that is rarely considered~\cite{menon2005diagnosing}, impacting the reliability of performance test results performed in virtual environments. 

Prior studies show that virtual environments are widely exploited in practice~\cite{Cito:2015:MCA:2786805.2786826,Nguyen:2012:ADP:2188286.2188344,xiong2013vperfguard}. Studies have investigated the overhead that is associated with virtual environments~\cite{menon2005diagnosing}. Such overheads may not impose effect on the results of performance tests carried out in physical and virtual environments. For example, if the performance (e.g., throughput) of the system follows the same trend in physical and virtual environments, such overhead would not significantly impact on the practitioners who examine the perform testing results. To the best of our knowledge, the discrepancy between performance testing results in virtual and physical environments has never been studied. Exploring, identifying and minimizing such discrepancy would help practitioners and researchers understand and leverage performance testing results from virtual and physical environments.




%Whether virtual environments are applicable in performance assurance activities or if they can be relied to behave equivalent to the physical servers still remains questionable. There have been limited instances of diagnosis of performance overheads~\cite{menon2005diagnosing} in the domain of performance testing however no concrete conclusions have been drawn yet. 

%The goal of our study is evaluating the discrepancy between the performance testing results from virtual and physical environments. 
%Additionally, we investigated if the performane tests done in virtual environments are repeatable or not. 
%Additionally, if they do not belong to the same population, then how impactful and effective are the prevalent discrepancies for a model-based regression testing approach.    


%what we do
In this paper, we perform a study on two open-source systems, DS2~\cite{delldvd} and CloudStore~\cite{cloudstore}, where performance tests are conducted on virtual and physical environments. Our study focuses on the discrepancy between the two environments, the impact of discrepancy on performance testing results and highlights potential opportunities to minimize the discrepancy. In particular, we compare performance testing results from virtual and physical environments based on the three widely examined aspects, i.e., individual performance metric, the relationship between the performance metrics and models that predict performance. 
%by running performance tests in both virtual and physical environments. We compare the performance test results that are generated from both the environments. In particular, we compare the performance testing results by 1) examining individual performance metric, 2) examining relationship among performance metrics and 3) building statistical models using performance metrics. 

%what we find
We find that 1) performance metrics have different shapes of distributions and trends in virtual environments compared to physical environments, 2) there are large differences in correlations among performance metrics measured in virtual and physical environments, and 3) statistical models using performance metrics from virtual environments do not apply to physical environments (i.e., produce high prediction error) and vice versa. Then, we examined the feasibility of using normalizations to help alleviate the discrepancy between performance metrics. We find that in some cases, normalizing performance metrics based on deviance may reduce the prediction error when using performance metrics collected from one environment and applying it on another. Our findings show that practitioners cannot assume that their performance tests that are observed on one environment will necessarily apply to another environment. The overhead from virtual environment does not only impact the scale of the performance metrics, but also impacts the relationship among performance metrics. On the other hand, we find that practitioners who leverage both, virtual and physical environments, may be able to reduce the discrepancy that may arise due to the environment (i.e., virtual vs. physical) by applying normalization techniques.

%Our findings highlight the need of awareness of the discrepancy between performance testing results in virtual and physical environments, and the need to research efforts on investigating how to improve the use of both virtual and physical environments to ensure system reliability.

%We leverage a heatmap to visualize the changes in correlations among performance metrics and the system load metric.

% cannot apply on the performance metrics collected from another environments (with high prediction error), even after normalizing the performance metrics.


%In particular, we first compare the performance metrics values and distributions. We then investigate the correlation of performance counters to the generated load. Finally, we build statistical models to help us reach conclusions. 
%We observed that majority of the performance counters do not belong to the same family of distribution. We also observed that the metric that are highly correlated with the load are not the same across both our subject systems.
%We concluded that due to the discrepancies present in the virtual environment, we can not rely on performance evaluated in the virtual environment as is. We then apply scaling techniques to the metrics generated in the virtual environment to build linear regression models. We concluded that the performance metrics in the virtual environment are not identical copies of the metrics in the physical environment. 


% to determine and compare the distributions. This is achieved by comparing plots and finding correlation between the metrics cross-environments. We use generalized linear regression models, as used in the performance assurance activities \cite{Shang:2015:ADP:2668930.2688052}, to determine the extent of the comparison between metrics. If the metrics are transferable between the environments, we should expect to see a low percentage error. We chose to build our regression models based on multiple performance metrics to see the effect of the clustered performance metrics. 

%We diffuse our findings by answering the following research questions:

%\begin{description}
%	\item[$\bullet$] RQ1: Do performance metrics from physical and virtual environments belong to the same population?
%	\item[$\bullet$] RQ2: Is the metric correlation with load same cross-environments?
%	\item[$\bullet$] RQ3: Do performance metrics from different environments impact performance modeling?
	
%\end{description}

The rest of the paper is organized as follows. Section~\ref{sec:related} presents the background and related work to this paper. Section~\ref{sec:case} presents the case study step. Section~\ref{sec:results} presents the results of our cases study, followed by a discussion of our results in Section~\ref{sec:discussion}. Section~\ref{sec:threats} discusses the threats to validity of our findings. Finally, Section~\ref{sec:conclusion} concludes this paper.


\section{Background and Related Work}
%\vspace{-0.1cm}
\label{sec:related}
In this section, we discuss the related work of this paper in two main areas: 1) analyzing performance metrics from performance testing and 2) analysis of VM overhead. 


\subsection{Analyzing performance metrics from performance testing} 

Prior research has proposed a slew of techniques to analyze performance testing results, i.e. performance metrics. Such techniques typically examine three different aspects of the metrics: 1) individual performance metrics, 2) the relationship among performance metrics, and 3) statistical modeling based on performance metrics.


\subsubsection{Individual performance metrics}
\label{sec:relatedindividual}
Nguyen \textit{et al$.$}~\cite{Nguyen:2012:ADP:2188286.2188344} introduce the concept of using control charts~\cite{shewhart1931economic} in order to detect performance regressions. Control charts use a predefined threshold to detect performance anomalies. However control charts assume that the output follows a uni-model distribution, which may be an inappropriate assumption for performance. Nguyen \textit{ et al$.$} propose an approach to normalize performance metrics between heterogeneous environments in order to build robust control charts. %However, the experiments are only carried out on the virtual machines in contrast to our approach.

Malik \emph{et al$.$}~\cite{Malik:2010:ACL:1955601.1955936, haroon} propose approaches that cluster performance metrics using Principal Component Analysis (PCA). Each component generated by PCA is mapped to performance metrics by a weight value. The weight value measures how much a metric contributes to the component. For every performance metric, a comparison is performed on the weight value of each component to detect performance regressions.

Heger \emph{et al$.$}~\cite{DBLP:conf/wosp/HegerHF13} present an approach that uses software development history and unit tests to diagnose the root cause of performance regressions. In the first step of their approach, they leverage Analysis of Variance (ANOVA) to compare the response time of the system to detect performance regressions. Similarly, Jiang \emph{et al$.$}~\cite{jackicsm2009} extract response time from system logs. Instead of conducting statistical tests, Jiang \emph{et al$.$} visualize the trend of response time during performance tests, in order to identify performance issues.


%\emad{how does this compare to our work?}
%

%The ad hoc approach may fail to detect performance regressions if the target counters do not capture the performance regressions. Moreover, such ad hoc analysis does not detect the change of relationships between counters, such as the relationship between I/O and CPU.


\subsubsection{Relationship among performance metrics}
\label{sec:relatedrelation}

Malik \emph{et al$.$}~\cite{5635038} leverage Spearman's rank correlation to capture the relationship among performance metrics. The deviance of correlation is calculated in order to pinpoint which subsystem should take responsibility of the performance deviation.

Foo\emph{ et al$.$}~\cite{foo2010mining} propose an approach that leverages association rules in order to address the limitations of manually detecting performance regressions in large scale software systems. Association rules capture the historical relationship among performance metrics and generate rules based on the results of prior performance tests. Deviations in the association rules are considered signs of performance regressions.

Jiang \emph{et al$.$}~\cite{5270324} use normalized mutual information as a similarity measure to cluster correlated performance metrics. Since metrics in one cluster are highly correlated, the uncertainty among metrics in the cluster should be low. Jiang \emph{et al$.$} leverage entropy from information theory to monitor the uncertainty of each cluster. A significant change in the entropy is considered as a sign of a performance fault. %During the evaluation of the approach, the authors were able to detect 77\% of the injected faults and the faulty subsystems, without having any false positives.

%Our approach also considers the correlations between performance counters by using regression models. However, our approach first aims to group performance counters into test aspects and focuses on detecting performance regressions instead of faults.

%Malik\textit{ et al.}~\cite{haroon} derive performance signatures based on both supervised and unsupervised learning to detect performance anomalies. They use Principal Component Analysis as their unsupervised learning technique if the past tests are not marked pass or fail. PCA is used to to examine the relationships between different performance metrics.

\subsubsection{Statistical modeling based on performance metrics}
\label{sec:relatedmodel}

Xiong \textit{et al$.$}~\cite{xiong2013vperfguard} proposed a model-driven approach named \textit{vPerfGuard} to detect software performance regressions in a cloud-environment. The approach builds models between workload metrics and a performance metric, such as CPU. The models can be used to detect workload changes and assists in identifying performance bottlenecks. Since the usage of \emph{vPerfGuard} is typically in a virtual environment, our study may help the future evaluation of \textit{vPerfGuard}. Similarly, Shang\textit{ et al.}~\cite{Shang:2015:ADP:2668930.2688052} propose an approach of including only a limited number of performance metrics for building statistical models. The approach leverages an automatic clustering technique in order to find the number of models to be build for the performance testing results. By building statistical models for each cluster, their approach is applicable to detect injected performance regressions. %We use the same technique in our study to inject performance regression in the target system nonetheless the limitation of their study to perform their experiments in a virtual environment persists.


Cohen \textit{et al$.$}~\cite{cohen2004correlating} propose an approach that builds probabilistic models, such as Tree-Augmented Bayesian Networks, to examine the causes that target the changes in the system's response time. Cohen \textit{et al$.$}~\cite{Cohen:2005:CIC:1095810.1095821} also proposed that system faults can be detected by building statistical models based on performance metrics. The approaches of Cohen \textit{et al$.$}~\cite{cohen2004correlating, Cohen:2005:CIC:1095810.1095821} were improved by Bodik \textit{et al.}~\cite{bodik2008hilighter} by using logistic regression models.

Jiang \emph{et al$.$}~\cite{Jiang:2009:SMM:1555228.1555233} propose an approach that improves the Ordinary Least Squares regression models that are built from performance metrics and use the model to detect faults in a system. The authors conclude their approach is efficient than the current linear-model approach.

In our work, we compare performance testing results from both virtual and physical environments based on all the above three types of analyses. Our findings can help better evaluate and understand the findings from the aforementioned research.

%Model-based approached use the target counters(e.g. DISC I/O and CPU Utilization) to build models and these models are then used to detect performance regression int he system. What make the model-based approach celebrated is the inclusion and comparison of numerous performance counters at the same time. 
%Jiang \textit{et al. }\cite{jiang2011system} used an improved least square regression models to detect system faults.

\subsection{Analysis of VM overhead}

Kraft \textit{et al$.$}~\cite{kraft2011io} discuss the issues that are related to disk I/O in a virtual environment. They examine the performance degradation of disk request response time by recommending a trace-driven approach. Kraft \textit{et al.}~\cite{kraft2011io} emphasize on the latencies existing in virtual machine requests for disc IO due to increments in time associated with request queues. 

Aravind \textit{et al$.$}~\cite{menon2005diagnosing} audit the performance overhead in Xen virtual machines. They uncover the origins of overhead that might exist in the network I/O causing a peculiar system behavior. However, there study is limited to Xen virtual machine only while mainly focusing on network related performance overhead.

Brosig \textit{et al$.$}~\cite{brosig2013evaluating} predict the performance overhead of virtualized environments using Petri-nets in Xen server. The authors focused on the visualization overhead with respect to queuing networks only. The authors were able to accurately predict server utilization but had significant errors for multiple VMs.


Huber \textit{et al$.$}~\cite{huber2011evaluating} present a study on cloud-like environments. The authors compare the performance of virtual environments and study the degradation between the two environments. Huber \textit{et al$.$} further categories factors that influence the overhead and use regression based models to evaluate the overhead. However, the modeling only considers CPU and memory.

%\textcolor{red}{where does the following related work fits in?}

Luo \textit{et al$.$}~\cite{Luo:2016:MPR:2901739.2901765} converge the set of inputs that may cause software regression. They apply genetic algorithms to detect such combinations. 

Prior research focused on overhead of virtual environments without considering the impact of such overhead on performance testing and assurance activities. In this paper, we evaluate the discrepancy between virtual and physical environments by focusing on the impact of performance testing results and investigate whether such impact can be minimized in practice.

%Previous literature has its limitations as most of the performance regression testing and regression modelling  is performed in virtual environments. Through our work we validate the usage of virtual environments in the field of performance engineering.



\section{Case Study Setup}
%\vspace{-0.1cm}
\label{sec:case}
%\begin{figure}[t!]
%	\includegraphics[width=1\textwidth]{approach.pdf}
%	\centering
%	\caption{Approach overview}
%    \label{fig:approach}
%\end{figure} 
%%\includepdf][pages={1}][width=\columnwidth{approach.pdf}

\begin{figure}[thb]
	\includegraphics[width=.9\textwidth]{figures/overview}
	\caption{An overview of our case study setup.}
	%\captionsetup{justification=centering}
	\label{fig:Approach}
\end{figure}

In this section, we present our case study setup. The goal of our case study is to evaluate the discrepancy between performance testing results from virtual and physical environments. An overview of our case study setup is shown in Figure~\ref{fig:Approach}.


\subsection{Subject Systems}
Dell DVD Store (DS2)~\cite{delldvd} is an online multi-tier e-commerce web application that are widely used in performance testing and prior performance engineering research~\cite{Shang:2015:ADP:2668930.2688052,Nguyen:2012:ADP:2188286.2188344, jackicsm2009}. We deploy DS2 on an Apache (Version 3.0.0) web application server with MySQL 5.6 database server~\cite{mysql}. CloudStore~\cite{cloudstore}, our second subject system is an open source application based on the TPC-W benchmark~\cite{tpcw}. CloudStore is widely used to evaluate the performance of cloud computing infrastructure when hosting web-based software systems and is leveraged in prior research~\cite{tarekmsr16}. We deploy CloudStore on \textit{Apache Tomcat}~\cite{tomcat} (version 7.0.65) with MySQL 5.6 database server~\cite{mysql}. 


\subsection{Environmental Setup}

The performance tests of the two subject systems are conducted on three machines in a lab environment. Each machine has an Intel i5 4690 Haswell Quad-Core 3.50 GHz CPU, with 8 GB of memory, connected to a local gigabyte ethernet. The first machine hosts the web server and application server (Apache and Tomcat). The second machine hosts the MySQL 5.6 database server. The load drivers (e.g., JMeter) was deployed on the third machine. We separate the load driver, the web/application server and the database server on different machines in order to avoid interference among these processes. The operating systems on the three machines are Windows 7. We disable all other processes and unrelated system services to minimize their performance impact. Since our goal is to compare performance metrics in virtual and physical environments, we setup the two different environments, which we detail next.

%was dedicated to the database server, the second machine was dedicated to the web server and the third machine was used to run the load driver.
\noindent \textbf{Virtual environment.} We install one Virtual Box (version 5.0.16) and create only one virtual machine on one physical machine to avoid the interference between virtual machines. For each virtual machine, we allocate two cores and 3GB of memory. 

%Our virtual setup was identical to our physical setup. The virtual environment was run on the same physical machines with all the resources provided to the host and the same set of aforementioned configuration for the virtual environment. We opted for single tenancy of the guest operating system to avoid any unwanted noise. 

\noindent \textbf{Physical environment.} To make the physical environment similar to the virtual environment, we only enable two cores and 3GB memory for each machine for the physical environment. 

%To make the systems' configuration identical prior to exercising the subject systems, we chose 2 cores and 3GB of memory dedicated to each environment to avoid crashes on the guest operating system in the virtual environment. We also made sure to kill all the processes before we start our performance testing to minimize any discrepancy present.

%\subsection{Exercising the database}
%Following the set up of our subject systems on the respective servers, the systems were exercised with an aid of drivers. These drivers generated multi-type web requests and simulated real-time user behavior depending on the input parameters provided. We ran our performance tests for numerous hours while recording all the performance metrics generated for varying load applied on our software systems.
\subsection{Performance tests}

DS2 is released with a dedicated load driver program that is designed to exercise DS2 for performance testing. We used the load driver to conduct performance testing on DS2. We used Apache JMeter~\cite{apachejmeter} to generate a workload for conducting performance tests on CloudStore. For both subject systems, the workload of the performance tests is variated periodically in order to avoid bias from a consistent workload. The workload variation was introduced by the number of threads. A higher number of threads represented a higher number of users accessing the system. Each performance test is run after a 15 minutes warming up period of the system and lasts for 9 hours. 


%and the last 30 minutes in order to 

 %The choice of load was random but consistent between both of the environments. As our study was based on exercising our systems and recording the performance metrics, and not stress testing \cite{stresstesting}, the respected limits were chosen in order to avoid the under-performance of the physical machine and system failure of the virtual machine.


%\subsection{Metrics Collection}
\subsection{Data collection and preprocessing}

\noindent \textbf{Performance metrics.} We used \textit{PerfMon}~\cite{perfmon} to record the values of performance metrics. \textit{PerfMon} is a performance monitoring tool used to observe and record performance metrics such as CPU utilization, memory usage and disk IOs. We record all the available performance metrics that can be monitored on a single process by \emph{PerfMon}.  We recorded the performance metrics of both the processes, web server and the database server, with an interval of 10 seconds. In total, we recorded 44 performance metrics. 

\noindent \textbf{System throughput.} We used the web server access logs from Apache and Tomcat to calculate the throughput of the system by measuring the number of requests per minute. The two data sets were then concatenated and mapped against requests using their respective timestamps.

In order to combine the two datasets of performance metrics and system throughput, and to minimize noise of performance metrics recording, we calculate the mean values of the performance metrics in every minute and combine the datasets of performance metrics and system throughput based on the time stamp on a per minute basis. A similar approach has been applied to address mining performance metrics challenges~\cite{foo2010mining}.



\section{Case Study Results}
%\vspace{-0.1cm}

\label{sec:results}
The objective of our case study is a comparative analysis of the two environments and to find out if there exists software performance regression between a dedicated server and the virtual environment. To analyze this, we divided our project into three research questions. Following the set up of the subject systems and data preprocessing, we used requests per minute as our independent variable and performance metrics as our dependent variables to answer our research questions. %Firstly, we observe if our models are transferable across the three chosen environments. Secondly, if the models can accurately predict counter values cross-servers. Lastly, if defect injection will generate similar performance metrics values compared   

\subsection{\textbf{Do performance metrics from physical and virtual environments belong to the same population?}}

\textbf{Motivation}: Performance assurance activities are not only bounded by the idea that performance regression can only be identified relative to different versions of the software. A software system might also regress once it is transferred from one environment to another. According to the norm, practitioners often rely on performance metrics generated by the virtual environment prior to releasing it on a disparate environment. We address this question by comparing the metric values between the physical and virtual server. 

\textbf{Approach}: As mentioned earlier in Section 3, after setting up the subject systems, we used the number of requests as our workload for the systems. A script was written to randomly alter the workload after every minute for CloudStore and every two minutes for DS2. The script was based on an underlying assumption that every unit of change in the performance metrics there will be a consequent unit of change in the workload which eventually introduced variety in our dataset. \cite{linearregression}. Subsequently, the requests per minute were mapped against the performance metrics recorded via \textit{perfmon}.

The numerical values for requests per minute were extracted out of the web server's access logs. These requests were recorded per second and reached up to thousands in a minute, depending on the workload. We then, with the aid of a script, grouped and extracted the requests per minute for as long as the test systems were exercised. The metrics, on the contrary, were recorded every 10 seconds. A set of six records was averaged out as a minute according to the timestamps. This was called a single \textit{run}. The metrics generated for the web and database server were concatenated. Ultimately, we had a dataset consisting of 500 runs where every record represented a \textit{run}; the metrics and requests generated for a given minute. This process was carried out for both of our environments. The two datasets, to analyze any discrepancies, were then analyzed using R's q-q plots and \textit{cor} function.

\begin{figure}[thb]
	\centering
	\includegraphics[width=0.9\columnwidth]{figures/ds2_qq.pdf}
	\caption{Q-Q plots: DS2}
%	\captionsetup{justification=centering}
	\label{fig:Results Table}
\end{figure}



\begin{figure}[thb]
	\centering
	\includegraphics[width=0.9\columnwidth]{figures/cloudstore_qq.pdf}
	\caption{Q-Q plots: CloudStore}
%	\captionsetup{justification=centering}
	\label{fig:Results Table}
\end{figure}



\begin{table}[t]
	\centering
			\caption{DS2: Correlation Values}
			\label{resultRQ3}
	\begin{tabular}{c|cc}
				\textbf{Performance Metrics}   & \textbf{Cor} & \textbf{p-value}\\  
			 
				\textbf{Web Servers' User Times} &  0.08 & 0.07\\
				\textbf{DB Servers' User Times} & -0.05 & 0.30\\
				\textbf{Web Servers' IO Data Ops/Sec}   & 0.25 & 0.000 \\
				\textbf{DB Servers' IO Data Ops/Sec} & -0.14 & 0.00\\
				\textbf{Web Servers' Memory Working Set} & 0.22 & 0.00\\
				\textbf{DB Servers' Memory Working Set} & 0.46 & 0.00\\
		\end{tabular}
\end{table}
	
\begin{table}[t]
		\caption{CloudStore: Correlation Values}
		\label{resultRQ3}
		\begin{tabular}{c|cc}
			\toprule
			\textbf{Performance Metrics}   & \textbf{Cor}& \textbf{p-value}\\
			\midrule 
			\midrule 
			\textbf{Web Servers' User Times} & \ 0.01& 0.87\\
			\textbf{DB Servers' User Times} & \ 0.20 & 0.00\\
			\midrule 
			\textbf{Web Servers' IO Data Ops/Sec}   & \ 0.17& 0.00 \\
			\textbf{DB Servers' IO Data Ops/Sec} & \ 0.18& 0.00\\
			\midrule 
			\textbf{Web Servers' Memory Working Set} &\ 0.69& 0.00\\
			\textbf{DB Servers' Memory Working Set} & -0.13 & 0.00\\
			\bottomrule             
		\end{tabular}
\end{table}

%\begin{figure}[t!]
%	\centering
%	\includegraphics[width=0.7\textwidth]{private_bytes.pdf}
%	\caption{QQ-plot of Private Bytes of Physical vs. Virtual Environment}
%	\captionsetup{justification=centering}
%	\label{fig:Results Table}
%\end{figure}

%\textbf{Approach}: 
%Following our data preprocessing, we decided to use Q-Q plot to identify if there exists a similarity between the distribution of our counters for the dedicated and the virtual server. We divided our metrics into three sub-categories namely \textit{processor, IO and memory}.

\textbf{Results}: Figure 2 and 3 show the results from our q-q plots. For the sake of brevity, we chose one q-q plot to be displayed from each subset of metrics i.e. CPU, IO and memory. If the distributions of our performance metrics were similar, we should see the plots closer to the line y=x. For the sake of reference, we named this line 'Z'. \cite{Cross_Validation}. We plotted the line Z on the same axes. 

As seen in Figure 2 for the subject system DS2, the CPU user times for the web servers are closer to the line Z however the plots database servers' CPU user times are highly deviated from the same line as it not visible in the same plot. The disk IO operations/sec for the web servers gradually deviate form the line Z whereas the database server is, again, highly deviated. The same can be concluded about the memory working sets for both of our environments.
Figure 3 shows the q-q plots for CloudStore. The web servers CPU user times are not congruent with the line Z. The database server CPU user times towards the tail of the plot are closer to the line Z however they still do not follow the line Z. The disk IO operations for both servers tends to follow the line Z initally but gradually moves away. Further on, the memory working set for both of our servers, as seen, are distant to the line Z.

Our Mann-Whitney \textit{U} Tests also concluded that for each of the performance metrics selected they do not belong to the same distributions with a p-value $<$ 0.05, except the web server's user times for DS2.

Table 1 and 2 shows the Spearman correlation values between the selected performance metrics. If the correlation value is closer to 1 the metrics have a strong correlation and if the value is closer to 0 the metrics have a weak correlation. A negative correlation means that if one of the metrics is increasing in value, the other is decreasing. For both of our subject systems we observe that the correlation values are mostly positive and are closer to 0 representing weak correlations. This conclusion, however, is not applicable to the user times from both environments as the p-value $>$ 0.05.

\subsection{\textbf{Is the correlation value between load and performance metrics same cross-environments?}}

\textbf{Motivation}: Building on the conclusion from the previous research question, we next address the change in metric correlation values amongst themselves and versus the load. Our goal was to explore whether the change in correlation cross-environments is identical for both of our subject systems. This way we will be able to conclude whether a certain type of discrepancy is always present between the two environments or is it the unstable nature of the performance of subject systems in different environments. 
%Before the software is deployed in a physical environment, the performance analyst relies on the heuristics generated from the virtual environment \cite{foo2010mining}. One of the approaches to detect performance regression is to compare every metric with the previously passed performance test \cite{Shang:2015:ADP:2668930.2688052}. As discussed in our previous research question, most of the performance metrics between two environments do not belong to the similar distribution. 

%As discussed in RQ1, we discovered that our comparison of performance metrics between our physical and virtual environments produced dissimilar results. Our next step was to explore what metrics changed the most 


%As performance testing spans our from handful of hours to several days , the reliability on such exercise is remarkably critical. A recent bug fix or a modification may require a reiteration of performance tests \cite{foo2010mining}. On the contrary, This led us to our next question, that whether the performance assurance activities run on virtual environments can be duplicated.

\textbf{Approach}: We looked for the top 5 metrics which are highly correlated with the load in each of our environments. We used R's \textit{cor} function to determine the Spearman value of the aforementioned associations. 
Next, we used heat maps to highlight the set of metrics which show a significant change in correlation values amongst themselves.
The correlation values between the metrics of physical server were stored in matrix A. The correlation values between the metrics of the virtual server were store in matrix B. Matrix Z was the absolute difference between these two matrices. For example the Spearman correlation value for CloudStore's physical server between web server's User Time and database server's IO write/Bytes sec is 0.94. The correlation value for the same pair of metrics in the virtual environment is 0.26. Then the Matrix Z will record a value which is the absolute difference of the aforementioned Spearman values i.e. 0.68. If the metric value has changed significantly, according to the legend, this will be denoted as a 'hot zone' in the heatmap denoted by a lighter gradient of color.

\textbf{Results}: Figure 4 and 5 are the heatmaps for the change in correlation values amongst the metrics of two environments. For DS2, figure 4, the hot spots are mostly prevalent between the IO operations, for both the web and database server, cross-environments. This means the correlations amongst IO operations in the physical environment are not the same as the correlation between IO operations in the virtual environment. While CloudStore's heatmap, figure 5, shows that the change in correlation values is not as similar to DS2. Most of the hot spots are scattered across the heatmaps, contrary to DS2 where we can see clusters around most of the IO operations. 

We also observed a similar change in correlation values between the processor times and other metrics. This trend may not be as strong as CloudStore's heatmap for DS2, however these changes in values can be found in both of our subject system.

Table 3-6 are the top five highly correlated metrics with the load. In DS2, most of the IO operations from the web driver are highly correlated with the load in the physical environment. However, the database server is highly correlated than any of the metrics from the web driver in the virtual environment.
Table 5 and 6 shows a much similar behavior of CloudStore in both the environments.

We primarily learned that the DS2 IO operations' behavior in one environment are not similar to that of the physical environment. We also learned that the change in nature of correlation cross-environments is non-uniform.
%Figure 4 and 5 are the heatmaps for the change in correlations' values between the metrics of two environments. For DS2, figure 4, 4he hot spots are mostly prevalent between the IO operations cross-environment while CloudStore's heatmap, figure 5, shows that the change in correlation values is not as similar to DS2. Most of the hot spots are scattered across the heatmaps, contrary to DS2 where we can see clusters around most of the IO operations.



\begin{figure}[tbh]
	\centering	
	\includegraphics[width=0.5\textwidth]{figures/ds2_heatmap.pdf}
	\caption{Heatmap: DS2}
%	\captionsetup{justification=centering}
	\label{fig:Results Table}
\end{figure}

\begin{figure}[tbh]
	\centering
	{\includegraphics[width=0.5\textwidth]{figures/cloudscale_heatmap.pdf}}
	\caption{Heatmap: CloudStore}
	%\captionsetup{justification=centering}
	\label{fig:Results Table}
\end{figure}


\begin{table}[tbh]
		\centering
		\caption{DS2: Top 5 highly correlated metrics with load (Physical Server)}
		\label{resultRQ3}
		\begin{tabular}{c}
			\toprule
			1. Web Server IO Other Operations/sec \\
			2. Web Server IO Other Bytes/sec \\
			3. Web Server IO Write Operations/sec \\
			4. Web Server IO Data Operations/sec \\
			5. Web Server IO Data Bytes/sec \\
			
			\bottomrule             
		\end{tabular}
\end{table}

\begin{table}[tbh]
	\centering
		\caption{DS2: Top 5 highly correlated metrics with load (Virtual Server)}
		\label{resultRQ3}
		\begin{tabular}{c}
			\toprule
			1. Database Server Handle Count \\
			2. Database Server Working Set-Peak \\
			3. Database Server Pool Paged Bytes \\
			4. Database Server IO Other Operations/sec \\
			5. Database Server Page File Bytes Peak \\
			\bottomrule             
		\end{tabular}
\end{table}

\begin{table}[tbh]
	\centering
		\caption{CloudStore: Top 5 highly correlated metrics with load (Physical Server)}
		\label{resultRQ3}
		\begin{tabular}{c}
			\toprule
			1. Database Server IO Other Bytes/sec  \\
			2. Database Server IO Read Operations/sec   \\
			3. Database Server IO Read Bytes/sec \\
			4. Database Server IO Data Operations/sec    \\
			5. Database Server IO Write Operations/sec \\
			\bottomrule             
		\end{tabular}
\end{table}


\begin{table}[tbh]
	\centering
		\caption{CloudStore: Top 5 highly correlated metrics with load (Virtual Server)}
		\label{resultRQ3}
		\begin{tabular}{c}
			\toprule
			1. Database Server IO Other Operations/sec  \\
			2. Database Server IO Write Operations/sec   \\
			3. Database Server IO Write Bytes/sec \\
			4. Database Server IO Data Bytes/sec    \\
			5. Database Server IO Read Bytes/sec \\
			\bottomrule             
		\end{tabular}
\end{table}


\subsection{\textbf{Do performance metrics from different environments impact performance modeling?}}

\textbf{Motivation}: As discussed in earlier work \cite{Shang:2015:ADP:2668930.2688052} \cite{Nguyen:2012:ADP:2188286.2188344}, performance tests require a large dedication of resources as it is carried out just before the system is on the brink of deployment. This gives insufficient time to the performance engineers, leaving them with minimal resources. As a result they leverage on the performance assurance activities carried out in the virtual environment. The motivation behind this research question is to investigate whether the performance models generated from one environment can be applied and held representatives of the other. In essence, this will help the practitioners conclude the reliability of the performance activities carried out in the virtual environment. 
% In practice, this may or may not be appropriate to assume. This also spawns the concept of including the entire set of performance metrics for analysis, which is slipshod and ineffectual. 

\textbf{Approach}: We first validated our models with an environment using 10-fold cross validation. \textit{K}-fold cross validation divides the data into \textit{k} parts which are known as folds. The model is trained on \textit{k}-1 folds and tested on the \textit{k}$^{th}$ fold. This process is iteratively repeated \textit{k} times \cite{10foldcross} \cite{kohavi1995study}.


We partitioned our results into two segments; the explanatory and the predictive part for our models, trailed by applying our model to predict load cross-environments. The explanatory part calculates the percentage of deviance explained by our models i.e. the fit of the model while the predictive part explains the error percentage between the actual and predicted load values. Both of these parts were addressed by building generalized linear models that only incorporated the metrics which were selected via R \textit{stepwise stepwise} or commonly knows as \textit{stepwise} function, from the complete set of metrics from the dataset.

\textit{Explanatory Power}

After removing any outliers for both of our subject systems, we built our GLM, initially, using all the physical server's metrics. We trained and tested our model on the same server i.e. physical. We then reduced our model, iteratively, using only the metrics that our contributing the most to the model. This was achieved using R's \textit{stepwise} function. The \textit{stepwise} function adds the independent variables one by one to the model to exclude any metrics that not contributing to the model \cite{RInAction}. Once the metrics were automatically selected out on the physical server, we used R's \textit{ANOVA}, or \textit{analysis of deviance} to rank the metrics according to their deviance value. Higher the deviance value for a given performance metric in the model, higher the contribution of the performance metric to the model. This approach was similarly applied to the set of metrics from the virtual server to build a generalized linear model. R's \textit{deviancepercentage} was used to determine the explanatory part or fit of our model. It is used to calculate the percentage of deviance for a given GLM model.

As a result, we had two scenarios per subject system:
\begin{description}
	\item[$\bullet$] Trained on physical, tested on physical.
	\item[$\bullet$] Trained on virtual, tested on virtual.
\end{description} 

\textit{Predictive Power}
%\begin{description}
	
	%\paragraph{Cross-Environment}
	
	The notion behind our predictive approach was to observe the percentage error between the actual and the predicted load. Our first set of predicted values were based on the model trained on the virtual environment's performance metrics and tested on the physical's server metrics. For the dataset of the virtual metrics, we wrote a script in R to remove the metrics that indicated practically zero fluctuation because else they will not have any impact on the GLM. This was trailed by the step to remove any highly correlated metrics and using \textit{stepwise} for every fold \cite{Shihab:2010:UIC:1852786.1852792}. 
	%From the statistical techniques available to validate our model, we used 10-fold cross \cite{kohavi1995study} \cite{10foldcross}.
	We used mean absolute percentage error (\textit{MAPE}) to measure the error between the predicted and actual load values \cite{mape}. MAPE serves as the percentage measurement of the deviance of our forecasted values from our real values which makes it easier to interpret our results. If the error is 3, we say the forecasted value are off by 3\%. 
	
	Based on our results, we then also scaled the virtual environment's metrics according to the physical environment. Our approach for scaling was based on by Nguyen \textit{et al.} previous work \cite{Nguyen:2012:ADP:2188286.2188344}:
	%We addressed the presence of a high percentage error by adjusting our virtual environment's metrics to the phsyiscal environment according to the following equation:
	%As explained in RQ1, we perceived that our distributions generated by the performance metrics in our environments are not the same. We used R's \textit{predict} to predict our desired set of load, based on the training and testing mentioned in the previous subsections. \textcolor{red}{we used 1-fold cross validation here, necesaary to mention?}
	
	\begin{equation*}
	Load_{physical}= \alpha_{physical} \times Counter_{physical} + \beta_{physical}
	\end{equation*}
	
	\begin{equation*}
	Load_{virtual}= \alpha_{virtual} \times Counter_{virtual} + \beta_{virtual}
	\end{equation*}
	
	\begin{equation}
	V_{scaledmetric} = (\frac{Load_{virtual}-\beta_{virtual}}{\alpha_{virtual}})\times\alpha_{physical}+\beta_{physical}
	\end{equation}
	
To scale our metrics from the virtual environment, we first build a GLM with the selected counter and the load in the each of the environment. The selected counter was chosen on the basis of counters selected by \textit{stepwise}, trained in virtual and tested in phyiscal environment. For every GLM there exists an intercept and a gradient value. We used the aforementioned values and applied them to the metric from the virtual environment as explained by equation 1.
We also assumed that for each of the metric there will be a gradient and intercept value.
 %The selection of metrics was based on their presence in the GLMs that were trained in the virtual environment and tested in the physical environment.


 
\textbf{Results}: Tables 7-10 show the results of our approach. In Table 7, for our subject system DS2, we see that the statistically significant metrics for the GLM from both of our environments are different. The significant of the performance metrics in an GLM is environment dependent. Due to automatic selection of metrics we see a high fit and lower \textit{MAPE} values for our environments. Table 8 shows us the results when the model from the virtual was applied to predict load for the physical server and vice-versa. We observe a an immensely high \textit{MAPE} value for the former while physical-virtual prediction is almost 50\%. This means that the predicted and actual values of the workload have a mean absolute error percentage of almost 50\%. When the same set of metrics from the virtual environment were scaled the MAPE value for DS2 reduces drastically.


Table 9 shows us the results from CloudStore. Again, the ranks of the metrics that prove to contribute most to the model are not the same compared to both environments. After the data validation, we see a \textit{MAPE} value of almost 16\% for physical server and almost 5\% for virtual. As the GLM was based on automatic selection via \textit{stepwise} regression, we see a high percentage fit for our models. 
Table 10 shows us the results of cross application of models. A model that was trained on virtual server and tested to predict the workload values for the physical server was off by almost 29\% whereas from physical applied to virtual it was off by 293\%. Similarly, when the metrics were scaled, instead of a an expected decrease in the MAPE value we see that the MAPE jumps from 28.95\% to 276.04\%.

Not only the \textit{MAPE} values are high when applying models cross environments, they are inconsistent between two environments. We conclude that for both of our subject systems, the models can not directly be applied to predict load for the other environment. We also observed that scaling according to the physical environment may or may not work. This is dependent on the selection of metrics in both of the environments. What might be significant in on of the models may not be applicable to the other model in another environment.




%We obtained the former equations with the models by training and testing on the respective environments. These equations are derived from the previous work of \cite{Nguyen:2012:ADP:2188286.2188344} Once, the model was constructed we extracted the \(\alpha\) and \(\beta\) values from each model and applied them to scale the virtual environment's metrics accordingly. Equation (1) shows the derived equation through which we scaled the metrics during the process of scaling the counters. 


	%As shown in Table 3, we concluded that most of the contributions to our models are by \textit{DISC IO} and the \textit{CPU}. Inclusion of the aforementioned set of performance metrics gave us a fit percentage closer to the models which included all of the performance metrics. Additionally, the models that were based on the statistically significant metrics ranked by the other environment showed a MAPE value of no more than 7\%. This helped us conclude the metrics that are contributing the most between are environments are overlapping.  
	%The unscaled transfer of metrics generated a \textit{MAPE} value \textbf{40\%} more than that of scaled counters. Our approach successfully identified that the performance metrics generated in the virtual environment can not be taken as the exact reflection of the performance of the system in the physical environment. We also deduced the diversity in environment does impact the regression models and the performance assurance activities. One of the solution to counters this is scaling, however, the error still remains relatively high as shown in Table 2.
	
	%\afterpage{%
	%    \clearpage% Flush earlier floats (otherwise order might not be correct)
	%    \thispagestyle{empty}% empty page style (?)
	%    \begin{landscape}% Landscape page
	%        \centering % Center table
	%        \begin{tabular}{llll}
	%          \begin{table}[thb!]
	%    \begin{center}
	%    \caption{Results}
	%    \label{tab:project_results}
	%            \begin{tabular}{c||cccc}
	%            \toprule
	%            \textbf{Training-Testing}   & \textbf{Physical - Physical} & \textbf{Physical - Virtual} & \textbf{Virtual - Virtual} & \textbf{Virtual - Physical} \\  
	%            \midrule 
	%            \textbf{Ranking}     &       1. IO Read Bytes/sec & 1.IO Read Bytes/sec & 1.User Time  & 1.IO Read Operations/sec \\
	%             &                               2. IO Data Operations/sec & 2. IO Data Operations/sec & 2. IO Read Bytes/sec & 2. IO Read Operations/sec \\
	%             & 3.IO Read Operations/sec & 3.IO Other Bytes/sec & 3. IO Data Operations/sec & 3. IO Other Bytes/sec \\
	%             & 4. IO Other Bytes/sec & 4. IO Read Operations/sec & 4. IO Other Bytes/sec & 4. IO Write Bytes/sec\\
	%             & 5.IO Data Bytes/sec & & 5.Elapsed Time &\\
	%             & 6.Page Faults/sec & & 6.IO Read Operations/sec &\\
	%             & & & 7. Thread Count &\\
	%             & & & 8. IO Write Bytes/sec &\\
	%             & & & 9.IO Write Operations/sec &\\
	%             & & & 10. Working Set &\\
	%             \midrule
	%             \textbf{Fit} & All metrics: 64.1\% & All metrics: 81\% & All metrics: 81\% & All metrics: 63.4\%\\
	%             & Statistically Significant Metrics: 63.5\% & Statistically Significant Metrics: 81\% & Statistically Significant Metrics: 81\% & Statistically Significant Metrics: 63.2\% \\
	%             \midrule
	%             \textbf{MAPE} & 6.17\% & 3.81\% & 3.89\% & 6.14\% \\
	%            \bottomrule             
	%        \end{tabular}
	%    \end{center}
	%\end{table}
	%\\
	%        \end{tabular}
	%        \captionof{table}{Table caption}% Add 'table' caption
	%    \end{landscape}
	%    \clearpage% Flush page
	%}
	
%\end{description}
	
%\begin{landscape}
	\begin{table}[tbh]
		\centering
			\caption{DS2: Ranking of Performance Counters and Prediction errors}
			\label{tab:resultRQ1}
			\resizebox{\columnwidth}{!}{%
			\begin{tabular}{c||cc}
				\toprule
				\textbf{Training-Testing} & \textbf{Physical - Physical} & \textbf{Virtual - Virtual} \\  
				\midrule 
				\textbf{Ranking}     & 1. Web Server Privileged Time & 1. Web Server IO Other Bytes/sec \\
				\\
				& 2. Database Server User Time & 2. Web Server Page Faults/sec \\
				\\
				& 3. Database Server IO Read Bytes/sec & 3.Database Server Working Set-Peak\\
				\\
				& 4. Web Server Page Faults/sec & 4. Web Server Handle Count \\
				\\
				& 5. Database Server Privileged Time & 5. Database Server IO Data Operations Bytes/sec \\
				\\
				& 6. Database Server IO Write Operations Bytes/sec &\\
				\\
				& 7. Database Server Pool Paged Bytes &\\
				\\
				& 7. Web Server Working Set-Private &\\
				\\
				\midrule
				\textbf{Fit} &  85.80\% &  67.10\% \\
				\midrule
				\textbf{MAPE} & 7.02\% & 10.49\% \\
				\bottomrule             
		\end{tabular}%
	}
	\end{table}
%\end{landscape}


	\begin{table}[tbh]
		\centering
			\caption{DS2: Prediction errors cross-environments}
			\label{resultRQ3}
			\begin{tabular}{c|c}
				\toprule
				\textbf{Training - Testing}   & \textbf{MAPE}\\  
				\midrule 
				Physical-Virtual       & 49.97\% \\
				Virtual - Physical        & 6033.16\% \\
				Virtual - Physical \textbf{(after scaling)}      & 16.04\% \\
				\bottomrule             
			\end{tabular}
	\end{table}
	
	
	\begin{table}[tbh]
		\centering
			\caption{CloudStore: Ranking of Performance Counters and Prediction errors}
			\label{tab:resultRQ1}
			\resizebox{\columnwidth}{!}{%
				\begin{tabular}{c||cc}
					\toprule
					\textbf{Training-Testing} & \textbf{Physical - Physical} & \textbf{Virtual - Virtual} \\  
					\midrule 
					\textbf{Ranking}     & 1. Web Server Privileged Time & 1. Web Server IO Data Bytes/sec \\
					\\
					& 2. Database Server Privileged Time & 2. Database Server User Time \\
					\\
					& 3. Web Server Page Faults/sec & 3.Database Server IO Other Bytes/sec\\
					\\
					& 4. Web Server Virtual Bytes & 4.Database Server Page Faults/sec\\
					\\
					& 5. Database Server Pool Nonpaged Bytes & \\
					\\
					& 6. Database Server Page Faults/sec & \\
					\\
					& 7. Database Server Page File Bytes &\\
					\\
					& 8. Database Server Working Set &\\
					\\
					\midrule
					\textbf{Fit} &  85.20\% &  89.10\% \\
					\midrule
					\textbf{MAPE} & 15.75\% & 4.70\% \\
					\bottomrule             
				\end{tabular}%
			}
	\end{table}
	%\end{landscape}
	
	
	\begin{table}[tbh]
		\centering
			\caption{CloudStore: Prediction errors cross-environments}
			\label{resultRQ3}
			\begin{tabular}{c|c}
				\toprule
				\textbf{Training - Testing}  & \textbf{MAPE}\\  
				\midrule 
				Physical-Virtual       & 293.62\% \\
				Virtual - Physical      & 28.95\% \\
				Virtual - Physical \textbf{(after scaling)} & 276.04\% \\
				\bottomrule             
			\end{tabular}
	\end{table}
	
	
	
	%
	%\begin{table*}[thb!]
	%    \begin{center}
	%    \caption{Results}
	%    \label{tab:project_results}
	%            \begin{tabular}{l| c c r c || c c c || c c c}
	%            \toprule
	%            \textbf{}   & \thead{Release}  & \thead{\# of classes}   & \thead{SLOC} & \thead{\# of \\contributors}  & \thead{\# of \\comments}   & \thead{\# of \\comments \\after filtering} & \thead{\# of \\TD \\comments} & \thead{\% of \\Design \\Debt} & \textbf{\% of Requirement Debt} & \thead{\% of \\Other \\Debts}\\ 
	%            \midrule 
	%            \textbf{Ranking}            & 1.7.0    & 1,475 & 115,881 & 74  & 21,587 &   4,137 &    131 &  72.51  & 09.92  & 17.55 \\
	%            ArgoUML        & 0.34     & 2,609 & 176,839 & 87  & 67,716 &   9,548 &  1,413 &  56.68  & 29.08  & 14.22 \\
	%            Columba        & 1.4      & 1,711 & 100,200 & 9   & 33,895 &   6,478 &    204 &  61.76  & 21.07  & 17.15 \\
	%            EMF            & 2.4.1    & 1,458 & 228,191 & 30  & 25,229 &   4,401 &    104 &  75.00  & 15.38  & 09.61 \\
	%            Hibernate      & 3.3.2 GA & 1,356 & 173,467 & 226 & 11,630 &   2,968 &    472 &  75.21  & 13.55  & 11.22 \\
	%            JEdit          & 4.2      &   800 &  88,583 & 57  & 16,991 &  10,322 &    256 &  76.56  & 05.46  & 17.96 \\
	%            JFreeChart     & 1.0.19   & 1,065 & 132,296 & 19  & 23,474 &   4,423 &    209 &  88.03  & 07.17  & 04.78 \\
	%            Jmeter         & 2.10     & 1,181 &  81,307 & 33  & 20,084 &   8,162 &    374 &  84.49  & 05.61  & 09.89 \\
	%            JRuby          & 1.4.0    & 1,486 & 150,060 & 328 & 11,149 &   4,897 &    622 &  55.14  & 17.68  & 27.17 \\ 
	%            SQuirrel       & 3.0.3    & 3,108 & 215,234 & 46  & 27,474 &   7,230 &    286 &  73.07  & 17.48  & 09.44 \\ 
	%            \bottomrule             
	%        \end{tabular}
	%    \end{center}
	%\end{table*}
	
	


%We retested our virtual and physical environments with a same set of load. As this experiment was a replica of our previous research question, our approach was based on the same lines. After obtaining the performance metrics, we preprocessed our data where every record represented the load and performance metrics generated for a minute. We removed the first thirty entries, which represented first thirty minutes in our tests to remove any noise.  After the preprocessing phase we had three set of metrics for every environment. We labeled one of them as the \textit{baseline metrics} and the rest as \textit{reference metrics 1} and \textit{reference metrics 2}. Succeedingly, these six datasets were imported in R for further analysis.
%To draw conclusion and validate the findings, firstly we removed any metric that showed no variance. This is pursued by removing metrics which are highly correlated with other metrics in the same data set, for example \textit{Processor Time} and \textit{Privileged Time}. We, then, train our data bolstered by the generalized linear model. \textit{Stepwise Regression} is used to filter the metrics which do not contribute to our model. The metrics are added one by one to until no further contribution is made to the model.  \cite{RInAction}. We trained our model with the set of \textit{baseline metrics} and tested our model by predicting the load for the \textit{reference metrics 1} and \textit{reference metrics 2} . We trained iteratively on the rest of the sets of metrics and tested for the remaining two sets of metrics. We repeated this approach for both of our environments.
%We calculated the accuracy of our approach by the \textit{mean absolute percentage error} where the absolute difference of real and predicted values is divided by the real values followed by taking the mean. \cite{perfmon}

%\textbf{Results}:  We observed that \textit{MAPE} values for majority of our experiments run on physical environment fell under \textbf{30\%} while for the virtual the values were more than 200\%. 
%We also concluded that because of the presence of the noise and the irregular behavior of the virtual environment, the experiments did not produce exact similar set of results. The difference in behavior was based on inconsistent \textit{User Time} and \textit{DISC IO} operations. This lead our to our next question that whether the performance testing done in virtual environment can be calibrated and applied to the physical environment, as profoundly practiced in the field of performance testing.

	
	


\section{Discussion}
\label{sec:discussion}

In the previous section, we find that there is a discrepancy between performance testing results from the virtual and physical environments. However, such discrepancy can also be due to other factors such 1) the instability of the virtual environments, 2) the virtual machine that we used or 3) the different hardware resources on the virtual environments. Therefore, in this section, we examine the impact of such factors to better understand our results. 


\subsection{Instability of virtual environment}

\subsubsection{Motivation}

A major challenge in our case studies was to make our virtual environment stable. If the performance of virtual machines are unstable, the observed discrepancy in Section~\ref{sec:results} may be due to the instability of virtual environment. We address the hypothesis that if our virtual environment was unstable, then the performance testing results should not be repeatable and congruent between different runs in the same virtual environment. 

\subsubsection{Approach}

In order to study whether the virtual environment is stable, we repeat the same performance tests, twice, on the virtual environments for both subject systems. In total, we had results from three performance tests. We perform the data analysis in Section~\ref{sec:model} by building statistical models using performance metrics. %Table~\ref{tab:stabilityvm} shows the median absolute percentage error from building a model using one virtual environment and testing on another virtual environment. 
As the previously mentioned approach, we build a model based on one of the runs, serving as our training data for the model, and tested it on another run. In this case, we define external validation when a model is trained on a different run than it is tested on. We validate our model by predicting the throughput of a different run.  
 
 \subsubsection{Results}
Values(prediction error) closer to 0 indicate that our model was able to successfully explain the variation of the throughput of a different run. This also means that the external validation error closer to 1 or higher depicts an instability of the environment. We find that external validation error (0.04 and 0.13 for CloudStore and DS2) is almost as low as the internal validation error (0.03 and 0.09 for CloudStore and DS2). Such low error shows that the performance testing results from the virtual environments are rather stable. 

\subsection{Virtual machine software for the virtual environment}

\subsubsection{Motivation}

We also investigated the impact of choosing different virtual machine software on our experimental results. It can be argued that our chosen subject systems behave differently in another environment. In order to address the aforementioned hypothesis, we set up another virtual environment using VMWare (version 12) with the same allocated computing resources as when we set up Virtual Box.

\subsubsection{Approach}


We repeat the performance tests for both subject systems. We train statistical models on the performance testing results from VMWare and test on the results from both the original virtual environment data (Virtual Box) and the results from the physical environments. We could not apply the normalization by deviance for the data from VMWare since some of the significant metrics in the model have a median absolute deviance of 0, making the normalized metric value to be infinite (see Equation~\ref{equ:mad}). We only apply the normalization by load. 

 \subsubsection{Results}
The low percentage error when our model was tested on Virtual Box in Table~\ref{tab:vmware} shows that the performance testing results from the two different virtual machine software is similar. In addition, the high error when predicting with physical environment agrees with the results when testing with the performance testing results from the Virtual Box (see Table~\ref{tab:errors}). Such results show that the discrepancy observed during our experiment also exits with the virtual environments that are set up with VMWare.

\begin{table}[tbh]
	\centering
	\caption{Median absolute percentage error from building a model using VMWare data.}
	\label{tab:vmware}
		\begin{tabular}{|c||c|c|}
			\hline
			\multirow{2}{*}{\textbf{Validation type}} & \multicolumn{2}{c|}{\textbf{Median absolute percentage error}} \\ \cline{2-3} 
			& \textbf{CloudStore} & \textbf{DS2} \\ %\hline
			\midrule
			\midrule
			External validation with Virtual Box results& 0.07&0.10\\ \hline
%			External validation with physical normalization by deviance & 0.07 &0.06 \\ \hline
			External validation with physical normalization by load & 7.52& 1.63 \\ \hline
		\end{tabular}
\end{table}

\subsection{Resource Allocation}


\subsubsection{Motivation}

The third hypothesis about our study may be the result of our analysis is based on the resources allocated and the configuration of the virtual environment. As we already were working on an optimal point for our vritual environment to avoid system crashes, we only could increase the allocated resources in order to ensure the execution of the subject systems.


\subsubsection{Approach}
We increase the computing resources allocated to the virtual environments by increasing the CPU to be 3 cores and increasing the memory to be 5GB. We cannot allocate more resource to the virtual environment since we need to keep resources for the hosting OS. We train statistical models on the new performance testing results and tested it on the performance testing results from the physical environment. 


\subsubsection{Results}
Similar to the results shown in Table~\ref{tab:errors}, the prediction error is high when we normalize by load as per Equation~\ref{equ:mad} (1.57 for DS2 and 1.25 for CloudStore), while normalizing based on deviance can significantly reduce the error (0.09 for DS2 and 0.07 for CloudStore). By altering the resource allocated, such results show the minimal impact to our findings. Moreover, our results demonstrate the ability of reducing the discrepancy in performance testing results by using normalization based on deviance. 

	
\section{Threats to Validity}
%\vspace{-0.1cm}
\label{sec:threats}
This section discusses the threats to our validity.

\subsection{External validity.}
We chose two subject systems, CloudStore and DS2 for our study and two virtual machine software, VirtualBox and VMware. The two subject systems have years of history and prior performance engineering research has studied both systems~\cite{jackicsm2009,Nguyen:2012:ADP:2188286.2188344,tarekmsr16}. The virtual machine software that we used is widely used in practice. Nevertheless more case studies on other subject systems in other domains with other virtual machine software are needed to evaluate our findings. We also present our results based on our subject systems only and do not generalize for all the virtual machines.

%Also, we made sure that our virtual environment is set up exactly the same as our physical environment by keeping a constant checks aided by scripts. Having said that, this study can be boosted by additional subject systems being tested in other types of virtual environment. 

\subsection{Internal Validity.}
Our approach is based on the recorded performance metrics. The quality of recorded performance metrics can impact the internal validity of our study. Replicating our study by other performance monitoring tools, such as psutil~\cite{psutil} may address this threat. Even though we build a statistical model using performance metrics and system throughput, we do not assume that there is causal relationship. The use of statistical models merely aims to capture the relationship among multiple metrics. Similar approaches have been used in the prior studies~\cite{Cohen:2005:CIC:1095810.1095821, Shang:2015:ADP:2668930.2688052, xiong2013vperfguard}. 


%All of our models are dependent on the performance metrics' accuracy. Which means if the load on the server is beyond the capacity of the system to handle and builds up a queue, there is a possibility of noise sneaking in the recording process of the performance metrics. 
%The scaling approach we have adopted assumes that the for a particular metric, there exists alpha and beta values. However, if the metric value is constant it not possible to have an alpha and beta value associated with it in the model. Hence looking for an alternate scaling process is necessary for such a performance metric.
%We build performance regression models to compare the metrics from both of our environments. This models are accurate if there exists a high number of records for the performance counters. Additionally, we also assume that none of our dependent variable is correlated to the independent variable or vice-versa. 

\subsection{Construct Validity.}
We monitor the performance by recording performance metrics every 10 seconds and combine the performance metrics for every minute together as an average value. There may exist unfinished system requests when we record the system performance, leading to noise in our data. We choose a time interval (10 seconds) that is much higher than the response time of the requests (less than 0.1 second), in order to minimize the noise. Repeating our study by choosing other time interval sizes would address this threat. We exploit two approaches to normalize performance data from different environments. We also see that our {$R^2$} value is high. Although a higher {$R^2$} determines our model is accurate but it may also be an indication of overfit. There may exist other advance approaches to normalize performance data from heterogeneous environment. We plan to extend our study on other possible normalization approaches. There may exist other ways of examining performance testing results. We plan to extend our study by evaluating the discrepancy of using other ways of examining performance testing results in virtual and physical environments.



%We compared our distributions using Mann-Whitney \textit{U} Tests. As this test is commonly used to compare two distributions, other tests like T-test can also be used for this purpose. 
%We used R's \textit{step()} function to build our generalized linear regression models. This function automatically selects the variables contributing most to the models however the models might be different if they are only based on p-values. To address this, we plan to build models with a predefined threshold for the p-values.



\section{Conclusion}
%\vspace{-0.1cm}

\label{sec:conclusion}
Performance assurance activities are vital in ensuring software reliability. Virtual environments are often used to conduct performance tests. However, the discrepancy between performance testing results in virtual and physical environments are never evaluated. In this paper, we evaluate such discrepancy by conducting performance tests on two open source systems (DS2 and CloudStore) in both virtual and physical environments. By examine the performance testing results, we find that there exist discrepancy between performance testing results in virtual and physical environments when examining individual performance metrics, relationship among performance metrics and building statistical models from performance metrics, even after we normalize performance metrics across different environments. The major contribution of this paper includes: 
%\vspace{-0.15cm}
\begin{itemize} \itemsep -0.8pt 
	\item Our paper is the first research attempt to evaluate the discrepancy between performance testing results in virtual and physical environments.
	\item We find that relationships among I/O related metrics have large differences between virtual and physical environments.
	\item We find that normalizing performance metrics based on deviance may reduce the discrepancy. Practitioners may exploit such normalization techniques when analyzing performance testing results from virtual environments.
\end{itemize}
%\vspace{-0.15cm}
Our results highlight the needs of awareness of discrepancy between performance testing results in virtual and physical environments, for both practitioners and researchers. Future research effort may focus on minimizing such discrepancy in order to improve the use of virtual environments in performance engineering and reliability assurance activities



%Our paper magnifies the impact of performance in difference environments. Additionally, we observed that the performance metrics from different environment do not belong to the same distribution. We tried to mitigate this impact by using scaled metrics in our performance regression models. As a result, the percentage error dropped drastically for one of our subject systems are increased for the other. We concluded that scaling may not apply to every subject system. We plan to see in what ways we can leverage the metrics from the virtual environment and use them for prediction of the metrics in the physical environment. We also plan to investigate the injection of regression in the system being hosted in a virtual environment will behave similar to that under regression in a physical environment. 

%\section*{Acknowledgment}

%The author thanks Emad Shihab and Thanh H$.$D$.$ Nguyen for their suggestions on the early version of this work. The author appreciates the generosity of the Performance Engineering team at Research In Motion (RIM). Working with the team as an embedded researcher, the author has gained an appreciation of the current practice of mining large-scale software logs.

\bibliographystyle{spmpsci}
\bibliography{references}
%\balance

\end{document}


